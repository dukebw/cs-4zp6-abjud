\documentclass{scrreprt}

\usepackage{xcolor} % for different colour comments
\usepackage{tabto}
\usepackage{mdframed}
\mdfsetup{nobreak=true}
\usepackage{xkeyval}
\usepackage{tabularx}
\usepackage{booktabs}
\usepackage{hyperref}
\hypersetup{
    colorlinks,
    citecolor=black,
    filecolor=black,
    linkcolor=red,
    urlcolor=blue
}
\usepackage[skip=2pt, labelfont=bf]{caption}
\usepackage{titlesec}
\usepackage{graphicx}
\usepackage[section]{placeins}
\graphicspath{ {image/} }

\titleformat{\paragraph}
{\normalfont\normalsize\bfseries}{\theparagraph}{1em}{}
\titlespacing*{\paragraph}
{0pt}{3.25ex plus 1ex minus .2ex}{1.5ex plus .2ex}


%% Comments
\newif\ifcomments\commentstrue

\ifcomments
\newcommand{\authornote}[3]{\textcolor{#1}{[#3 ---#2]}}
\newcommand{\todo}[1]{\textcolor{red}{[TODO: #1]}}
\else
\newcommand{\authornote}[3]{}
\newcommand{\todo}[1]{}
\fi

\newcommand{\wss}[1]{\authornote{magenta}{SS}{#1}}
\newcommand{\ds}[1]{\authornote{blue}{DS}{#1}}


%% The following are used for pretty printing of events and requirements
\makeatletter

\define@cmdkey      [TP] {test}     {name}       {}
\define@cmdkey      [TP] {test}     {desc}       {}
\define@cmdkey      [TP] {test}     {type}       {}
\define@cmdkey      [TP] {test}     {init}       {}
\define@cmdkey      [TP] {test}     {input}      {}
\define@cmdkey      [TP] {test}     {output}     {}
\define@cmdkey      [TP] {test}     {pass}       {}
\define@cmdkey      [TP] {test}     {user}       {}
\define@cmdkey      [TP] {test}     {reqnum}     {}


\newcommand{\getCurrentSectionNumber}{%
  \ifnum\c@section=0 %
  \thechapter
  \else
  \ifnum\c@subsection=0 %
  \thesection
  \else
  \ifnum\c@subsubsection=0 %
  \thesubsection
  \else
  \thesubsubsection
  \fi
  \fi
  \fi
}

\newcounter{TestNum}

\@addtoreset{TestNum}{section}
\@addtoreset{TestNum}{subsection}
\@addtoreset{TestNum}{subsubsection}

\newcommand{\testauto}[1]{
\setkeys[TP]{test}{#1}
\refstepcounter{TestNum}
\begin{mdframed}[linewidth=1pt]
\begin{tabularx}{\textwidth}{@{}p{3cm}X@{}}
{\bf Test \getCurrentSectionNumber.\theTestNum:} & {\bf \cmdTP@test@name}\\[\baselineskip]
{\bf Description:} & \cmdTP@test@desc\\[0.5\baselineskip]
{\bf Type:} & \cmdTP@test@type\\[0.5\baselineskip]
{\bf Initial State:} & \cmdTP@test@init\\[0.5\baselineskip]
{\bf Input:} & \cmdTP@test@input\\[0.5\baselineskip]
{\bf Output:} & \cmdTP@test@output\\[0.5\baselineskip]
{\bf Pass:} & \cmdTP@test@pass\\[0.5\baselineskip]
{\bf Req. \#:} & \cmdTP@test@reqnum
\end{tabularx}
\end{mdframed}
}

\newcommand{\testmanual}[1]{
\setkeys[TP]{test}{#1}
\refstepcounter{TestNum}
\begin{mdframed}[linewidth=1pt]
\begin{tabularx}{\textwidth}{@{}p{3cm}X@{}}
{\bf Test \getCurrentSectionNumber.\theTestNum:} & {\bf \cmdTP@test@name}\\[\baselineskip]
{\bf Description:} & \cmdTP@test@desc\\[0.5\baselineskip]
{\bf Type:} & \cmdTP@test@type\\[0.5\baselineskip]
{\bf Testers:} & \cmdTP@test@user\\[0.5\baselineskip]
{\bf Pass:} & \cmdTP@test@pass\\[0.5\baselineskip]
{\bf Req. \#:} & \cmdTP@test@reqnum
\end{tabularx}
\end{mdframed}
}

\makeatother

\newcommand{\ZtoT}{
\begin{tabularx}{3.85cm}{@{}p{0.35cm}p{0.35cm}p{0.35cm}p{0.35cm}p{0.35cm}p{0.35cm}p{0.35cm}p{0.35cm}p{0.35cm}p{0.35cm}p{0.35cm}@{}}
0 & 1 & 2 & 3 & 4 & 5 & 6 & 7 & 8 & 9 & 10
\end{tabularx}
}

\begin{document}
\title{\bf Text to Motion Database\\[\baselineskip]\Large Test Plan}
\author{Brendan Duke\\Andrew Kohnen\\Udip Patel\\David Pitkanen\\Jordan Viveiros}
\date{\today}
	
\maketitle

\pagenumbering{roman}
\tableofcontents
\listoftables
\listoffigures


\begin{table}[bp]
\caption*{\bf Revision History}
\begin{tabularx}{\textwidth}{p{3.5cm}p{2cm}X}
\toprule {\bf Date} & {\bf Version} & {\bf Notes}\\
\midrule
October 25, 2015 & 1.0 & Created document\\
October 31, 2015 & 1.1 & Major additions to all sections\\
November 1, 2015 & 1.2 & Final version for rev 0\\
\bottomrule
\end{tabularx}
\end{table}

\newpage

\pagenumbering{arabic}

\chapter{Overview}

\section{Test Case Format}

\section{Automated Testing}

\subsection{Testing Tools}

\section{Manual Testing}

\subsection{User Experience Testing}

\section{List of Constants}

\chapter{Proof of Concept Testing}

\section{Significant Risks}

\section{Demonstration Plan}

\section{Proof of Concept Test}

\chapter{System Testing}

\chapter{Constraints Testing}

\section{Solution Constraints Testing}

\testmanual{
        name = {Deep Learning Methods Test},
        desc = {Test whether the human pose estimation component of the
                software uses modern deep learning methods.},
        type = {Manual},
        user = {Supervisor (Dr. Taylor)},
        pass = {Dr. Taylor should confirm that the deep learning methods used
                are satisfactory and relevant to current research, with a yes
                or no reponse.},
        reqnum = {1}
}

\testauto{
        name = {Standard Data Format Test},
        desc = {Tests whether the human pose data format used in the project is
                standard, and compatible with existing software libraries.},
        type = {Automated},
        init = {Initialize database query interface.},
        input = {Random ID of a record, containing human pose data, in the
                 database.},
        output = {Tuple containing data in HDF5 format.},
        pass = {The human pose datum should be parseable by an existing HDF5
                data library.},
        reqnum = {2}
}

\testauto{
        name = {Linux Platform Build and Run Test},
        desc = {Confirms that all nightly build tests, as well as the automated
                test suite, are working under Linux.},
        type = {Automated},
        init = {None (build test).},
        input = {Commands to begin build and run sequence.},
        output = {Compile and run success, or errors.},
        pass = {Compile and run success.},
        reqnum = {3}
}

\testauto{
        name = {Python API Hook Testing},
        desc = {Confirms that major module interfaces, such as the image pose
                estimation interface, and database query interface, have working Python
                hooks.},
        type = {Automated},
        init = {Initialization specific to each module interface under test.},
        input = {Valid parameters for each module interface, written in Python.},
        output = {Expected success-case outputs for each module interface,
                  written in Python.},
        pass = {Interface calls completed without error, and returned their
                expected outputs.},
        reqnum = {4}
}

\chapter{Functional Requirements Testing}

\testauto{
        name = {Supported Video Encodings Test},
        desc = {Tests whether the ReadFrames API is able to decode MP4, MP2 and
                AAC video files.},
        type = {Automated},
        init = {Call read frames initialization procedure.},
        input = {30 second MP4 video file at 30 FPS.},
        output = {A set of 900 $(30\times30)$ frames.},
        pass = {The 900 frames match a set of 900 expected frames from a reference
                frame-reading system.},
        reqnum = {7}
}

\testauto{
        name = {Frame Reading Timestamp Accuracy Test},
        desc = {Tests whether the timestamps on the frames returned by the
                ReadFrames API match their temporal position in the original video
                stream.},
        type = {Automated},
        init = {Call read frames initialization procedure.},
        input = {30 second MP4 video file at 30 FPS.},
        output = {A set of 900 $(30\times30)$ frames, which include timestamps.},
        pass = {The timestamps on the 900 frames match a set of timestamps on a
                test vector of expected timestamps for the 900 frames.},
        reqnum = {8}
}

\testmanual{
        name = {Video Human Pose Estimation Data Quality Test},
        desc = {Test to ensure the data quality produced by the human pose
                estimator component. A set of Charades videos will be processed
                by the human pose estimator, and skeleton animations
                corresponding to the generated human pose data will be
                created (this is a scoped part of the software pipeline). A
                double-blind test will be ran, where testers will be shown
                random mixed sets of the skeleton animations produced by
                McMaster Text to Motion, together with skeletons from actual
                motion capture data coming from CMU's motion capture lab.
                Testers will indicate whether they think the motion capture
                data came from actual motion capture, or from the pose
                estimation software.},
        type = {Manual},
        user = {Testing Group},
        pass = {Within a 5\% confidence interval, the McMaster Text to Motion
                skeletons will be indicated as being actual motion capture data
                with the same probability that the CMU motion capture skeletons
                are indicated as being actual motion capture data.},
        reqnum = {8}
}

\testauto{
        name = {Database Output Full Range Coverage Test},
        desc = {Tests whether the range of the text-to-motion database search
                is equal to the entire set of data stored in the database.},
        type = {Automated},
        init = {Initialize database-query and full text search module interfaces.
                Populate database with Charades data.},
        input = {A random matching keyword from the text description of each
                 video (acquired automatically).},
        output = {A set of video-pose data from the database that should
                  include the original datum that the input keyword was taken from.},
        pass = {The returned set of data contains the original video record.},
        reqnum = {9}
}

\testauto{
        name = {Database No False Positives Test},
        desc = {Tests whether the results retrieved from text searches of the
                database contain any false positives, i.e. results whose text
                descriptions do not contain any of the searched keywords.},
        type = {Automated},
        init = {Initialize database-query and full text search module
                interfaces. Populate database with Charades data.},
        input = {For each video, a random set of keywords not in that video's
                 text description.},
        output = {A set of video-pose entries.},
        pass = {The output set of data should not contain the original video
                that was chosen to be outside the subset of the output range for this
                input.},
        reqnum = {10}
}

\testauto{
        name = {Full Text Search Order by Relevance Test},
        desc = {A test of whether the full text search interface is returning a
                set of entries that are ordered by relevance to the search keywords.},
        type = {Automated},
        init = {Initialize database-query and full text search module
                interfaces. Populate database with Charades data.},
        input = {A random set of search keywords, drawn automatically from the
                 set of text descriptions in the database.},
        output = {A set of entries in the database, in some order.},
        pass = {The output set of entries should be randomly ordered and input
                to a reference full text search engine, which will produce an expecteed
                ordering by relevance. A statistical test of the similarity of
                the McMaster Text to Motion ordering and the reference ordering
                should be done, and the McMaster Text to Motion ordering should
                be expected to be the same within a 5\% confidence interval.},
        reqnum = {11}
}

\chapter{Non-Functional Requirements Testing}

\section{Look and Feel Requirements Testing}

\testmanual{
        name = {Colour Scheme Test},
        desc = {Test user satisfaction of the web interface colour scheme.},
        type = {Manual},
        user = {Testing Group},
        pass = {On a one to ten scale, the average user rating is above six.},
        reqnum = {12}
}

\chapter{Timeline}

\chapter{Appendix A:  Testing Survey}

\end{document}
