% Original work copyright 2014 Jean-Philippe Eisenbarth
% Modified work copyright 2016 of Brendan Duke and Jordan Viveiros.

% This program is free software: you can 
% redistribute it and/or modify it under the terms of the GNU General Public 
% License as published by the Free Software Foundation, either version 3 of the 
% License, or (at your option) any later version.
% This program is distributed in the hope that it will be useful,but WITHOUT ANY 
% WARRANTY; without even the implied warranty of MERCHANTABILITY or FITNESS FOR A 
% PARTICULAR PURPOSE. See the GNU General Public License for more details.
% You should have received a copy of the GNU General Public License along with 
% this program.  If not, see <http://www.gnu.org/licenses/>.

% Based on the code of Yiannis Lazarides
% http://tex.stackexchange.com/questions/42602/software-requirements-specification-with-latex
% http://tex.stackexchange.com/users/963/yiannis-lazarides
% Also based on the template of Karl E. Wiegers
% http://www.se.rit.edu/~emad/teaching/slides/srs_template_sep14.pdf
% http://karlwiegers.com
\documentclass{scrreprt}
\usepackage{listings}
\usepackage{underscore}
\usepackage[bookmarks=true]{hyperref}
\usepackage[utf8]{inputenc}
\usepackage[english]{babel}
\hypersetup{
    bookmarks=false,    % show bookmarks bar?
    pdftitle={Software Requirement Specification},    % title
    pdfauthor={Jean-Philippe Eisenbarth},                     % author
    pdfsubject={TeX and LaTeX},                        % subject of the document
    pdfkeywords={TeX, LaTeX, graphics, images}, % list of keywords
    colorlinks=true,       % false: boxed links; true: colored links
    linkcolor=blue,       % color of internal links
    citecolor=black,       % color of links to bibliography
    filecolor=black,        % color of file links
    urlcolor=purple,        % color of external links
    linktoc=page            % only page is linked
}%
\def\myversion{0.0 }
\date{}
%\title
\usepackage{hyperref}
\begin{document}

\begin{flushright}
    \rule{16cm}{5pt}\vskip1cm
    \begin{bfseries}
        \Huge{SOFTWARE REQUIREMENTS\\ SPECIFICATION}\\
        \vspace{1.9cm}
        for\\
        \vspace{1.9cm}
        CS 4ZP6 Capstone Project\\
        \vspace{1.9cm}
        \LARGE{Version \myversion}\\
        \vspace{1.9cm}
        Prepared by Brendan Duke, Andrew Kohnen, Udip Patel, David Pitkanen, Jordan Viveiros\\
        \vspace{1.9cm}
        McMaster Text to Motion Database\\
        \vspace{1.9cm}
        \today\\
    \end{bfseries}
\end{flushright}

\tableofcontents

\chapter*{Revision History}

\begin{center}
    \begin{tabular}{|c|c|c|c|}
        \hline
	    Name & Date & Reason For Changes & Version\\
        \hline
	    Brendan Duke & Oct. 7th, 2016 & Initial Version & 0.0\\
        \hline
    \end{tabular}
\end{center}

\newcounter{RequirementNumber}

\chapter{Project Drivers}

\section{The Purpose of the Project}

\subsection{The User Business or Background of the Project Effort}

\subsection{Goals of the Project}

\section{The Client, the Customer, and Other Stakeholders}

\subsection{The Client}

\subsection{The Customer}

\subsection{Other Stakeholders}

\section{Users of the Product}

\subsection{The Hands-on Users of the Product}

\subsection{Priorities Assigned to Users}

\subsection{User Participation}

\subsection{Maintenance Users and Service Technicians}

\chapter{Project Constraints}

\section{Mandated Constraints}

\subsection{Solution Constraints}

\begin{center}
    \begin{tabular}{ | p{4cm} | p{10cm} |}
    \hline
    Requirement Number & \theRequirementNumber \\ \hline
    Requirement Type & 4a. Solution Constraint \\ \hline
    Event/Use Case Numbers & Entire product. \\ \hline
    Description & The Text-to-Motion Software Suite must run under Linux.\\ \hline
    Rationale & Linux is the operating system used by the Guelph Machine
            Learning research lab, and also the most commonly used operating
            system in the research community.\\ \hline
    Originator & Dr. Graham Taylor \\ \hline
    Fit Criterion & Automated builds and testing should pass on popular Linux
            distributions: Ubuntu, Fedora and RHEL.\\ \hline
    Customer Satisfaction & 5 \\ \hline
    Customer Dissatisfaction & 5 \\ \hline
    Priority & High priority. \\ \hline
    Conflicts & None. \\ \hline
    Supporting Materials & None. \\ \hline
    History & Created September 26th, 2016.\\
\hline
    \end{tabular}
\end{center}
\stepcounter{RequirementNumber}

\begin{center}
    \begin{tabular}{ | p{4cm} | p{10cm} |}
    \hline
    Requirement Number & \theRequirementNumber \\ \hline
    Requirement Type & 4a. Solution Constraint \\ \hline
    Event/Use Case Numbers & Entire product. \\ \hline
    Description & Major APIs to the Text-to-Motion database must be accessible
            from the Python programming language.\\ \hline
    Rationale & Python is a popular, easy-to-use, and quick-to-prototype
            language, and is therefore one of the most favoured programming
            languages among the Machine Learning research community.\\ \hline
    Originator & Dr. Graham Taylor \\ \hline
    Fit Criterion & There must be hooks to all major interfaces written in
            Python, and there must be tests that are directly testing the
            Python interfaces.\\ \hline
    Customer Satisfaction & 5 \\ \hline
    Customer Dissatisfaction & 5 \\ \hline
    Priority & High priority. \\ \hline
    Conflicts & None. \\ \hline
    Supporting Materials & None. \\ \hline
    History & Created September 26th, 2016.\\
\hline
    \end{tabular}
\end{center}
\stepcounter{RequirementNumber}

\subsection{Implementation Environment of the Current System}

\subsection{Partner or Collaborative Applications}

\subsection{Off-the-Shelf Software}

\subsection{Anticipated Worklace Environment}

\subsection{Schedule Constraints}

\subsection{Budget Constraints}

\section{Naming Conventions and Definitions}

\subsection{Definitions of All Terms, Including Acronyms, Used in the Project}

\subsection{Data Dictionary for any Included Models}

\section{Relevant Facts and Assumptions}

\subsection{Facts}

\subsection{Assumptions}

\chapter{Functional Requirements}

\section{The Scope of the Work}

\subsection{The Current Situation}

\subsection{The Context of the Work}

\subsection{Work Partitioning}

\section{The Scope of the Product}

\subsection{Product Boundary}

\subsection{Product Use-case List}

\subsection{Individual Product Use Cases}

\section{Functional and Data Requirements}

\subsection{Functional Requirements}

\begin{center}
    \begin{tabular}{ | p{4cm} | p{10cm} |}
    \hline
    Requirement Number & \theRequirementNumber \\ \hline
    Requirement Type & 9a. Functional Requirement \\ \hline
    Event/Use Case Numbers & \\ \hline
    Description & The text-to-motion software suite will provide an API to read
            individual frames in RGB format from a video stream. At least MP4,
            MP2 and AAC must be supported.\\ \hline
    Rationale & Researchers may wish to do their own processing on RGB frames
            before feeding those frames into the human pose estimation
            module.\\ \hline
    Originator & Brendan Duke. \\ \hline
    Fit Criterion & For a given set of test video streams, the frame-capture
            API must produce RGB frames identical to known reference frames.\\
            \hline
    Customer Satisfaction & 3 \\ \hline
    Customer Dissatisfaction & 3 \\ \hline
    Priority & Moderate priority. \\ \hline
    Conflicts & None. \\ \hline
    Supporting Materials & None. \\ \hline
    History & Created October 5th, 2016.\\
\hline
    \end{tabular}
\end{center}
\stepcounter{RequirementNumber}

\subsection{Data Requirements}

\chapter{Nonfunctional Requirements}

\section{Look and Feel Requirements}

\subsection{Appearance Requirements}

\subsection{Style Requirements}

\section{Usability and Humanity Requirements}

\subsection{Ease of Use Requirements}

\subsection{Personalization and Internationalization Requirements}

\subsection{Learning Requirements}

\subsection{Understandability and Politeness Requirements}

\subsection{Accessibility Requirements}

\section{Performance Requirements}

\subsection{Speed and Latency Requirements}

\subsection{Safety-Critical Requirements}

\subsection{Precision or Accuracy Requirements}

\subsection{Reliability and Availability Requirements}

\subsection{Robustness or Fault-Tolerance Requirements}

\subsection{Capacity Requirements}

\subsection{Scaling of Extensibility Requirements}

\subsection{Longevity Requirements}

\section{Operational and Environmental Requirements}

\subsection{Expected Physical Environment}

\subsection{Requirements for Interfacing with Adjacent Systems}

\subsection{Productization Requirements}

\subsection{Release Requirements}

\section{Maintainability and Support Requirements}

\subsection{Maintenance Requirements}

\subsection{Supportability Requirements}

\subsection{Adaptability Requirements}

\section{Security Requirements}

\subsection{Access Requirements}

\subsection{Integrity Requirements}

\subsection{Privacy Requirements}

\subsection{Audit Requirements}

\subsection{Immunity Requirements}

\section{Cultural and Political Requirements}

\subsection{Cultural Requirements}

\subsection{Political Requirements}

\section{Legal Requirements}

\subsection{Compliance Requirements}

\subsection{Standards Requirements}

\chapter{Project Issues}

\section{Open Issues}

\section{Off-the-Shelf Solutions}

\subsection{Ready-Made Products}

\subsection{Reusable Components}

\subsection{Products That Can Be Copied}

\section{New Problems}

\subsection{Effects on the Current Environment}

\subsection{Effects on the Installed Systems}

\subsection{Potential User Problems}

\subsection{Limitations in the Anticipated Implementation Environment That May
            Inhibit the New Product}

\subsection{Follow-Up Problems}

\section{Tasks}

\subsection{Project Planning}

\subsection{Planning of the Development Phases}

\section{Migration to the New Product}

\subsection{Requirements for Migration of the New Product}

\subsection{Data That Has to Be Modified or Translated for the New
            System}

\section{Risks}

\section{Costs}

\section{User Documentation and Training}

\subsection{User Documentation Requirements}

\subsection{Training Requirements}

\section{Waiting Room}

\section{Ideas for Solutions}

\chapter{Appendix}

\section{Appendix A: Glossary}
%see https://en.wikibooks.org/wiki/LaTeX/Glossary

\section{Appendix B: Analysis Models}

\section{Appendix C: To Be Determined List}

\end{document}
