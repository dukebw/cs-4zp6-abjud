% Original work copyright 2014 Jean-Philippe Eisenbarth
% Modified work copyright 2016 of Brendan Duke and Jordan Viveiros.

% This program is free software: you can 
% redistribute it and/or modify it under the terms of the GNU General Public 
% License as published by the Free Software Foundation, either version 3 of the 
% License, or (at your option) any later version.
% This program is distributed in the hope that it will be useful,but WITHOUT ANY 
% WARRANTY; without even the implied warranty of MERCHANTABILITY or FITNESS FOR A 
% PARTICULAR PURPOSE. See the GNU General Public License for more details.
% You should have received a copy of the GNU General Public License along with 
% this program.  If not, see <http://www.gnu.org/licenses/>.

% Based on the code of Yiannis Lazarides
% http://tex.stackexchange.com/questions/42602/software-requirements-specification-with-latex
% http://tex.stackexchange.com/users/963/yiannis-lazarides
% Also based on the template of Karl E. Wiegers
% http://www.se.rit.edu/~emad/teaching/slides/srs_template_sep14.pdf
% http://karlwiegers.com
\documentclass{scrreprt}
\usepackage{listings}
\usepackage{underscore}
\usepackage[bookmarks=true]{hyperref}
\usepackage[utf8]{inputenc}
\usepackage[english]{babel}
\usepackage{xcolor}
\usepackage{indentfirst}
\usepackage[section]{placeins}
\hypersetup{
    bookmarks=false,    % show bookmarks bar?
    pdftitle={Software Requirement Specification},    % title
    pdfauthor={Jean-Philippe Eisenbarth},                     % author
    pdfsubject={TeX and LaTeX},                        % subject of the document
    pdfkeywords={TeX, LaTeX, graphics, images}, % list of keywords
    colorlinks=true,       % false: boxed links; true: colored links
    linkcolor=blue,       % color of internal links
    citecolor=black,       % color of links to bibliography
    filecolor=black,        % color of file links
    urlcolor=purple,        % color of external links
    linktoc=page            % only page is linked
}%
\def\myversion{0.0 }
\date{}
%\title
\usepackage{hyperref}
\begin{document}

\begin{flushright}
    \rule{16cm}{5pt}\vskip1cm
    \begin{bfseries}
        \Huge{SOFTWARE REQUIREMENTS\\ SPECIFICATION}\\
        \vspace{1.9cm}
        for\\
        \vspace{1.9cm}
        CS 4ZP6 Capstone Project\\
        \vspace{1.9cm}
        \LARGE{Version \myversion}\\
        \vspace{1.9cm}
        Prepared by Brendan Duke, Andrew Kohnen, Udip Patel, David Pitkanen, Jordan Viveiros\\
        \vspace{1.9cm}
        McMaster Text to Motion Database\\
        \vspace{1.9cm}
        \today\\
    \end{bfseries}
\end{flushright}

\tableofcontents

\chapter*{Revision History}

\begin{center}
    \begin{tabular}{|c|c|c|c|}
        \hline
            Name & Date & Reason For Changes & Version\\
        \hline
	    Brendan Duke & Oct. 7th, 2016 & Initial Version & 0.0\\
        \hline
    \end{tabular}
\end{center}

\newcounter{RequirementNumber}

\chapter{Project Drivers}

\section{The Purpose of the Project}

\subsection{The User Business or Background of the Project Effort}

With the current advancement of deep learning architectures, specifically the
use of Recurrent Neural Networks used for sequential data processing like
natural language or pose estimates. Using a Recurrent Neural Networks is made
more available through the creation of large databases that contain video with
descriptive labeling and annotations like MovieQA, Charades, or MSR-VTT.
Expanding on these technologies can be utilized to build a system for
"Computational Storytelling" that takes in a short story of 5 lines and outputs
an animated video between an AI and human director.

\subsection{Goals of the Project}

The goal of this project is to support a text to motion subcomponent of a
larger collaboration between the University of Guelph, SRI, and other
institutions. The creation of a database, website and manipulation of already
established pose estimation software will be required. Creating this database
and website will allow the larger text to motion project use the relationship
developed through the pose estimation in order to provide a pose and word
pairing for animating.

\section{The Client, the Customer, and Other Stakeholders}

The current stakeholders in this project are the:
    - Supervisors of the project (Dr. He, Dr. Taylor)
    - University of Guelph "text to motion" research group
    - The McMaster Capstone group

\subsection{The Client}

The current clients for this project are Dr. Taylor and his graduate student
Thor Jonsson. Dr. Taylor is the primary driver to develop a website and
database where annotated motion information could be generated and pulled from
as a growth point into the larger text to motion project. They will be using
the database to create Recurrent Neural Networks that will pair actions and
their pose found within the database to words or combinations found in the
input story.

\subsection{The Customer}

The customers are included within the clients since building this database and
website combination will be utilized by Dr. Taylors research team and their
external partners. In addition to Dr. Taylor and his research team this project
would appeal to anyone that needed a pairing of actions and pose estimations as
the website would be readily available to others.

\subsection{Other Stakeholders}

In addition to the stakeholders listed above the success of the project could
bring in additional stakeholders that are interested in the method of pose
estimation, database information or want to use this information for a project
of their own. This could be another research group or deep learning framework
that is looking to expand.

\section{Users of the Product}

\subsection{The Hands-on Users of the Product}

The primary users of this project will be Dr. Taylor and his research team as
they need this data in order to complete the Computational Storytelling. They
are also comfortable with the process of deep learning and would provide input
on how the pairings should be set up.

\subsection{Priorities Assigned to Users}

\subsection{User Participation}

\subsection{Maintenance Users and Service Technicians}

\chapter{Project Constraints}

\section{Mandated Constraints}

\subsection{Solution Constraints}

\begin{center}
    \begin{tabular}{ | p{4cm} | p{10cm} |}
    \hline
    Constraint Number & \theRequirementNumber \\
    Constraint Type & 4a. Solution Constraint \\
    Event/Use Case Numbers & Entire product. \\
    Description & The Text-to-Motion Software Suite must run under Linux.\\
    Rationale & Linux is the operating system used by the Guelph Machine
            Learning research lab, and also the most commonly used operating
            system in the research community.\\
    Originator & Dr. Graham Taylor \\
    Fit Criterion & Automated builds and testing should pass on popular Linux
            distributions: Ubuntu, Fedora and RHEL.\\
    Customer Satisfaction & 5 \\
    Customer Dissatisfaction & 5 \\
    Priority & High priority. \\
    Conflicts & None. \\
    Supporting Materials & None. \\
    History & Created September 26th, 2016.\\
\hline
    \end{tabular}
\end{center}
\stepcounter{RequirementNumber}

\begin{center}
    \begin{tabular}{ | p{4cm} | p{10cm} |}
    \hline
    Constraint Number & \theRequirementNumber \\
    Constraint Type & 4a. Solution Constraint \\
    Event/Use Case Numbers & Entire product. \\
    Description & Major APIs to the Text-to-Motion database must be accessible
            from the Python programming language.\\
    Rationale & Python is a popular, easy-to-use, and quick-to-prototype
            language, and is therefore one of the most favoured programming
            languages among the Machine Learning research community.\\
    Originator & Dr. Graham Taylor \\
    Fit Criterion & There must be hooks to all major interfaces written in
            Python, and there must be tests that are directly testing the
            Python interfaces.\\
    Customer Satisfaction & 5 \\
    Customer Dissatisfaction & 5 \\
    Priority & High priority. \\
    Conflicts & None. \\
    Supporting Materials & None. \\
    History & Created September 26th, 2016.\\
\hline
    \end{tabular}
\end{center}
\stepcounter{RequirementNumber}

\begin{center}
    \begin{tabular}{ | p{4cm} | p{10cm} |}
    \hline
    Constraint Number & \theRequirementNumber \\
    Constraint Type & 4a. Solution Constraint \\
    Event/Use Case Numbers & Human Pose Estimation Event. \\
    Description & The human pose estimation component should use deep learning
            methods.\\
    Rationale & This constraint is to allow Dr. Taylor's group to integrate the
            software into their existing text-to-motion pipeline\\
    Originator & Dr. Graham Taylor \\
    Fit Criterion & Dr. Taylor should confirm that the deep learning methods
            used in the human pose estimator are satisfactory.\\
    Customer Satisfaction & 5 \\
    Customer Dissatisfaction & 4 \\
    Priority & High priority. \\
    Conflicts & None. \\
    Supporting Materials & None. \\
    History & Created September 26th, 2016.\\
\hline
    \end{tabular}
\end{center}
\newcounter{HumanPoseDeepLearningConstraint}
\setcounter{HumanPoseDeepLearningConstraint}{\theRequirementNumber}
\stepcounter{RequirementNumber}

\begin{center}
    \begin{tabular}{ | p{4cm} | p{10cm} |}
    \hline
    Constraint Number & \theRequirementNumber \\
    Constraint Type & 4a. Solution Constraint \\
    Event/Use Case Numbers & Entire product. \\
    Description & The project must be completed by April 5th, 2017.\\
    Rationale & The project is part of the CS 4ZP6 Capstone Project course.\\
    Originator & Dr. He \\
    Fit Criterion & All documentation, testing and implementatoin must be
            completed and checked in to GitHub by April 5th, 2017.\\
    Customer Satisfaction & 5 \\
    Customer Dissatisfaction & 5 \\
    Priority & High priority. \\
    Conflicts & None. \\
    Supporting Materials & None. \\
    History & Created September 21st, 2016.\\
\hline
    \end{tabular}
\end{center}
\stepcounter{RequirementNumber}

\subsection{Implementation Environment of the Current System}

\subsection{Partner or Collaborative Applications}

\subsection{Off-the-Shelf Software}

\subsection{Anticipated Worklace Environment}

\subsection{Schedule Constraints}

\subsection{Budget Constraints}

\section{Naming Conventions and Definitions}

\subsection{Definitions of All Terms, Including Acronyms, Used in the Project}

\textbf{The Project} when used, is referring to the McMaster Text to Motion
Database project. The project aims to generate a database of human pose
estimation model information that is linked to videos of human motion
containing rich text annotations.

\textbf{Human Pose Estimation} is the process of estimating the configuration,
or pose, of the body based on a single still image or a sequence of images that
comprise a video. Human pose estimation may find the chin, radius, humerus, and
other bone and joint positions.

\textbf{Feedforward Neural Networks} are artifical neural networks where
connections between the units do \textit{not} form a cycle). They are the
simplest type of neural network, because information moves in only one
direction.

\textbf{ConvNets} or \textbf{Convolutional Neural Networks} are a type of
feed-forward artificial neural network. ConvNets are inspired by the visual
cortex and are commonly used in visual recognition applications.

\textbf{RNNs} or \textbf{Recurrent Neural Networks} are a class of artificial
neural networks where units form a directed cycle, in contrast with
feed-forward neural networks.

\textbf{Deep Belief Networks} are a type of deep neural network composed of
multiple layers of "hidden units" (variables that are not observable), with
connections between layers but not between units of a given layer.

\subsection{Data Dictionary for any Included Models}

\section{Relevant Facts and Assumptions}

\subsection{Facts}

\subsection{Assumptions}

\chapter{Functional Requirements}

\section{The Scope of the Work}

\subsection{The Current Situation}

There is a large amount of existing research into human pose estimation, which
this project will leverage. Based on constraint
\theHumanPoseDeepLearningConstraint, we focus on existing solutions that use
deep learning methods.

\cite{DBLP:journals/corr/PfisterCZ15} present a ConvNet architecture for human
pose estimation from videos, which is able to benefit from temporal context
across multiple frames using optical flow. This work is focused on upper-body
human pose estimation only.

\cite{DBLP:journals/corr/BelagiannisZ16} propose a ConvNet model for predicting
2D human body poses in an image. This model is able to achieve state-of-the-art
results using a simple architecture, and draws on the work done in
\cite{DBLP:journals/corr/PfisterCZ15}.

\cite{DBLP:journals/corr/WeiRKS16} introduces \textit{Convolutional Pose
Machines (CPMs)} for pose estimation in images. CPMs consist of a sequence of
ConvNets that iteratively produce 2D belief maps.

\subsection{The Context of the Work}

\subsection{Work Partitioning}

\begin{table}
\caption{Business Event List}
\begin{center}
    \begin{tabular}{ | p{5cm} | p{5cm} | p{5cm} |}
    \hline
    Event Name & Input and Output & Summary \\
    \hline
    Web Interface Skeleton Overlay
            & \textbf{IN}: An image or video with humans in it.\newline
            \textbf{OUT}: The same image or video, with a skeleton overlaid on
            top of all humans indicating their bone and joint positions.
            & Allow users to observe the human pose estimation component in
            real time through a web interface.\\
    Web Interface Text-to-Motion
            & \textbf{IN}: Word or phrase describing a human pose or action.\newline
            \textbf{OUT}: Rich-text-annotated video corresponding to the input
            word/phrase, complete with overlaid skeleton.
            & Allow users to see the output of searches on the database using
            pose and/or action keywords, such as ``run'' or ``kneeling''.\\
    Database Interface Skeleton Overlay
            & \textbf{IN}: A stream of video with humans depicted.\newline
            \textbf{OUT}: A set of human pose estimations corresponding to the
            video, in a standard data format.
            & Users should be able to use the human pose estimation solution to
            generate their own motion data set.\\
    Database Interface Text-to-Motion
            & \textbf{IN}: Word or phrase describing a human pose or action.\newline
            \textbf{OUT}: Video in common encoding (e.g. MP4), associated
            rich-text-annotations, and human pose estimations in a standardized
            format.
            & Provide users direct access to the raw motion-estimation data
            format based on action-keyword database lookup.\\
    \hline
    \end{tabular}
\end{center}
\end{table}

\section{The Scope of the Product}

\subsection{Product Boundary}

\subsection{Product Use-case List}

\subsection{Individual Product Use Cases}

\section{Functional and Data Requirements}

\subsection{Functional Requirements}

\begin{center}
    \begin{tabular}{ | p{4cm} | p{10cm} |}
    \hline
    Requirement Number & \theRequirementNumber \\ \hline
    Requirement Type & 9a. Functional Requirement \\ \hline
    Event/Use Case Numbers & \\ \hline
    Description & The text-to-motion software suite will provide an API to read
            individual frames in RGB format from a video stream. At least MP4,
            MP2 and AAC must be supported.\\ \hline
    Rationale & Researchers may wish to do their own processing on RGB frames
            before feeding those frames into the human pose estimation
            module.\\ \hline
    Originator & Brendan Duke. \\ \hline
    Fit Criterion & For a given set of test video streams, the frame-capture
            API must produce RGB frames identical to known reference frames.\\
            \hline
    Customer Satisfaction & 3 \\ \hline
    Customer Dissatisfaction & 3 \\ \hline
    Priority & Moderate priority. \\ \hline
    Conflicts & None. \\ \hline
    Supporting Materials & None. \\ \hline
    History & Created October 5th, 2016.\\
\hline
    \end{tabular}
\end{center}
\stepcounter{RequirementNumber}

\subsection{Data Requirements}

\chapter{Nonfunctional Requirements}

\section{Look and Feel Requirements}

\subsection{Appearance Requirements}

\subsection{Style Requirements}

\section{Usability and Humanity Requirements}

\subsection{Ease of Use Requirements}

\subsection{Personalization and Internationalization Requirements}

\subsection{Learning Requirements}

\subsection{Understandability and Politeness Requirements}

\subsection{Accessibility Requirements}

\section{Performance Requirements}

\subsection{Speed and Latency Requirements}

\subsection{Safety-Critical Requirements}

\subsection{Precision or Accuracy Requirements}

\subsection{Reliability and Availability Requirements}

\subsection{Robustness or Fault-Tolerance Requirements}

\subsection{Capacity Requirements}

\subsection{Scaling of Extensibility Requirements}

\subsection{Longevity Requirements}

\section{Operational and Environmental Requirements}

\subsection{Expected Physical Environment}

\subsection{Requirements for Interfacing with Adjacent Systems}

\subsection{Productization Requirements}

\subsection{Release Requirements}

\section{Maintainability and Support Requirements}

\subsection{Maintenance Requirements}

\subsection{Supportability Requirements}

\subsection{Adaptability Requirements}

\section{Security Requirements}

\subsection{Access Requirements}

\subsection{Integrity Requirements}

\subsection{Privacy Requirements}

\subsection{Audit Requirements}

\subsection{Immunity Requirements}

\section{Cultural and Political Requirements}

\subsection{Cultural Requirements}

\subsection{Political Requirements}

\section{Legal Requirements}

\subsection{Compliance Requirements}

\subsection{Standards Requirements}

\chapter{Project Issues}

\section{Open Issues}

\section{Off-the-Shelf Solutions}

\subsection{Ready-Made Products}

\subsection{Reusable Components}

\subsection{Products That Can Be Copied}

\section{New Problems}

\subsection{Effects on the Current Environment}

\subsection{Effects on the Installed Systems}

\subsection{Potential User Problems}

\subsection{Limitations in the Anticipated Implementation Environment That May
            Inhibit the New Product}

\subsection{Follow-Up Problems}

\section{Tasks}

\subsection{Project Planning}

\subsection{Planning of the Development Phases}

\section{Migration to the New Product}

\subsection{Requirements for Migration of the New Product}

\subsection{Data That Has to Be Modified or Translated for the New
            System}

\section{Risks}

\section{Costs}

\section{User Documentation and Training}

\subsection{User Documentation Requirements}

\subsection{Training Requirements}

\section{Waiting Room}

\section{Ideas for Solutions}

\bibliographystyle{IEEEtran}
\bibliography{IEEEabrv,SoftwareRequirementsSpecification} 

\end{document}
