% Original work copyright 2014 Jean-Philippe Eisenbarth
% Modified work copyright 2016 of Brendan Duke and Jordan Viveiros.

% This program is free software: you can
% redistribute it and/or modify it under the terms of the GNU General Public
% License as published by the Free Software Foundation, either version 3 of the
% License, or (at your option) any later version.
% This program is distributed in the hope that it will be useful,but WITHOUT ANY
% WARRANTY; without even the implied warranty of MERCHANTABILITY or FITNESS FOR A
% PARTICULAR PURPOSE. See the GNU General Public License for more details.
% You should have received a copy of the GNU General Public License along with
% this program.  If not, see <http://www.gnu.org/licenses/>.

% Based on the code of Yiannis Lazarides
% http://tex.stackexchange.com/questions/42602/software-requirements-specification-with-latex
% http://tex.stackexchange.com/users/963/yiannis-lazarides
% Also based on the template of Karl E. Wiegers
% http://www.se.rit.edu/~emad/teaching/slides/srs_template_sep14.pdf
% http://karlwiegers.com
\documentclass{scrreprt}
\usepackage{listings}
\usepackage{underscore}
\usepackage[bookmarks=true]{hyperref}
\usepackage[utf8]{inputenc}
\usepackage[english]{babel}
\usepackage{xcolor}
\usepackage{indentfirst}
\usepackage[section]{placeins}
\usepackage{graphicx}
\usepackage{graphics}
\usepackage{longtable}
\usepackage{booktabs}
\providecommand{\tightlist}{%
  \setlength{\itemsep}{0pt}\setlength{\parskip}{0pt}}
\hypersetup{
    bookmarks=false,    % show bookmarks bar?
    pdftitle={Software Requirement Specification},    % title
    pdfauthor={Jean-Philippe Eisenbarth},                     % author
    pdfsubject={TeX and LaTeX},                        % subject of the document
    pdfkeywords={TeX, LaTeX, graphics, images}, % list of keywords
    colorlinks=true,       % false: boxed links; true: colored links
    linkcolor=blue,       % color of internal links
    citecolor=black,       % color of links to bibliography
    filecolor=black,        % color of file links
    urlcolor=purple,        % color of external links
    linktoc=page            % only page is linked
}%
\def\myversion{0.0 }
\date{}
%\title
\usepackage{hyperref}
\begin{document}

\begin{flushright}
    \rule{16cm}{5pt}\vskip1cm
    \begin{bfseries}
        \Huge{SOFTWARE REQUIREMENTS\\ SPECIFICATION}\\
        \vspace{1.4cm}
        for\\
        \vspace{1.4cm}
        CS 4ZP6 Capstone Project\\
        \vspace{1.4cm}
        \LARGE{Version \myversion}\\
        \vspace{1.4cm}
        Prepared by Brendan Duke, Andrew Kohnen, Udip Patel, David Pitkanen, Jordan Viveiros\\
        \vspace{1.4cm}
        Using Volere Template Edition 13\\
        \vspace{1.4cm}
        McMaster Text to Motion Database\\
        \vspace{1.4cm}
        \today\\
    \end{bfseries}
\end{flushright}

\tableofcontents

\chapter*{Revision History}

\begin{center}
    \begin{tabular}{|c|c|c|c|}
        \hline
            Name & Date & Reason For Changes & Version\\
        \hline
	    Brendan Duke & Oct. 7th, 2016 & Initial Version & 0.0\\
        \hline
    \end{tabular}
\end{center}

\newcounter{ConstraintNumber}
\newcounter{RequirementNumber}

% Requirement template -------------------------------------------------------
\newcommand{\requirement}[9]{%
\fbox{\parbox{\textwidth}{%
\parbox[t]{.333\textwidth}{\raggedright%
\textbf{Req. \#}: \refstepcounter{RequirementNumber} \arabic{RequirementNumber} \label{#1}}%
\parbox[t]{.333\textwidth}{\centering%
\textbf{Req. Type}: #2}%
\parbox[t]{.333\textwidth}{\raggedleft%
\textbf{Use Case \#}: \ref{#3}}
\newline\\
\textbf{Description}: #4\\\\
\textbf{Rationale}: #5\\\\
\textbf{Originator}: #6\\\\
\textbf{Fit Criterion}: #7\\\\
\textbf{Priority}: #8 \hfill \textbf{History}: #9\\\\
}}}
% End Requirement template ---------------------------------------------------

\chapter{Project Drivers}

\section{The Purpose of the Project}

\subsection{The User Business or Background of the Project Effort}

There is an existing project at the University of Guelph that aims to create a
system for ``computational storytelling''. The goal of the computational
storytelling project is to create a system that takes as input a basic story
composed of five sentences, and outputs an animated movie based on the story,
which is produced in collaboration between an AI and human director.

As an initial step in the computational storytelling project, the University of
Guelph group requires a database of ``human motion'' that is stored with rich
text annotations. Such a database is required as a source of training data for
the computational storytelling project to use in their methods to convert text
to animated motion.

No satisfactory database of human motion data that is stored with associated
text descriptions exists currently. However, there are existing databases of
videos of people doing various actions with accompanying text describing those
actions, for example the Charades database or MSR-VTT.

Our McMaster group has been approached to assist in this initial step in the
computational storytelling project in two ways.  Firstly, we are to develop
software, based on existing research, that is able to process video and derive
human motion data (e.g. joint positions over time) from the video. Secondly, we
are to utilize data from an existing database that already has text
annotations, such as Charades, using our video-to-motion processing software to
generate a new database that contains both rich text annotations and motion data.

\subsection{Goals of the Project}

The goal of this project is to create a database, web-interface to said
database, and a deployable software bundle providing access to
already-established human pose estimation methods. Creating this database,
website and software suite will allow the larger text-to-motion project to use
the relationships between motion data and text annotations developed through
the pose estimation software in order to provide a pose and word pairing, which
can be used for animation.

\section{The Client, the Customer, and Other Stakeholders}

\subsection{The Client}

The current clients for this project are Dr.\ Taylor and his graduate student
Thor Jonsson. Dr.\ Taylor is the primary driver to develop a website and
database where annotated motion information can be generated and pulled from as
a growth point into the larger text to motion project. They will be using the
database to train Recurrent Neural Networks (RNNs) that will pair actions and
their pose found within the database to words or combinations found in the
input story.

\subsection{The Customer}

The customers are included within the clients since building this database and
website combination will be utilized by Dr.\ Taylors research team and their
external partners. In addition to Dr.\ Taylor and his research team this project
would appeal to anyone that needed a pairing of actions and pose estimations as
the website would be readily available to others.

In general, customers of the product will be researchers in the machine
learning community who are interested in multi-modal learning, and specifically
in systems that link text to human motion. Said customers will have a high
degree of knowledge related to machine learning theory. However, they cannot be
assumed to have a high degree of skill in any programming language with a steep
skill curve, such as C++ or Haskell. Also, the customer is unlikely to be
willing to invest a large amount of time in learning how to use the software
produced by the McMaster Text-to-Motion project.

\subsection{Other Stakeholders}

Other stakeholders affected by the project include Dr.\ He, our group's internal
supervisor and teacher of the CS 4ZP6 Capstone Project course, and the team
members of our group.

Dr.\ He is a professor at McMaster who may not have the same specialized
research knowledge as members of Dr.\ Taylor's group. Dr.\ He requires an
explanation of all aspects of the project, as she will be responsible for
assigning a grade to the entire group. Dr.\ He will require updates on the
progress of the group in the form of deliverables that are part of the CS 4ZP6
syllabus.

The members of our CS 4ZP6 capstone group, namely Brendan Duke, Andrew Kohnen,
Udip Patel, David Pitkanen and Jordan Viveiros, are also stakeholders affected
by the project. For the most part, our group members did not have any
specialized knowledge related to the project before commencing, although that
knowledge is being acquired as the project progresses. The group members will
require full involvement in all aspects of the project, as well as supporting
knowledge and direction from Thor Jonsson and Dr.\ Taylor.

\section{Users of the Product}

\subsection{The Hands-on Users of the Product}

\begin{center}
    \begin{tabular}{ | p{2cm} | p{4cm} | p{2cm} | p{4cm} |}
    \hline
    User Category & User Role & Subject Matter Experience &
    Technological Experience \\ \hline
    Dr.\ Taylor's Group & Using the database to train an RNN to create
            animations from text. & Master & Master \\
    Other machine learning researchers & Using the product for any multi-modal
            machine learning use-case involving text and human motion. & Master
            & Journeyman. This user category cannot be assumed to have a high
            degree of skill in complex programming languages such as C++. \\
    Amateur machine learning enthusiasts & Using the McMaster Text-to-Motion
            database and software suite to learn about multi-modal machine
            learning and human pose estimation. & Journeyman & Journeyman \\
    \hline
    \end{tabular}
\end{center}

\subsection{Priorities Assigned to Users}

Our \textbf{key users} are members of Dr.\ Taylor's research group.
\textbf{Secondary users} are other members of the machine learning community.
Amateur machine learning enthusiasts are \textbf{unimportant users}.

\subsection{User Participation}

Thor Jonsson and Dr.\ Taylor will be expected to assist in supporting our group
with their domain knowledge of deep learning methods. They will also be
expected to participate in shaping the interfaces to the product (both the web
interface and the programming interface to the database) by using the
prototypes of those interfaces and providing feedback.

The minimum amount of participation from Dr.\ Taylor and Thor would be
participation in a meeting with our group members on a bi-weekly to monthly
basis, as well as participating in weekly correspondence electronically (e.g.
by e-mail).

\subsection{Maintenance Users and Service Technicians}

Maintenance users would certainly be members of Dr.\ Taylor's research group,
as they will be using the software produced by the project after its completion
and may need to add changes to the product.

Once the product is open-sourced into the community, maintenance users could
range from machine learning researchers to amateur machine learning
enthusiasts. These users could be expected to fix bugs or add new features that
were not in the initial scope of the project.

\chapter{Project Constraints}

Note that constraints also use the requirement shell, and therefore also use
the requirement numbering.

Customer satisfaction and dissatisfaction have been evaluated in the "Priority" entry.

Also there are no conflicts between requirements, or supporting materials, so
these entries have been dropped.

\section{Mandated Constraints}

\subsection{Solution Constraints}
\label{req-solution-constraint}

\requirement
{Constraint_DeepLearningMethods}
{\ref{req-solution-constraint} Solution Constraint}
{software-skeleton-event}
{The human pose estimation component should use deep learning methods.}
{This constraint is to allow Dr.\ Taylor's group to integrate the software into
 their existing text-to-motion pipeline.}
{Dr.\ Graham Taylor}
{Dr.\ Taylor should confirm that the deep learning methods used in the human
 pose estimator are satisfactory.}
{High}
{Created September 26th, 2016.}

\requirement
{Constraint_DataStorageFormat}
{\ref{req-solution-constraint} Solution Constraint}
{database-text-to-motion-event}
{Use a standard format such as LMDB or HDF5 for storing text-motion data.}
{Having the data in a standard format will enable users to re-use existing code
 to manipulate that data.}
{Thor Jonsson }
{Run a set of existing tests to manipulate the standard data format (e.g. LMDB)
 and assert that those tests must pass.}
{High}
{Created October 3rd, 2016.}

\subsection{Implementation Environment of the Current System}
\label{req-implementation-environment}

\requirement
{Constraint_Linux}
{\ref{req-implementation-environment} Implementation Environment}
{software-skeleton-event}
{The Text-to-Motion Software Suite must run under Linux.}
{Linux is the operating system used by the Guelph Machine Learning research
 lab, and also the most commonly used operating system in the research
 community.}
{Dr.\ Graham Taylor}
{Automated builds and testing should pass on popular Linux distributions:
 Ubuntu, Fedora and RHEL.}
{High}
{Created September 26th, 2016.}

\requirement
{Constraint_PythonAPI}
{\ref{req-implementation-environment} Implementation Environment}
{software-skeleton-event}
{Major APIs to the Text-to-Motion database must be accessible from the Python
 programming language.}
{Python is the language used by the rest of Dr.\ Taylor's text-to-motion
 pipeline. Python is a popular, easy-to-use, and quick-to-prototype language,
 and is therefore one of the most favoured programming languages among the
 Machine Learning research community.}
{Dr.\ Graham Taylor}
{There must be hooks to all major interfaces written in Python, and there must
 be tests that are directly testing the Python interfaces.}
{High}
{Created September 26th, 2016.}

\subsection{Partner or Collaborative Applications}

The McMaster Text-to-Motion database software will collaborate with the rest of
the University of Guelph group's text-to-motion pipeline. While their
application is still in a research stage, we can expect that their application
will be written in Python.

Furthermore, the animation component of Dr.\ Taylor's group's project will be
using the proprietary software application Muvizu.

\subsection{Off-the-Shelf Software}

There are a number of off-the-shelf libraries that can and should be used by
this project in order to implement the requirements. In particular, deep
learning libraries should be used for the training of neural networks and for
doing numerical computations.

Two notable deep learning frameworks are listed below.

\textbf{Caffe}, originally developed by the Berkeley Vision and Learning Center
(BVLC),  is a deep learning framework written in C++ with Python and Matlab
wrappers.

Caffe has a strong ConvNet implementation and is popular amongst the
computer vision community. However, Caffe's RNN implementation and support for
language models is lacking compared with other libraries.

Caffe is released under the BSD 2-Clause license.

\textbf{TensorFlow}, originally developed by researchers at Google, is a deep
learning and numerical computation framework written in C++ and Python.

TensorFlow has a clean, modular architecture and since it provides both Python
and C++ interfaces, code can be prototyped in a rich high-level interpreted
language before deployment in a high-performance, scalable environment.

TensorFlow is open sourced under the Apache 2.0 open source license.

It will also be necessary to use a library for accessing video codecs. This is
in order to convert video streams to and from sets of images, which will be
input to the human pose estimation software.

\textbf{FFmpeg} is a leading multimedia framework that is able to decode and
encode video, in addition to having many other features. FFmpeg can be utilized
for any video encoding and decoding purposes in the project.

To implement search-by-text functionality into the McMaster Text-to-Motion
Database, it will be helpful to leverage an existing software package that
implements full-text search.

\textbf{Sphinx} is such a software package, as it implements full-text search.
Sphinx also integrates well with SQL databases.

Sphinx is licensed under GPL version 2.

\subsection{Anticipated Workplace Environment}

As the McMaster Text-to-Motion Database is a software product, there are no
requirements considerations specifically pertaining to the anticipated
workplace environment of users of the product. Any considerations of
environment, such as operating system environment, have already been covered
under the ``Implementation Environment'' section.

\subsection{Schedule Constraints}
\label{req-schedule-constraint}

\requirement
{Constraint_DemoSchedule}
{\ref{req-schedule-constraint} Schedule Constraint }
{software-skeleton-event}
{A demonstration of the product must be completed by February 13th, 2017.}
{This is a deadline that is part of the CS 4ZP6 course.}
{Dr.\ Wen Bo He}
{By February 13th, 2017, fully functional versions of the web interface,
 database interface and human pose estimation software must all be working as
 verified by Dr.\ He.}
{High}
{Created September 26th, 2016.}

\requirement
{Constraint_ProjectDeadline}
{\ref{req-schedule-constraint} Schedule Constraint}
{software-skeleton-event}
{The project must be completed by April 5th, 2017.}
{The project is part of the CS 4ZP6 Capstone Project course.}
{Dr.\ Wen Bo He}
{All documentation, testing and implementation must be completed and checked in
 to GitHub by April 5th, 2017.}
{High}
{Created September 21th, 2016.}

\subsection{Budget Constraints}

There is no budget allocated for this project. As such, hardware requirements
such as GPUs to carry out the computations required for the project must be
provided by the client, Dr.\ Taylor, or borrowed through other means such as
seeking help from McMaster professors.

\section{Naming Conventions and Definitions}

\subsection{Definitions of All Terms, Including Acronyms, Used in the Project}

\textbf{The Project} when used, is referring to the McMaster Text to Motion
Database project. The project aims to generate a database of human pose
estimation model information that is linked to videos of human motion
containing rich text annotations.

\textbf{Human Pose Estimation} is the process of estimating the configuration,
or pose, of the body based on a single still image or a sequence of images that
comprise a video. Human pose estimation may find the chin, radius, humerus, and
other bone and joint positions.

\textbf{Charades} is a dataset composed of approximately 10K videos of daily
indoor activities, complete with associated action-describing sentences,
collected through Amazon Mechanical Turk\cite{charades}.

\textbf{MSR-VTT}, standing for ``Microsoft Research Video to Text'', is a
large-scale video benchmark for the task of translating video to text. MSR-VTT
provides 10K video clips spanning 41.2 hours and containing 200K clip-sentence
pairs in total\cite{msr-vtt}.

\textbf{Feedforward Neural Networks} are artifical neural networks where
connections between the units do \textit{not} form a cycle). They are the
simplest type of neural network, because information moves in only one
direction.

\textbf{ConvNets} or \textbf{Convolutional Neural Networks} are a type of
feed-forward artificial neural network. ConvNets are inspired by the visual
cortex and are commonly used in visual recognition applications.

\textbf{RNNs} or \textbf{Recurrent Neural Networks} are a class of artificial
neural networks where units form a directed cycle, in contrast with
feed-forward neural networks.

\textbf{Deep Belief Networks} are a type of deep neural network composed of
multiple layers of ``hidden units'' (variables that are not observable), with
connections between layers but not between units of a given layer.

\textbf{Multi-modal neural language models} are models of natural language that
can be conditioned on other modalities, e.g. high-level image
features\cite{DBLP:journals/corr/KirosSZ14}.

\textbf{Muvizu} is an interactive 3D animation package designed for quick storytelling.

An \textbf{autoencoder} is an artificial neural network used to learn a
representation of a set of data. The simplest form of an autoencoder is a
feed-forward neural net with an input layer, an output layer, and one or more
hidden layers. Instead of being trained to predict an output $Y$ from input
$X$, an autoencoder is trained to reconstruct its own input $X'$ from $X$.

\textbf{Ground truth} refers to the expected classification, i.e. the expected
output in a machine learning classification problem, based on a given input.

\textbf{CMU Graphics Lab Motion Capture Database} is a dataset of human motions
that were collected at the Motion Capture Lab of Carnegie Mellon University.
The motion capture data was collected by recording the movements of humans
wearing a body suit with 41 markers taped on. The 3D motion data in this
dataset is available in two formats (.vsk/.v or .asf/.amc), which describe a
\textit{skeleton and its joints} and \textit{movement data}, respectively.

\subsection{Data Dictionary for any Included Models}

Here we define the different information flows displayed in Figure
\ref{work-context-diagram}, the work context diagram.

A \textbf{skeleton overlay} refers to a set of human joint and body-part
estimation data that has been associated with a given video stream. A skeleton
overlay may also encompass an actual graphical stick-figure, displayed on top
of the video stream, with circles indicating the position of each joint.

\textbf{Text annotated video} refers to a video that has a time-stamped
natural-language description associated with it. Similarly, \textbf{text
annotated motion data} refers to human pose data that is time-stamped and
coupled with a natural-language description of the actions comprised by those
human pose data.

\section{Relevant Facts and Assumptions}

\subsection{Facts}

\begin{itemize}
        \item The training of deep neural networks can take on the order of
                days to weeks. For example, training two of the different
                neural network architectures used in
                \cite{DBLP:journals/corr/PfisterCZ15} on four NVIDIA GTX Titan
                GPUs took 3 and 7 days, respectively.
        \item ConvNets take a fixed size input and generate a fixed size
                output, whereas RNNs can handle arbitrary input/output lengths.
                This is one reason why ConvNets are more popular in vision
                applications, while RNNs are used for language applications.
                However, neural network architectures can be used that combine
                both ConvNets and RNNs.
        \item The uncompressed size of the Charades dataset is 55GB at full
                resolution, and 16GB at 480p.
        \item The goals of this project, specifically generating a database of
                text-annotated human motion data, constitute new research and
                there are no existing solutions available.
\end{itemize}

\subsection{Assumptions}

\begin{itemize}
        \item It has been assumed that it is possible to re-use existing
                software components in order to achieve the human pose
                estimation use-case, as it would take a full eight months just
                to implement an algorithm such as that of
                \cite{DBLP:journals/corr/PfisterCZ15} from scratch.
        \item It has been assumed that this project will not encompass
                generating a dataset of videos with text annotations, and that
                we will be able to rely on an existing dataset such as the
                Charades dataset.
        \item It is assumed that a storage format for human pose estimation
                data already exists, and will be available for this project to
                leverage.
\end{itemize}

\chapter{Functional Requirements}

\section{The Scope of the Work}

\subsection{The Current Situation}

There is a large amount of existing research into human pose estimation, which
this project will leverage. Based on Constraint
\ref{Constraint_DeepLearningMethods}, we focus on existing solutions that use
deep learning methods.

\cite{DBLP:journals/corr/PfisterCZ15} present a ConvNet architecture for human
pose estimation from videos, which is able to benefit from temporal context
across multiple frames using optical flow. This work is focused on upper-body
human pose estimation only.

\cite{DBLP:journals/corr/BelagiannisZ16} propose a ConvNet model for predicting
2D human body poses in an image. This model is able to achieve state-of-the-art
results using a simple architecture, and draws on the work done in
\cite{DBLP:journals/corr/PfisterCZ15}.

\cite{DBLP:journals/corr/WeiRKS16} introduces \textit{Convolutional Pose
Machines (CPMs)} for pose estimation in images. CPMs consist of a sequence of
ConvNets that iteratively produce 2D belief maps. Source code from this paper
is available, and makes use of the Caffe deep learning library.

In \cite{zhou2016sparseness}, the authors estimate 3D full-body human poses
from a monocular image sequence. They train a ConvNet to predict the locations
of joints in 2D, then estimate human poses in 3D via an
Expectation-Maximization algorithm over the entire video sequence.

On the animation side of the larger computational storytelling problem our
project is assisting in solving, \cite{Holden2016} use
deep learning to synthesize character movements based on high-level parameters
such as a curve that the character should follow. The learned motion is
represented by the hidden units of a convolutional autoencoder.

\subsection{The Context of the Work}

The context diagram, displaying adjacent systems to this project, is shown in
Figure \ref{work-context-diagram}.

\begin{figure}[!ht]
        \caption{McMaster Text-to-Motion Database Work Context Diagram}
        \label{work-context-diagram}
        \centering
        \includegraphics[width=0.8\textwidth]{../data/mcmaster-text-to-motion-work-context-diagram.png}
\end{figure}

\subsection{Work Partitioning}
\label{work-partitioning}

The events/use-cases for the project are laid out in Table \ref{business-event-list-table}.

\begin{table}
\begin{enumerate}
\caption{Business Event List}
\label{business-event-list-table}
    \begin{tabular}{  p{1cm} | p{3cm} | p{5cm} | p{5cm} }
    \hline
    Event \# & Event Name & Input and Output & Summary \\
    \hline
    \item \label{web-skeleton-event}
            & Web Interface Skeleton Overlay
            & \textbf{IN}: An image or video with humans in it.\newline
            \textbf{OUT}: The same image or video, with a skeleton overlaid on
            top of all humans indicating their bone and joint positions.
            & Allow users to observe the human pose estimation component in
            real time through a web interface.\\
    \item \label{web-text-to-motion-event}
            & Web Interface Text-to-Motion
            & \textbf{IN}: Word or phrase describing a human pose or action.\newline
            \textbf{OUT}: Rich-text-annotated video corresponding to the input
            word/phrase, complete with overlaid skeleton.
            & Allow users to see the output of searches on the database using
            pose and/or action keywords, such as ``run'' or ``kneeling''.\\
    \item \label{software-skeleton-event}
            & Software Interface Skeleton Overlay
            & \textbf{IN}: A stream of video with humans depicted.\newline
            \textbf{OUT}: A set of human pose estimations corresponding to the
            video, in a standard data format.
            & Users should be able to use the human pose estimation solution to
            generate their own motion data set.\\
    \item \label{database-text-to-motion-event}
            & Database Interface Text-to-Motion
            & \textbf{IN}: Word or phrase describing a human pose or action.\newline
            \textbf{OUT}: Video in common encoding (e.g. MP4), associated
            rich-text-annotations, and human pose estimations in a standardized
            format.
            & Provide users direct access to the raw motion-estimation data
            format based on action-keyword database lookup.\\
    \hline
    \end{tabular}
\end{enumerate}
\end{table}

\section{The Scope of the Product}

\subsection{Product Boundary}

We do not include a product boundary diagram, and instead refer to Figure
\ref{work-context-diagram}.

All interfaces to the McMaster Text-to-Motion Database will be software
interfaces except the web interfaces, which will have a human component. There
will be a search box requiring user input to do text-to-motion lookups, and
there will be a graphical web interface for users to upload videos to use the
skeleton-overlay functionality.

\subsection{Product Use-case List}

The product use cases correspond exactly to the business events list, and
therefore to see product use-cases one should refer to Table
\ref{business-event-list-table}.

\subsection{Individual Product Use Cases}

Use cases \ref{web-skeleton-event} and \ref{web-text-to-motion-event} would be
used in a scenario where the user is looking to do experiments with immediate
feedback, in order to gain intuition about the product's functionality. E.g. a
researcher may want to use the Web Interface Skeleton Overlay feature (use case
\ref{web-skeleton-event}) to visually gauge the human pose estimation
software's accuracy before spending the time to process a large amount of video
data.

Use cases \ref{software-skeleton-event} and \ref{database-text-to-motion-event}
could be used in a more large scale experiment. E.g. human motion data could be
attached to a large amount of video data using use case
\ref{software-skeleton-event}, or a large amount of text-annotated motion data
could be retrieved from the database programmatically using use case
\ref{database-text-to-motion-event}.

\section{Functional and Data Requirements}

\subsection{Functional Requirements}
\label{req-functional-requirement}

\requirement
{Req_FrameEncodeDecode}
{\ref{req-functional-requirement} Functional Requirement}
{software-skeleton-event}
{The text-to-motion software suite will provide an API to read individual
 frames in RGB format from a video stream. At least MP4, MP2 and AAC must be
 supported.}
{Researchers may wish to do their own processing on RGB frames before feeding
 those frames into the human pose estimation module.}
{Brendan Duke}
{For a given set of test video streams, the frame-capture API must produce RGB
 frames identical to known reference frames.}
{Moderate}
{Created October 5th, 2016.}

\requirement
{Req_VideoProcessingCorrectness}
{\ref{req-functional-requirement} Functional Requirement}
{software-skeleton-event}
{The skeleton overlay software must output a set of human pose data that is
 time-stamped and accurate to the ground truth joint positions in the frames
 corresponding to those time stamps.}
{If the software does not have accurate output, then the output cannot be used
 to generate human motion.}
{Brendan Duke}
{A set of double-blind tests should be designed that allows test subjects to
 compare the human pose data output from the McMaster Text-to-Motion software
 suite again human motion skeletons rendered from the CMU Graphics Lab Motion
 Capture Database. The fit criterion would be that the human motion data
 produced by the McMaster software is statistically indistinguishable from
 CMU's MoCap data from sensors on actual humans.}
{High}
{Created September 26th, 2016.}

\requirement
{Req_EveryVideoRetrievable}
{\ref{req-functional-requirement} Functional Requirement}
{database-text-to-motion-event}
{The domain of inputs to the text-to-motion database search should output a
 range that covers the entire set of videos in the database.}
{There is text associated with each video, therefore for every video there
 should be some set of matching text that can retrieve it.}
{Brendan Duke}
{A test application could be designed to intentionally pull text data from the
 description of each video and assert that the video from which that text data
 was obtained appears in the output of the search.}
{High}
{Created October 10th, 2016.}

\requirement
{Req_RetrievedVideoNoFalsePositives}
{\ref{req-functional-requirement} Functional Requirement}
{database-text-to-motion-event}
{The videos retrieved from a text-to-motion search should not contain any
 results with text descriptions not containing the search terms.}
{False positives would make the text-to-motion retrieval difficult to use,
 since returned videos would not reliably match the search terms.}
{Brendan Duke}
{A test application could be designed to, for each video, search a random set
 of terms not in that video's text description and assert that the given video
 is not returned in the search results.}
{High}
{Created October 10th, 2016.}

\requirement
{Req_RetrievedVideoRelevance}
{\ref{req-functional-requirement} Functional Requirement}
{database-text-to-motion-event}
{The videos retrieved from a text-to-motion search should be ordered by relevance.}
{Users will want to use the motion data that most closely fits the text they
 have searched for.}
{Brendan Duke}
{A smaller set of test data can be created by hand and sorted by a human, and
 the results can be compared against an expected output, also generated by hand.
 The output from a larger set of test data could be compared against the output
 from a known full-text search engine, such as Sphinx, or if we are using
 Sphinx, then Lucene.}
{High}
{Created October 10th, 2016.}

\subsection{Data Requirements}

The McMaster Text-to-Motion project is not strictly adhering to an
object-oriented design, and therefore there is no class model that must be
followed as part of the requirements.

Interactions between classes, or modules, will be diagrammed as part of the
module hierarchy during the design iterations of the project development.

\chapter{Nonfunctional Requirements}

\section{Look and Feel Requirements}

\subsection{Appearance Requirements}
\label{req-appearance}

\requirement
{Req_RetrievedVideoRelevance}
{\ref{req-appearance} Appearance Requirement}
{web-skeleton-event} % add <web-text-to-motion-event>
{The web interface shall have a color-scheme that is focused on contrast.}
{A color shceme with bright and dark colors clashing is bound to captivate users of the website.}
{Udip Patel}
{At least 6 out of every 10 users should be left with a positive impression of the color scheme.}
{Low}
{Created October 8th, 2016.}

\subsection{Style Requirements}
\label{req-style}

\requirement
{Req_RetrievedVideoRelevance}
{\ref{req-style} Style Requirement}
{web-skeleton-event} % add <web-text-to-motion-event>
{The web interface shall be designed to look minimalistic (in terms of text) and visually informative through graphs and other visual aides.}
{To be able to convey the complex processes of a deep learning model in a simple way that can be understood by any end user, regardless of their level of knowledge. The amount of text displayed should also be minimized to avoid overwhelming a first-time user with too much information.}
{Udip Patel}
{The web interface shall be able to display a graph that tracks how the deep learning model analyzes an input image. This is similar to an open-source website called \href{http://playground.tensorflow.org/}{TensorFlow Playground}.}
{Low}
{Created October 8th, 2016.}

\section{Usability and Humanity Requirements}

\subsection{Ease of Use Requirements}
\label{req-ease-of-use}

\requirement
{Req_RetrievedVideoRelevance}
{\ref{req-ease-of-use} Ease of Use Requirement}
{web-skeleton-event} % add <web-text-to-motion-event>
{The functionalites of the web interface shall be easy to use for any casual internet user above the age of 12 who understands how to upload and download files to a website.}
{Having a simpler interface can open up this web app to a lot of potential users.}
{Udip Patel}
{Any end user at least 12 years of age shall be able know how to upload an image to the web interface within 15 seconds of visiting the website.}
{Medium}
{Created October 10th, 2016.}

\requirement
{Req_RetrievedVideoRelevance}
{\ref{req-ease-of-use} Ease of Use Requirement}
{web-skeleton-event}
{The user shall not be expected to remember anything to be able to interact with the Web Interface.}
{This is just to reduce the amount of work that the user has to do in order to be able to get the desired result.}
{Udip Patel}
{The Web Interface shall only ask for images, videos or text from the user, and without prompting the user again, it will return the given image or video (with or without text annotations).}
{Low}
{Created October 10th, 2016.}

\requirement
{Req_RetrievedVideoRelevance}
{\ref{req-ease-of-use} Ease of Use Requirement}
{web-text-to-motion-event}
{The user shall be able to type in a descriptive word or phrase into a text box from the web interface.}
{This way, the functionality provided by the web application will be made obvious and accessible to the user.}
{Udip Patel}
{The end user shall be able to input one word or phrase into the text box (written in HTML/CSS) using their keyboard, and be able to click on a 'Submit' button to view the functionality of thew web app.}
{Medium}
{Created October 10th, 2016.}

\subsection{Personalization and Internationalization Requirements}

As of Revision 0, there are no requirements for Language support. Since the web Interface and the Software Interface would rely on Natural Language Processing for English, adding language support for multiple languages would be too large of a feature to implement for now.

\subsection{Learning Requirements}
\label{req-learning}

\requirement
{Req_RetrievedVideoRelevance}
{\ref{req-learning} Learning Requirement}
{web-skeleton-event}
{Any end-user shall be able to use the web application and generate a skeleton-overlay without having any prior training.}
{This is because all the user needs to be able to do is upload and download a file to/from the website.}
{Udip Patel}
{Any user that visits the website shall be able to understand how to use the skeleton-overlay functionality within 1 minute of visiting the website.}
{High}
{Created October 10th, 2016.}

\requirement
{Req_RetrievedVideoRelevance}
{\ref{req-learning} Learning Requirement}
{web-text-to-motion-event}
{Any end-user shall be able to use the web application to instantiate a 'text-to-motion' event without having any prior training.}
{This is because all the user needs to do is type in a descriptive word or phrase into a text box.}
{Udip Patel}
{Any user that visits the website shall be able to understand how to use the 'text-to-motion' functionality within 30 seconds of visiting the website.}
{High}
{Created October 10th, 2016.}

\requirement
{Req_RetrievedVideoRelevance}
{\ref{req-learning} Learning Requirement}
{software-skeleton-event} % ADD <database-text-to-motion-event>
{The functionalities of the software interface shall be easily understood by programmers who have experience with APIs.}
{This is because using a command line API still takes some base knowledge of software to set up, and we cannot assume that every user will be able to set up our API without any problems even if they follow all of the detailed steps.}
{Udip Patel}
{Any user should be able to set up the Software Interface and be able to use its functionalities (described in events 3 and 4) through the command line interface within 15 minutes.}
{Medium}
{Created October 10th, 2016.}

\subsection{Understandability and Politeness Requirements}
\label{req-understandability-politeness}

\requirement
{Req_RetrievedVideoRelevance}
{\ref{req-understandability-politeness} Understandability and Politeness Requirement}
{web-skeleton-event}
{The end user of the web interface shall not be able to see the inner workings
of the deep learning model that is used to estimate the human pose or action
from an image/video}
{This is because the user should be obscured from having to deal with the
complexities of deep learning. Furthermore, it would be computationally
expensive to show the users such low-level detail in a way that they could
understand it.}
{Udip Patel}
{The end user of the web interface shall be able to get a valid output from the
skeleton-overlay function without seeing any of the deep learning code or
processes.}
{High}
{Created October 10th, 2016.}

\requirement
{Req_RetrievedVideoRelevance}
{\ref{req-understandability-politeness} Understandability and Politeness Requirement}
{database-text-to-motion-event}
{The end user of the software interface shall not know how the program is able
to map a word or phrase to a human pose or action}
{Using the software interface should be simple, and if we show these highly
complex data mappings to the user, we might risk overwhelming the user by
giving them a large amount of information that is of very little use.}
{Udip Patel}
{The end user of the web interface shall be able to get a valid output from the
software interface without being exposed to the database mappings that are used
to determine the valid output.}
{High}
{Created October 11th, 2016.}

\subsection{Accessibility Requirements}
\label{req-accessibility}
As of Revision 9, there are no accessbility requirements for this application.
Since both interfaces only contain a virtual/visual component and no physical
mechanisms, there is no need to provide any accessibility requirements for now

\section{Performance Requirements}

\subsection{Speed and Latency Requirements}
\label{req-speed-latency}

\requirement
{Req_RetrievedVideoRelevance}
{\ref{req-speed-latency} Speed and Latency Requirement}
{web-skeleton-event}
{The web interface shall be able to connect to an external database with no problems and be able to store items and query for them quickly }
{The speed is necessary here because generating the actual output itself is going to be computationally intensive so the program should not be spend a lot of time waiting on responses from the database.}
{Udip Patel}
{The speed can be tested through a test that measures throughput of database updates and queries.}
{Medium}
{Created October 11th, 2016.}

\requirement
{Req_RetrievedVideoRelevance}
{\ref{req-speed-latency} Speed and Latency Requirement}
{software-skeleton-event}
{The software interface's deep learning model shall be able to process an image (RGB format) in a span of one minute or a set of frames in relative time (ex. 5 frames = 5 minutes)}
{The deep learning model is the most computationally expensive aspect of the software interface, and thus, should be able to process images/videos quickly enough so that the user does not find the process too time consuming.}
{Udip Patel}
{The Quantative goal has been mentioned in the Description. The deep learning model shall be able to process an image in a span of 1 minute}
{High}
{Created October 11th, 2016.}

\requirement
{Req_RetrievedVideoRelevance}
{\ref{req-speed-latency} Speed and Latency Requirement}
{web-text-to-motion-event}
{The web interface, given a word or phrase as input, shall be able to respond with an output of an image/video detailing a pose or action within 2 minutes}
{The web interface will need a considerable amount of time to process the request since it has to be able to parse the text input, be able to interpret it as a description of a pose or action using a set of defined database mappings. The web interface is is then responsible for constructing a visual object that encapsulates all of the bone and joint positions and exporting the object in a standardized data format. This is a very time consuming set of processes to compute, so it should be expected to take at least 2 minutes.}
{Udip Patel}
{The Quantative goal has been mentioned in the Description. The entire workset shall be completed in a span of 2 minutes, the workset itself: 1) parsing input, 2) mapping the text input to a description, 3) generating a visual object based on the description, 4)exporting the object in a data format that is accessible to the user }
{High}
{Created October 11th, 2016.}

\subsection{Safety-Critical Requirements}
\label{req-safety-critical}
As of Revision 0, there are no safety critical requirements. Both of the interfaces are very simple in terms of the output they produce, and even if the program is fed with bad input, there will be no actual physical or virtual damage to the system.

\subsection{Precision or Accuracy Requirements}
\label{req-precision-accuracy}
\requirement
{Req_RetrievedVideoRelevance}
{\ref{req-precision-accuracy} Precision and Accuracy Requirement}
{database-text-to-motion-event}
{The software interface shall be able to deliver a data object that can accurately encapsulate the bone positions and joint angles of a human being's skeleton}
{The software interface would not be very useful to the end user if it is not able to accurately portray a human bodym, so this is a very high priority requirement.}
{Udip Patel}
{The software interface shall be able to create a data object that can use vectors to provide the positions of well-defined bones relative to each other in a 3D grid. The vectors which describe the bones should all be able to correspond to the anatomical proportions of an average human, with an uncertainty of 5 percent.}
{High}
{Created October 11th, 2016.}

\requirement
{Req_RetrievedVideoRelevance}
{\ref{req-precision-accuracy} Precision and Accuracy Requirement}
{web-skeleton-event}
{Given an image or a set of frames as its input, the web interface shall be able to accurately draw a skeleton overlay around a human being, if one is found in the given image or set of frames}
{This is one of the main use cases of the web interface, so it stands to reason that the web interface should be able to provide an accurate skeleton overlay. This is crucial to the satisfaction of the customer or end user }
{Udip Patel}
{The web interface shall be able to draw an outline around a human being in a picture or set of frames with an uncertainty of 20 pixels (+-).}
{High}
{Created October 11th, 2016.}

\subsection{Reliability and Availability Requirements}
\label{req-reliability-availability}
\requirement
{Req_RetrievedVideoRelevance}
{\ref{req-reliability-availability} Reliability and Availablilty Requirement}
{software-skeleton-event}
{The software interface shall be available for use 24 hours a day, 365 days a year.}
{This will make it so that the user is able to decide when they want to use the functionality, and will not be constrained by any scheduling issues.}
{Udip Patel}
{The quantative goal of this requirement has been stated in the description}
{Medium}
{Created October 11th, 2016.}

\requirement
{Req_RetrievedVideoRelevance}
{\ref{req-reliability-availability} Reliability and Availablilty Requirement}
{web-skeleton-event}
{The web interface shall be available for use 24 hours a day, 365 days a year, except for when the system has to be audited, tested or migrated to a different environment}
{Any standard web app should be available for use at any time, this is a major standard in web development, so this project aims to abide by it.}
{Udip Patel}
{The quantative goal of this requirement has been stated in the description}
{High}
{Created October 11th, 2016.}

\subsection{Robustness or Fault-Tolerance Requirements}

\label{req-robustness}
\requirement
{Req_RetrievedVideoRelevance}
{\ref{req-robustness} Robustness or Fault-Tolerance Requirement}
{web-skeleton-event}
{The web interface shall throw error messages for every possible exception or error caught}
{To let the users know when they have done something wrong or are not using the web interface as intended. This will also help the users learn how to interact correctly with the web interface}
{Udip Patel}
{The web interface shall be able to recognize when the set of inputs is invalid (ex. a completely monotone image), and shall alert the user with a error message right away}
{High}
{Created October 11th, 2016.}

\requirement
{Req_RetrievedVideoRelevance}
{\ref{req-robustness} Robustness or Fault-Tolerance Requirement}
{web-text-to-motion-event}
{The web interface shall filter the input word or phrase before passing the text to the database, where it can be processed}
{This is to avoid making confusing requests to the database. If an unintelligible word or phrase is entered, there is no need to apply the database mapping functionality because it will not be able to return a significant response.}
{Udip Patel}
{The web interface shall be able to parse the text input to weed out unintelligible words or curse words and be able to respond to the user with an error message right away }
{High}
{Created October 11th, 2016.}

\subsection{Capacity Requirements}
\label{req-capacity}

\requirement
{Req_RetrievedVideoRelevance}
{\ref{req-capacity} Capacity Requirement}
{web-skeleton-event}
{The web interface shall be able to serve multiple connections}
{This is a standard requirement for a simple web application.}
{Udip Patel}
{The web interface shall be able to serve at least 5 users at the same time}
{Low}
{Created October 11th, 2016.}

\requirement
{Req_RetrievedVideoRelevance}
{\ref{req-capacity} Capacity Requirement}
{web-text-to-motion-event}
{The web interface shall use a database that stores a lot of information about the relation of words and phrases to human poses and actions}
{This is so that the deep learning model can accurately map a word to an action or pose correctly, this is a major part of the requirements.}
{Udip Patel}
{The web interface shall reference a database of mappings between words and poses/actions that is at least 5 GB large, to begin with }
{Medium}
{Created October 11th, 2016.}


\subsection{Scaling of Extensibility Requirements}
\label{req-scaling}

\requirement
{Req_RetrievedVideoRelevance}
{\ref{req-capacity} Scaling of Extensibility Requirement}
{software-skeleton-event}
{The deep learning model that is used by the software interface shall be put through rigorous training so that it can predict the correct skeleton-overlay position more accurately}
{The deep learning model can make better predictions if it is given more data to train with. End users will be satisified with the product if the skeleton overlay functionality is extremely precise}
{Udip Patel}
{The deep learning model shall be trained with thousands, if not tens of thousands of images that depict a myriad of human poses and actions.}
{Medium}
{Created October 11th, 2016.}



\subsection{Longevity Requirements}
\label{req-longevity}
\requirement
{Req_RetrievedVideoRelevance}
{\ref{req-longevity} Longevity Requirement}
{web-skeleton-event}
{The web interface shall be able to last for years to come}
{Since this is a web application, the software can continue to run for a long time, until the server stops working. Even then, the application can be migrated to a different server. Either way, this web interface will last a long time}
{Udip Patel}
{The web interface shall be up and running at least until September 2017}
{Low}
{Created October 11th, 2016.}


\section{Operational and Environmental Requirements}

\subsection{Expected Physical Environment}
\label{req-op-env}

\requirement
{Req_RetrievedVideoRelevance}
{\ref{req-op-env} Expected Physical Requirement}
{web-skeleton-event}
{The web interface shall operate on a Linux-friendly server that can access a tensorflow model, either directly or indirectly}
{The web application will need a runtime environment and will use up resources, so a server is the tool to use }
{Udip Patel}
{The web interface shall be able to operate on an ngnix or Apache server with no problems}
{Medium}
{Created October 11th, 2016.}


\subsection{Requirements for Interfacing with Adjacent Systems}
\label{req-interfacing}

\requirement
{Req_RetrievedVideoRelevance}
{\ref{req-interfacing} Requirement for Interfacing with Adjacent Systems}
{web-skeleton-event}
{The web interface shall interface with the Tensorflow library and a pre-trained Tensorflow model specialized in image processing}
{The web interface itself cannot be expected to implement the deep learning. The Tensorflow library is a configurable and accessible deep learning network, and should be used to do the actual 'predicting' when it comes to drawing a skeleton outline over an image or a set of frames.}
{Udip Patel}
{The web interface shall be able to use a simple Tensorflow model to get one simple yes or no answer. (ex. is there a person in a given picture [data content]) The web interface shall be able to contact the Tensorflow model through Tensorflow's Python API [medium]}
{High}
{Created October 11th, 2016.}



\subsection{Productization Requirements}
\label{req-product}

\requirement
{Req_RetrievedVideoRelevance}
{\ref{req-product} Productization Requirement}
{software-skeleton-event}
{The software interface shall export the resulting output (an image or set of images with a skeleton outline over a human being's silhouette, if one exists) to all major operating systems }
{This helps us cover all of our possible users and ensure that they are able to use the functionality without running into versioning or environment-based problems}
{Udip Patel}
{The software interface shall allow the user to choose the format of the exported image that is best suited to their environment (JPEG, PNG, DDS, etc) }
{Medium}
{Created October 11th, 2016.}




\subsection{Release Requirements}
\label{req-release}
As of Revision 0, there are no release requirements. This is because the application has yet to be fully documented, and we are not under contractual obligations to release future versions within a given timeframe


\section{Maintainability and Support Requirements}

\subsection{Maintenance Requirements}
\label{req-maintenance}

As of revision 0, there are no maintenance requirements. This is because a simple protoype must be built first before we can identify the aspects of the system that need a lot of maintenance

\subsection{Supportability Requirements}
\label{req-supportability}

\requirement
{Req_RetrievedVideoRelevance}
{\ref{req-supportablility} Supportablility Requirement}
{web-skeleton-event}
{The web interface shall provide a small 'tour' of the functionalities with simple examples}
{This is to ensure that first-time users can gain a full understanding of what the web interface is supposed to do and the features it provides}
{Udip Patel}
{The web interface shall have a small 'help' button that will trigger the showing of several modals (web development term) that are meant to guide the user through every available feature}
{Medium}
{Created October 11th, 2016.}


\subsection{Adaptability Requirements}
\label{req-adaptablility}

\requirement
{Req_RetrievedVideoRelevance}
{\ref{req-adaptability} Adaptability Requirement}
{web-skeleton-event}
{The web interface shall be available for use from all the major browsers}
{This is to ensure that all of the possible users of the systems can have access to its features}
{Udip Patel}
{The web interface shall be able to operate on Chrome, Safari, Firefox, IE 7+, Edge, and Opera}
{High}
{Created October 11th, 2016.}


\section{Security Requirements}

\subsection{Access Requirements}

There is no data considered sensitive in our product. After deployment the
entire product will be openly available. Therefore we do not consider any
special access-privilege requirements.

\subsection{Integrity Requirements}
\label{req-integrity}

\requirement
{Req_DataIntegrity}
{\ref{req-integrity} Integrity Requirement}
{database-text-to-motion-event}
{The motion data stored on our server should maintain its integrity despite any
attempts at attack.}
{Researchers will rely on the integrity of our text and motion data.}
{Brendan Duke}
{Since the data is static, it would be possible to keep a hash of the data and
check periodically that it is the same. The hash of the data should be stored
somewhere secure where the hash itself cannot be updated.}
{High}
{Created October 10th, 2016.}

\subsection{Privacy Requirements}

Users should not be submitting any private or sensitive information to the
McMaster Text-to-Motion software, and therefore we do not consider any privacy
requirements for this project.

\subsection{Audit Requirements}

There are no required audit checks that must be complied with by the software.

\subsection{Immunity Requirements}

As the software includes a user-mode program and a web application running in a
browser, the software cannot introduce security vulnerabilities in users'
systems that do not already exist in their operating system or browser.
Therefore we do not consider immunity requirements for the software.

\section{Cultural and Political Requirements}

\subsection{Cultural Requirements}

There are no cultural factors affecting the acceptability of the McMaster
Text-to-Motion project.

\subsection{Political Requirements}

There are no political factors affecting the acceptability of the McMaster
Text-to-Motion project.

\section{Legal Requirements}

\subsection{Compliance Requirements}
\label{req-compliance}

\requirement
{Req_Licensing}
{\ref{req-compliance} Compliance Requirement}
{software-skeleton-event}
{The licensing conditions of any open-source software included as part of this
product must be adhered to. E.g., if a GPL 2 component is used, then the source
code of our product will have to be distributed with the product.}
{Legal action against us could be pursued by the authors of the open-source
software if we do not adhere to the conditions of their license.}
{Brendan Duke}
{We would require a lawyer to sign off that our product meets the terms and
conditions of all of the open-source software we are using.}
{High}
{Created October 10th, 2016.}

\subsection{Standards Requirements}

There are no standards set by industry bodies, or internal to our group, that
pertain to this project.

\chapter{Project Issues}

\section{Open Issues}

{Here we discuss issues not addressed in the functional or
non-functional requirements.}

One open issue common to most machine learning problems is the quality of the
fit, as this can always be improved upon.

We are using ready-made algorithms, and there are many such options. So
another open problem is to choose amongst these pose estimation algorithms. A
possible solution is, since there are multiple options that there could be an
option where the user can select their own algorithm to analyze the data.
However due to time constraints we will need to choose one without first
testing. So one problem is simply choosing an algorithm that will work without
having tested it first.

A second problem is the hardware that we intend to use for the problem.
We are analyzing and performing pose estimations on an enormous
database of video. This is a computationally intensive task so we need
a fast computer hardware system to perform our computations. One
problem is choosing the hardware solution to this problem. We will also
be creating a web application so it may be in our best interest to host
the application using a server of our creation. In this case we would
also have to create and maintain this system.

\section{Off-the-Shelf Solutions}
\label{off-the-shelf-solutions}

\subsection{Ready-Made Products}

{In choosing an algorithm for the pose estimation problem there is
already software that is available online that implements these
algorithms. We intend to use these resources in our project.}

\subsection{Reusable Components}
\label{reusable-components}

{We are creating an intermediate software for pre-processing data and
placing this data in a database that can be easily queried through a
user interface. The pre-processing and sorting/searching of data is a
common task in machine learning development. For this reason it is
hopeful that the product we develop is not specialized to the
pre-processing algorithms we use or the database that contains the
database we analyze.}

{Ideally we will create a multi-tiered system that will have many
components that could be re-used for future data analysis and viewing
tasks.}

\subsection{Products That Can Be Copied}

{The product that we aim to create will implement machine learning
algorithms on a database of media in order to produce a new database of
processed media.}

{Some popular machine learning algorithms are the Short Long Term Memory
(LSTM) neural networks. Due to their popularity many web applications
have been created which allow users to visualize the way that these
networks learn and they allow them to see the output of the networks for
given input. Two such web applications are the TensorFlow playground
produced by Google and Visual Analysis of Recurrent Neural Networks
offered by Harvard. We will aim to reproduce many of the features that
are present in these products.}

{The machine learning algorithms typically have many parameters in them
which can be adjusted to optimize their performance. The previously
mentioned products let the user choose these parameters through the user
interface. We will aim to implement these features as well.}

\section{New Problems}

{There are several problems which we wish to avoid when creating our
system. Our project is a small part of a larger project. Each of the
sub-projects have been broken down as a modular hierarchy. We hope that
our system will be able to work with the other modules without
corrupting their performance and correctness.}

{Our project will house a large database of media data. The database
will need to be queried and the data manipulated. However through our
manipulation we will not want the data stored in the database to be
corrupted so that the performance of our system degrades the more it is
used. We will also aim to make it so that our program does not cause
side-effects on the larger projects that make use of it.}

\subsection{Effects on the Current Environment}

{Currently if someone wishes to analyze the video data offered online
they will have to analyze raw video data. Our application will offer a
new interface in which this data has been processed in a manner that is
useful if the user should wish to analyze the human motion in the
video.}

{The alternative web interface we offer will be different from the previously
made available systems. However the format of the video will be preserved in
our transformations.}

{The finished product will exist as a standalone web-application hosted on a
server. The server will allow queries to be made. We hope that the system
will thereby be able to exist independently of the larger project of which it
is a part of, so that our product will not have to be installed on every user's
computer.}

\subsection{Effects on the Installed Systems}

{The transformations that we will apply to the video data should be very
useful for machine learning researchers. Having transformed data can
speed up the learning rate in many machine learning algorithms and we
expect the product will benefit users.}

\subsection{Potential User Problems}

Users of our software should not suffer any adverse or ill effects due to said
usage, so this section is not applicable.

\subsection{Limitations in the Anticipated Implementation Environment That May
            Inhibit the New Product}

We refer here to constraints \ref{Constraint_Linux} and \ref{Constraint_PythonAPI}.
While the Linux constraint should not inhibit the new product, the limitation
of using a Python API may present issues with performance, scalability, and
maintainability in the long term.

\subsection{Follow-Up Problems}

{As we have mentioned this project is a sub-component of a larger project that
is meant to facilitate the change in representations from text to motion. In
this current project we are only analyzing the problem of pre-processing the
data. So, another problem that will need to be considered in the future is
how the data we generate can be used to train a deep neural network to make the
previously mentioned transformation from text to motion.}

\section{Tasks}

\subsection{Project Planning}

{To develop our software product we will make use of the iterative and
incremental software development process shown in Figure~\ref{fig-project-planning}.}

{\begin{figure}[!ht]
        \label{fig-project-planning}
        \caption{Iterative and Incremental Software Development Process}
        \includegraphics[width=0.8\textwidth]{../data/project-planning.png}
\end{figure}}

\subsection{Planning of the Development Phases}

{We have passed through the initial planning step. In this step we decided on
a project to pursue, and explored whether or not the idea was achievable by
talking to researchers and exploring published papers in this area.}

{The next step of the project is the requirements stage where we formally state
the objectives of our project.}

{An important feature of this stage is that the process: Requirements, Design,
Implementations, Verification and Evaluation can be run in parallel for each
module of the product, followed by systems level testing at the end. This
method of unit testing in parallel with development ensures that each module in
the system will run without producing errors with the other parts.}

{Our system involves several separate components. It is a web
application and needs to store a lot of data in a database. So the
program will consist of a data layer, a layer that performs the machine
learning algorithm on the data and a layer for the user interface. Each
of these separate layers can be developed in parallel in our system.}

\section{Migration to the New Product}

\subsection{Requirements for Migration of the New Product}

{As we have mentioned in Reusable Components
(Section~\ref{reusable-components}) that we plan to use the implemented
versions of the machine learning algorithms that are already available online
by the authors of the papers. However much of this code is written in Matlab
and since the program is to be Python based some of the code will need to be
rewritten in this language. However since we are re-writing the code we will be
able to test our output by comparing it to the output of the older code we are
rewriting.}

\subsection{Data That Has to Be Modified or Translated for the New
            System}

If we choose to migrate existing software from one deep learning framework to
another, e.g. from Caffe to Tensorflow, it may be necessary to regenerate the
model corresponding to the neural network architecture on one framework into a
model that is compatible with the new framework.

\section{Risks}

{We are taking algorithms that are meant to analyze images and applying
them to video. The transition from image to video is a step up in
computational tractability. There is a risk that the increase in volume
of data will cause performance issues for our program.}

{Another worry is the possibly that algorithms will not generalize well
from images to video. They have not been tested on these tasks so it is
possible that they will create very noisy data in their output and will
not generalise well.}

\section{Costs}

{For our program we will be using freely available and open source programs.
The list of programs we will make use of was mentioned in Off-the-Shelf
Solutions (Section~\ref{off-the-shelf-solutions}).}

{However, for our hardware we may wish to run our own server to host the
application. In this case there will be maintenance and hardware costs.  The
maintenance cost will be in terms of depreciation of the equipment, the cost of
connecting to a network and streaming data over the internet and power
consumption. The power consumption, however, should be low.}

The hardware we would need to purchase would include an NVIDIA GTX 1070 and an
Intel i5 (\$1600).

The reason for the NVIDIA GPU is to maintain compatibility with deep learning
libraries, which are relying on NVIDIA's proprietary CUDA hardware abstraction
layer at the lowest level.

The reason for an Intel i5 is so that the CPU will not bottleneck the GPU's
performance.

\begin{longtable}[c]{@{}ll@{}}
\toprule
\begin{minipage}[t]{0.47\columnwidth}\raggedright\strut
{Description}
\strut\end{minipage} &
\begin{minipage}[t]{0.47\columnwidth}\raggedright\strut
{Cost}
\strut\end{minipage}\tabularnewline\midrule
\begin{minipage}[t]{0.47\columnwidth}\raggedright\strut
{GTX 1070 Intel i5}
\strut\end{minipage} &
\begin{minipage}[t]{0.47\columnwidth}\raggedright\strut
{\$1600}
\strut\end{minipage}\tabularnewline
\begin{minipage}[t]{0.47\columnwidth}\raggedright\strut
{Data Plan (approximate use)}
\strut\end{minipage} &
\begin{minipage}[t]{0.47\columnwidth}\raggedright\strut
{\$10/month}
\strut\end{minipage}\tabularnewline
\begin{minipage}[t]{0.47\columnwidth}\raggedright\strut
{Power}
\strut\end{minipage} &
\begin{minipage}[t]{0.47\columnwidth}\raggedright\strut
{\textless{}\$5}
\strut\end{minipage}\tabularnewline
\begin{minipage}[t]{0.47\columnwidth}\raggedright\strut
{Maintenance of equipment}
\strut\end{minipage} &
\begin{minipage}[t]{0.47\columnwidth}\raggedright\strut
{Unknown at this point}
\strut\end{minipage}\tabularnewline
\bottomrule
\end{longtable}

{The second cost that we have to consider is the amount of time that the
project will require in order to implement. }

{Our project has four use-cases discussed in Section~\ref{work-partitioning} of
our report. We estimate the amount of time required for one person to complete
each use-case, assuming they spend 20 hours per week.}

Due to the research nature of the project, these estimates are subject to
change based on the success or failure of our prototyping efforts.

\begin{longtable}[c]{@{}ll@{}}
\toprule
\begin{minipage}[t]{0.47\columnwidth}\raggedright\strut
{Functional Requirement}
\strut\end{minipage} &
\begin{minipage}[t]{0.47\columnwidth}\raggedright\strut
{Time Necessary}
\strut\end{minipage}\tabularnewline\midrule
\begin{minipage}[t]{0.47\columnwidth}\raggedright\strut
{Web Interface Skeleton Overlay}
\strut\end{minipage} &
\begin{minipage}[t]{0.47\columnwidth}\raggedright\strut
{Two weeks implementation, one month testing and debugging.}
\strut\end{minipage}\tabularnewline
\begin{minipage}[t]{0.47\columnwidth}\raggedright\strut
{Web Interface Text-to-Motion}
\strut\end{minipage} &
\begin{minipage}[t]{0.47\columnwidth}\raggedright\strut
{One month implementation, two months of testing and debugging.}
\strut\end{minipage}\tabularnewline
\begin{minipage}[t]{0.47\columnwidth}\raggedright\strut
{Software Interface Skeleton Overlay}
\strut\end{minipage} &
\begin{minipage}[t]{0.47\columnwidth}\raggedright\strut
{One month research, one month implementation, two months of testing and
        debugging.}
\strut\end{minipage}\tabularnewline
\begin{minipage}[t]{0.47\columnwidth}\raggedright\strut
{Database Interface text to motion}
\strut\end{minipage} &
\begin{minipage}[t]{0.47\columnwidth}\raggedright\strut
{One month implementation, two months of testing and debugging.}
\strut\end{minipage}\tabularnewline
\bottomrule
\end{longtable}

\section{User Documentation and Training}

\subsection{User Documentation Requirements}

{As one component of our application will be a standalone web application, we
will have a help feature.}

\begin{enumerate}
\tightlist
\item
  {Technical Specifications document - specifies which algorithms our
  program uses and their expected accuracy}
\item
  {User Manual or help feature - instructs the user on navigating the
  user interface, provides information on the contents of the database as well
  as a list of contacts for additional help.}
\end{enumerate}

\subsection{Training Requirements}

{The intended users of our program will be machine learning experts who
have plenty of experience working with software interfaces. So the
training these users will need to use our program will be minimal. The
software will only have the four use-cases we have mentioned in our
functional requirements.}

\section{Waiting Room}

{As we have mentioned we are using off the shelf version for the machine
learning algorithms used in this project. A next step in our project would be
to improve on these algorithms. The procedures are designed for working on
images, but it would be interesting to optimize or change the algorithms so they
take advantage of the fact that the data we are using is video and not strictly
images.}

{We are at the moment focusing on human motion and labelling body
parts. The motion of other objects such as cars and animals is also
interesting. A next interesting step in this project would be to
consider the motion of a more diverse range of objects.}

\section{Ideas for Solutions}

{The ultimate solution of the product is the generation of animated motion from
text. This is the step that is to be taken after our project is complete.}

\bibliographystyle{IEEEtran}
\bibliography{IEEEabrv,SoftwareRequirementsSpecification}

\end{document}
