\documentclass[a4paper, 12pt]{article}

\usepackage{cite}
\usepackage{amsmath}
\usepackage{fullpage}
\usepackage{url}
\usepackage{graphicx}
\usepackage{inputenc}
\usepackage{titling}

\date{\today}
\title{Proof of Concept Plan for McMaster Text to Motion Database \\CS 4ZP6} 

\author{Brendan Duke\\
        Andrew Kohnen\\
        Udip Patel\\
        Dave Pitkanen\\
        Jordan Viveiros}

\begin{document}

\maketitle

\section{Risks}
The significant risks of this project are split into three major sections
represented by the website, database, and deep learning network. The largest
risk to the project is linking the three sections together so that the website
can pull information from the database, and the deep learning network can use
this information to run pose estimation.

Some significant risks are involved within each of these sections and will be elaborated on below:

\begin{itemize}
    \item The website must use some form of database query in order to
            correctly return information that the user searched for.
    \item The database must contain all the required videos and text pairings
            from larger libraries like Charades.
    \item The deep learning network is going to use Caffe and requires the
            steep learning curve that is associated with deep learning.
    \item Generating the proper test cases to provide proof of correctness is
            difficult on large databases.
\end{itemize}

\section{Datasets}
{Our primary mode of input data for the program will be uploading an image to
the website from the user's computer. We will be largely using the "Charades"
Database in order to act as input for our program.}

\section{Deliverables}
{As a proof of concept demo we will show a working website that will function
as a base for running pose estimation. This is to show that the associated
risks mentioned above can be overcome and fully worked out as the project
matures.

To utilize the pose estimation the user will either have to upload an
image, or search the database from a subset of the options. The prototype will
have two aspects to the demonstration with the addition of a third if full text
search if it is implemented on a subset of the given data. The first aspect
will be a website. Once on the website running the pose estimation will be
shown by either searching for an action or uploading a picture and displaying
the position of the chin, left and right humerus, left and right radius/ulna,
left and right femur, left and right tibia/fibula and the spine of the person
found within the image. 

\begin{itemize}
        \item A functional website, as an interface for running pose esimtation.
        \item The ability to upload an image and update the database with that
                image.
\end{itemize}
}

\section{Performance Metrics}
There are a few key aspects to which we can measure the effectiveness of our
proof of concept demonstration:

\begin{itemize}
        \item Page loads should happen in real-time.
        \item Image query should take less than ten seconds.
\end{itemize}

\section{Resources}
\begin{itemize}
    \item Caffe
    \item TensorFlow
    \item FFmpeg
    \item Sphinx 
\end{itemize}

\end{document}
