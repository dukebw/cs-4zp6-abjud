\documentclass[a4paper, 12pt]{article}

\usepackage{cite}
\usepackage{amsmath}
\usepackage{fullpage}
\usepackage{url}
\usepackage{graphicx}
\usepackage{inputenc}
\usepackage{titling}

\date{\today}
\title{Proof of Concept Plan for McMaster Text to Motion Database \\CS 4ZP6} 

\author{Brendan Duke\\
        Andrew Kohnen\\
        Udip Patel\\
        Dave Pitkanen\\
        Jordan Viveiros}

\begin{document}

\maketitle

\section{Risks}

The significant risks of this project are split into three major sections
represented by the website, database, and deep learning network. The largest
risk to the project is linking the three sections together so that the website
can pull information from the database, and the deep learning network can use
this information to run pose estimation.

Some significant risks are involved within each of these sections and are
elaborated on below:

\begin{itemize}
    \item The website must use some form of database query in order to
            correctly return information that the user searched for.
    \item The database must contain all the required videos and text pairings
            from larger libraries like Charades.
    \item The deep learning network is going to use Caffe and requires the
            steep learning curve that is associated with deep learning.
\end{itemize}

All of the above are risks in the sense that they pose a challenge to complete
the items in time to be demonstrated.

\section{Datasets}
{Our primary mode of input data for the program will be uploading an image to
the website from the user's computer. We will be largely using the ``Charades''
Database in order to act as input for our program.}

\section{Deliverables}
{
The following set of deliverables will be completed for the proof of concept
demonstration.

\begin{itemize}
        \item A functional website, as an interface for running pose estimation.
        \item Said website should contain a database.
        \item The ability to upload images and videos, and to update the
                database with those uploaded data.
        \item The ability to run human pose estimation on any uploaded image
                and video. By human pose estimation we mean that for the
                uploaded media, the skeletons and joints of any humans in those
                media will be indicated visually.
        \item The ability to search for uploaded images and videos through some
                means, e.g. by tag or name.
\end{itemize}
}

\section{Performance Metrics}
There are a few key aspects to which we can measure the effectiveness of our
proof of concept demonstration:

\begin{itemize}
        \item Page loads should happen in real-time.
        \item Image query should take less than ten seconds.
\end{itemize}

\section{Resources}
\begin{itemize}
    \item Caffe
    \item TensorFlow
    \item FFmpeg
    \item Sphinx 
\end{itemize}

\end{document}
\grid
