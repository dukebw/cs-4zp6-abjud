\documentclass{scrreprt}
\usepackage{listings}
\usepackage{underscore}
\usepackage{ragged2e}
\usepackage[bookmarks=true]{hyperref}
\usepackage[utf8]{inputenc}
\usepackage[english]{babel}
\usepackage{xcolor}
\usepackage{amsmath,amssymb}
\usepackage{indentfirst}
\usepackage[section]{placeins}
\usepackage{graphicx}
\usepackage{graphics}
\usepackage{longtable}
\usepackage{booktabs}
\usepackage{xcolor} % for different colour comments
\usepackage{tabto}
\usepackage{array}
\usepackage{mdframed}
\mdfsetup{nobreak=true}
\usepackage{xkeyval}
\usepackage{tabularx}
\hypersetup{
    colorlinks,
    citecolor=black,
    filecolor=black,
    linkcolor=red,
    urlcolor=blue
}
\usepackage[skip=2pt, labelfont=bf]{caption}
\usepackage{titlesec}
\usepackage[section]{placeins}
\graphicspath{ {image/} }

\titleformat{\paragraph}
{\normalfont\normalsize\textbfseries}{\theparagraph}{1em}{}
\titlespacing*{\paragraph}
{0pt}{3.25ex plus 1ex minus .2ex}{1.5ex plus .2ex}


%% Comments
\newif\ifcomments\commentstrue

\ifcomments
\newcommand{\authornote}[3]{\textcolor{#1}{[#3 ---#2]}}
\newcommand{\todo}[1]{\textcolor{red}{[TODO: #1]}}
\else
\newcommand{\authornote}[3]{}
\newcommand{\todo}[1]{}
\fi

\newcommand{\wss}[1]{\authornote{magenta}{SS}{#1}}
\newcommand{\ds}[1]{\authornote{blue}{DS}{#1}}

\begin{document}
\title{\textbf Text to Motion Database\\[\baselineskip]\Large Test Report}
\author{Brendan Duke\\Andrew Kohnen\\Udip Patel\\David Pitkanen\\Jordan Viveiros}
\date{\today}

\maketitle
\tableofcontents
% \listoftables
% \listoffigures


\begin{table}[bp]
\caption*{\textbf Revision History}
\begin{tabularx}{\textwidth}{p{3.5cm}p{2cm}X}
\toprule {\textbf Date} & {\textbf Version} & {\textbf Notes}\\
\midrule
March 14, 2017 & 0.0 & File created\\
March 23, 2017 & 0.1 & Initial Template Completed \\
\bottomrule
\end{tabularx}
\end{table}

\newpage

\pagenumbering{arabic}


\chapter{List of Tables}
    tables for a specific tests have been removed to improve readability

\chapter{Introduction}

\section{Purpose of Document}
This document summarizes the testing and the findings of the tests for the Text to Motion project. This documentuses the implementation outlined in the test plan.

\section{Scope of Testing}

\section{Organization}
    Section 1 is our introduction and introduces this report. Section 2 describes the test cases as outlined in our test plan and their results. Section 4 describes traceability to the requirements. Section 4 describes what changes have been made in response to our testing.

    \textbf{Testing structure}

    Automated Testing

    \quad generating inputs

    Manual Testing

%I have added a few examples of what the tables should be.
\chapter{Preliminary Testing}

\section{Proof of Concept}
%An example of non functional requirements. In test cases where we we want an average we add an additional cell in bold
\subsection{Functional website for pose estimation}

\subsubsection{Description}
\begin{flushleft}
This is a baseline test to make sure the website is running , appears in a web browser when its url is entered and that pictures can be uploaded and viewed on the site.

Put the proper url into the browser and click on each of the links to make sure that no error occurs.  Meaning that there are no excessive wait times and that the broswer does not crash.  Make sure there are poses shown on the site.
\end{flushleft}

\subsubsection{Results}
%Any comments we wish to make or add can be added here should we wish to expand on the results. An overall summary can also be here
 \centering
 \begin{tabular}{||p{5cm}|p{2.5cm}||}
 \hline
 \textbf Test & \textbf Result\\
 \hline\hline
  TextToMotion link & Pass  \\
   \hline
  Register Link & Pass  \\
   \hline
  Sign up Link & Pass  \\
     \hline
 Poses shown when TextToMotion is clicked & Pass  \\
      \hline
 Search Link clicked & Pass  \\
      \hline
 About Link clicked & Pass  \\
      \hline
 ImagePoseDraw link clicked & Pass  \\
 \hline
\end{tabular}

\subsection{Updating the database}
\subsubsection{Description}
\begin{flushleft}
The database allows users to upload images to the database through the website
interface. A test is considered a pass if an image or video can be uploaded,
then accessed by user through the web interface.  Again we test this on the
same files we have used in the previous tests: TomBrady.jpg, AverageGirl.jpg
and AverageGuy.jpg.
\end{flushleft}
\subsubsection{Results}
 \centering
 \begin{tabular}{||p{2.5cm}|p{2.5cm}||}
 \hline
 \textbf Test & \textbf Result\\
 \hline\hline
  Tom Brady & Pass  \\
   \hline\hline
  Average Girl & Pass  \\
   \hline\hline
  Average Guy & Pass  \\
\hline

   %fill in these entries once we have them
 \hline
 \end{tabular}

\subsection{Running Pose Estimation}
\subsubsection{Description}
\begin{flushleft}
Once a user uploads an image they should be able to run pose estimation on an
uploaded image. A test is considered a pass if an image or video can be
uploaded, have pose estimation run on it, and have the result be visible to the
user.
\end{flushleft}
\subsubsection{Results}
 \centering
 \begin{tabular}{||p{2.5cm}|p{2.5cm}||}
 \hline
 \textbf Test & \textbf Result\\
 \hline\hline
  Tom Brady & Pass  \\
   \hline\hline
  Average Girl & Pass  \\
   \hline\hline
  Average Guy & Pass  \\
\hline

   %fill in these entries once we have them
 \hline
 \end{tabular}

\subsection{Search by Tag or Name}
\subsubsection{Description}
\begin{flushleft}
The database has the ability to search through the database using a video tag
or the video name. A test is considered a pass if we can search the dataase
using a tag or name and the correct video appears.
\end{flushleft}
\subsubsection{Results}
 \centering
 \begin{tabular}{||p{2.5cm}|p{2.5cm}||}
 \hline
 \textbf Test & \textbf Result\\
 \hline\hline
  Tom Brady & Pass  \\
   \hline\hline
  Average Girl & Pass  \\
   \hline\hline
  Average Guy & Pass  \\
\hline

   %fill in these entries once we have them
 \hline
 \end{tabular}
\begin{flushleft}
All of these tests passed. It was possible to search through the database and
retrieve entries by tag and video name.
\end{flushleft}
%test for functional requirements.
\section{Solution Constraints Testing}

\subsection{Deep Learning Methods Test}
\subsubsection{Description}
\begin{flushleft}
Our Supervisor Dr. Taylor will check if the Deep Learning Methods used are modern and up-to-date.
\end{flushleft}
\subsubsection{Results}
 \centering
 \begin{tabular}{||p{2.5cm}|p{2.5cm}||}
 \hline
 \textbf Test & \textbf Result\\
 \hline\hline
   &  \\ %fill in these entries once we have them
 \hline
 \end{tabular}

\subsection{Standard Data Format Test}
\subsubsection{Description}
\begin{flushleft}
An automated test that checks if the human pose data used for the project is standard and compatible with existing software libraries.
\end{flushleft}
\subsubsection{Input Data}
 \centering
 \begin{tabular}{p{3cm}p{6cm}}
 \hline\hline
 Input & Description\\
 \hline\hline
   &  \\ %fill in these entries once we have them
 \hline
 \end{tabular}
\subsubsection{Results}
%Results are largely unchanged from nonfunctional requirements, though somtimes we might choose to simply describe the result without a table.

\subsection{Linux Platform Build and Run Test}
\subsubsection{Description}
\begin{flushleft}
An automated test that confirms that the pose estimation software can be built correctly on a Linux platform.
\end{flushleft}
\subsubsection{Input Data}
 \centering
 \begin{tabular}{p{3cm}p{6cm}}
 \hline\hline
 Input & Description\\
 % NEED THE BUILD COMMANDS WITH DESCRIPTIONS
 \hline\hline
   &  \\ %fill in these entries once we have them
 \hline
 \end{tabular}
\subsubsection{Results}

\subsection{Python API Hook Testing}
\subsubsection{Description}
\begin{flushleft}
An automated test that confirms the major interfaces for the project have working python hooks.
\end{flushleft}
\subsubsection{Input Data}
 \centering
 \begin{tabular}{p{3cm}p{6cm}}
 \hline\hline
 Input & Description\\
 % NEED THE BUILD COMMANDS WITH DESCRIPTIONS
 \hline\hline
   &  \\ %fill in these entries once we have them
 \hline
 \end{tabular}
\subsubsection{Results}

\chapter{Functional Requirements}
\section{Supported Video Encodings Test}
\subsection{Description}
\begin{flushleft}
Tests whether the ReadFrames API is able to decode MP4, MP2 and AAC video files.
\end{flushleft}
\subsection{Input Data}
 \centering
 \begin{tabular}{p{3cm}p{6cm}}
 \hline\hline
 Input & Description\\
 % NEED THE BUILD COMMANDS WITH DESCRIPTIONS
 \hline\hline
   &  \\ %fill in these entries once we have them
 \hline
 \end{tabular}
\subsection{Results}

\section{Frame Reading Timestamp Accuracy Test}
\subsection{Description}
\begin{flushleft}
Tests whether the timestamps on the frames returned by the ReadFrames API match their temporal position in the original video stream.
\end{flushleft}
\subsection{Input Data}
 \centering
 \begin{tabular}{p{3cm}p{6cm}}
 \hline\hline
 Input & Description\\
 % NEED THE BUILD COMMANDS WITH DESCRIPTIONS
 \hline\hline
   &  \\ %fill in these entries once we have them
 \hline
 \end{tabular}
\subsection{Results}

\section{Human Pose Estimation Data Quality Test}
\subsection{Description}
\begin{flushleft}
Test to ensure the data quality produced by the human pose estimator component. A set of Charades videos will be processed by the human pose estimator, and skeleton animations corresponding to the generated human pose data will be created (this is a scoped part of the software pipeline). A double-blind test will be ran, where testers will be shown random mixed sets of the skeleton animations produced by McMaster Text to Motion, together with skeletons from actual motion capture data coming from CMU’s motion capture lab. Testers will indicate whether they think the motion capture data came from actual motion capture, or from the pose estimation software.
\end{flushleft}
\subsection{Results}
 \centering
 \begin{tabular}{||p{2.5cm}|p{2.5cm}||}
 \hline
 \textbf Test & \textbf Result\\
 \hline\hline
   &  \\ %fill in these entries once we have them
 \hline
 \end{tabular}

\section{Database Output Full Range Coverage Test}
\subsection{Description}
\begin{flushleft}
Tests to be sure all entries in the database can be successfully searched for.
\end{flushleft}
\subsection{Input Data}
 \centering
 \begin{tabular}{p{3cm}p{6cm}}
 \hline\hline
 Input & Description\\
 % NEED THE BUILD COMMANDS WITH DESCRIPTIONS
 \hline\hline
   &  \\ %fill in these entries once we have them
 \hline
 \end{tabular}
\subsection{Results}

\section{Databse No False Positives}
\subsection{Description}
\begin{flushleft}
Tests that the database search does not return any false positives, such as videos or images that do not contains searched words.
\end{flushleft}
\subsection{Input Data}
 \centering
 \begin{tabular}{p{3cm}p{6cm}}
 \hline\hline
 Input & Description\\
 % NEED THE BUILD COMMANDS WITH DESCRIPTIONS
 \hline\hline
   &  \\ %fill in these entries once we have them
 \hline
 \end{tabular}
\subsection{Results}

\section{Full Text Search Order by Relevance Test}
\subsection{Description}
\begin{flushleft}
%TO FILL
\end{flushleft}
\subsection{Input Data}
 \centering
 \begin{tabular}{p{3cm}p{6cm}}
 \hline\hline
 Input & Description\\
 % NEED THE BUILD COMMANDS WITH DESCRIPTIONS
 \hline\hline
   &  \\ %fill in these entries once we have them
 \hline
 \end{tabular}
\subsection{Results}

%For tests that require a time use this
\chapter{Non-Functional Requirements}
\section{Usability}
\begin{flushleft}
In order to determine the usability of the Text to Motion Database a small sample of users were asked to use the website to preform some predetermined actions and answer questions afterwards. Before the participants were asked to preform any actions they were given a minute to familiarize themselves with the interface but were not given any guidance or tips from the development team. Once the time was up they were asked to upload an image on mobile, or desktop through the camera, URL or file saved within the computer. While preforming the required action the participants time was recorded and used to determine if a requirement had passed or failed. Upon completion of the task the users were asked to rate the style and design of the website on a scale of 1 - 10.
\end{flushleft}

\section{Results}
\begin{flushleft}
The results from the participants can be seen throughout the Non-Functional Requirements along with a pass or fail based on the requirements description.
\end{flushleft}

\section{Look and Feel Requirements}
\subsection{Description}
\begin{flushleft}
A test to see if the color scheme of the website is visually appealing. The participants were asked to rate the websites colour scheme on a scale from 1-10, and any result above a 6 will be considered a pass.
\end{flushleft}
 
 \centering
 \begin{tabular}{||p{2.5cm}|p{2.5cm}|p{2.5cm}||}
 \hline
 \bf User & \bf Rating & \bf Result\\
 \hline\hline
 Nick & 7 & Pass \\
 \hline
 Maddy & 8 & Pass\\ %fill in these entries once we have them
 \hline
 Sarah & 7 & Pass \\
 \hline
 \end{tabular}

\section{Style Requirements}
\section{Description}
\begin{flushleft}
The website interface should be minimal and should inform the user of valid actions through visual means. Participants were asked to rate the design from 1-10, and any result above a 5 will be considered a pass.
\subsubsection{Results}
\end{flushleft}
 
 \centering
 \begin{tabular}{||p{2.5cm}|p{2.5cm}|p{2.5cm}||}
 \hline
 \bf User & \bf Rating & \bf Result\\
 \hline\hline
 Nick & 9 & Pass \\
 \hline
 Maddy & 8 & Pass\\ %fill in these entries once we have them
 \hline
 Sarah & 8 & Pass \\
 \hline
 \end{tabular}

\section{Ease of Use Requirements}

\subsection{Upload/Download}
\subsubsection{Description}
\begin{flushleft}
Through the web interface a user should be able to upload a picture using either a mobile phone camera, URL, or saved file. The participant will be on the Home page and asked to upload an image through one of the methods listed before. In order for this to be considered a pass it should take the users 30 seconds or less to complete the upload process and click the button. 
\subsubsection{Results}
\end{flushleft}

 \centering
 \begin{tabular}{||p{4.5cm}|p{2.5cm}|p{2.5cm}|p{2.5cm}||}
 \hline
 \bf Test & \bf User & \bf Time & \bf Result \\
 \hline\hline
    Uploading based on URL & Nick & 28 seconds & Pass\\
 \hline\hline
    Uploading image from a mobile device & Maddy & 22 seconds  & Pass\\
 \hline\hline
    Uploading file from Desktop &  Sarah & 26 seconds & Pass\\
 \hline
 \end{tabular}
 
\subsection{Text Box Functionality}
\subsubsection{Description}
\begin{flushleft}
The user should be able to input a descriptive word or phrase into a text-box from within the web interface in order to search for a video. In order to complete this task the users were asked to search for a specific word and display the results. Any time below 10 seconds will be considered a pass.
\subsubsection{Results}
\end{flushleft}

 \centering
 \begin{tabular}{||p{4.5cm}|p{2.5cm}|p{2.5cm}|p{2.5cm}||}
 \hline
 \bf Test & \bf User & \bf Time & \bf Result \\
 \hline\hline
   Search for "Woman" & Nick & 4 seconds  & Pass\\ %fill in these entries once we have them
 \hline\hline
   Search for "f" & Maddy & 3 seconds  & Pass\\
 \hline\hline
   Search for "the" & Sarah & 4 seconds  & Pass\\
 \hline
 \end{tabular}

\section{Learning Requirements}

\subsection{Usability Tests}
\subsubsection{Description}
\begin{flushleft}
The user should be able to interact with the website without prior knowledge. They will given a minute to explore the website. After that time the participants were asked to rate the usability on a scale of 1-10. An average of 6 is required for a pass.
\subsubsection{Results}
\end{flushleft}

 \centering
 \begin{tabular}{||p{2.5cm}|p{2.5cm}|p{2.5cm}||}
 \hline
 \bf User & \bf Rank & \bf Result\\
 \hline\hline
 Nick & 6 & Pass \\
 \hline
 Maddy & 8 & Pass \\ 
 \hline
 Sarah & 8 & Pass\\
 \hline
 \end{tabular}

\subsection{Text to Motion Training }
\subsubsection{Description}
\begin{flushleft}
The user should be able to interact with the website without prior knowledge. They will given a minute to explore the website. After that time they are asked to rate the usability of the website on a scale of 1-10. An average of 6 is required for a pass.
\subsubsection{Results}
\end{flushleft}

 \centering
 \begin{tabular}{||p{2.5cm}|p{2.5cm}|p{2.5cm}||}
 \hline
 \bf User & \bf Rank & \bf Result\\
 \hline\hline
 Nick & 7 & Pass \\
 \hline
 Maddy & 8 & Pass \\ 
 \hline
 Sarah & 6 & Pass\\
 \hline
 \end{tabular}

\section{Politeness and Understandability Requirements}
\subsection{Hiding the Inner Workings}
\subsubsection{Description}
\begin{flushleft}
Users should not be able to see the deep learning model and its training when using the pose estimation. When prompted the website should display the correct skeletons without any low-level detail. Once uploaded the participants were asked if they saw anything that seemed out of place or any information on the deep learning process, if they did not it will be considered a pass.
\end{flushleft}

\subsubsection{Results}
 \centering
 \begin{tabular}{||p{2.5cm}|p{2.5cm}|p{2.5cm}||}
 \hline
 \textbf Test & \textbf User & \textbf Result\\
 \hline\hline
  Uploading an image from URL & Nick & Pass\\
 \hline\hline
  Uploading an image from mobile & Maddy & Pass\\
 \hline\hline
  Uploading an image from desktop & Sarah & Pass\\
 \hline
 \end{tabular}

\section{Speed and Latency Testing}

\subsection{External Database Connection Response Time}
\subsubsection{Description}
\begin{flushleft}
The web interface should be able to connect to an external database and store or query items. In order for this test to be considered a pass the confirmation of the image being uploaded would have to occur within 30 seconds so that additional resources are not wasted by the database. Testing this will occur by uploading an image and testing the total time taken.
\subsubsection{Input Data}
 \centering
 \begin{tabular}{p{3cm}p{6cm}}
 \hline\hline
 Input & Description\\
 % NEED THE BUILD COMMANDS WITH DESCRIPTIONS
 \hline\hline
  Image from a URL & An image of a male  \\
 \hline\hline
  Image that was saved within the desktop & An image of Seth Rogan \\
 \hline
 \end{tabular}
\subsubsection{Results}
\end{flushleft}
 \centering
 \begin{tabular}{||p{1.5cm}|p{1.5cm}|p{1.5cm}||}
 \hline
 \textbf Test & \textbf Time & \textbf Result \\
 \hline\hline
  Uploading the image from a URL & 28 seconds  & Pass\\
 \hline\hline
  Uploading an image from desktop & 29 seconds & Pass\\
 \hline
 \end{tabular}
\vspace{1cm}

\subsection{Deep Learning Model Response Time}
\subsubsection{Description}
\begin{flushleft}
The web interface should be able to connect to an external database and store or query items. In order for this test to be considered a pass the confirmation of the image being uploaded would have to occur within 30 seconds so that additional resources are not wasted by the database.

\subsubsection{Input Data}
 \centering
 \begin{tabular}{p{3cm}p{6cm}}
 \hline\hline
 Input & Description\\
 % NEED THE BUILD COMMANDS WITH DESCRIPTIONS
 \hline\hline
  Image from a URL & An image of a male  \\
 \hline\hline
  Image that was saved within the desktop & An image of Seth Rogan \\
 \hline
 \end{tabular}
\subsubsection{Results}
\end{flushleft}
 \centering
 \begin{tabular}{||p{1.5cm}|p{1.5cm}|p{1.5cm}||}
 \hline
 \textbf Test & \textbf Time & \textbf Result \\
 \hline\hline
  Uploading the image from a URL & 28 seconds  & Pass\\
 \hline\hline
  Uploading an image from desktop & 29 seconds & Pass\\
 \hline
 \end{tabular}
\vspace{1cm}

 \subsection{Website Search Responsiveness}
\subsubsection{Description}
\begin{flushleft}
When given a word or phrase the web interface will be able to respond with an image or video of a pose or action within a two minutes.
\subsection{Input Data}
 \centering
 \begin{tabular}{p{3cm}p{6cm}}
 \hline\hline
 Input & Description\\
 % NEED THE BUILD COMMANDS WITH DESCRIPTIONS
 \hline\hline
  Creepy & Searched using a tag within the description\\ %fill in these entries once we have them
 \hline\hline
  Seth Rogan & Searched using a tag within the description\\
 \hline
 \end{tabular}
\subsubsection{Results}
\end{flushleft}
 \centering
 \begin{tabular}{||p{1.5cm}|p{1.5cm}|p{1.5cm}||}
 \hline
 \textbf Test & \textbf Time & \textbf Result \\
 \hline\hline
  Search for "Seth" & 1 second & Pass\\ %fill in these entries once we have them
 \hline\hline
  Search for "Creepy" & 2 seconds & Pass\\
 \hline
 \end{tabular}


\chapter{Other Relavent Testing}

\section{Presicion and Accuracy}
\subsection{Bone and Joint Position}
\subsubsection{Description}
\begin{flushleft}
The pose estimation should accurately predict the placement of joints and bones of the person in the provided photo. This will be determined with visual means with an uncertainty range of 20 pixels.
\end{flushleft}
\subsubsection{Input Data}
 \centering
 \begin{tabular}{p{3cm}p{6cm}}
 \hline\hline
 Input & Description\\
 % NEED THE BUILD COMMANDS WITH DESCRIPTIONS
 \hline\hline
   &  \\ %fill in these entries once we have them
 \hline
 \end{tabular}
\subsubsection{Results}

\section{Reliability and Availability Requirements}
\subsection{Software Availibility}
\subsubsection{Description}
\begin{flushleft}
The software component of the project should be availible at all times. If we can recieve correct output for a given input during a scheduling constraint then this will be considered a pass.
\end{flushleft}
\subsubsection{Input Data}
 \centering
 \begin{tabular}{p{3cm}p{6cm}}
 \hline\hline
 Input & Description\\
 % NEED THE BUILD COMMANDS WITH DESCRIPTIONS
 \hline\hline
   &  \\ %fill in these entries once we have them
 \hline
 \end{tabular}
\subsubsection{Results}

\subsection{Software Availibility}
\subsubsection{Description}
\begin{flushleft}
The software component of the project should be availible at all times with the exception of maintaince and migration. When we make a web server call we should recieve and HTTP verified response.
\end{flushleft}
\subsubsection{Input Data}
 \centering
 \begin{tabular}{p{3cm}p{6cm}}
 \hline\hline
 Input & Description\\
 % NEED THE BUILD COMMANDS WITH DESCRIPTIONS
 \hline\hline
   &  \\ %fill in these entries once we have them
 \hline
 \end{tabular}
\subsubsection{Results}

\section{Robustness or Fault-Tolerance Requirements}
\subsection{Web Interface Error Handling}
\subsubsection{Description}
\begin{flushleft}
The web interface should respond to unhandled exceptions by throwing the corresponding error messages. If an exception is thrown and an error message is displayed then the test is considered a pass.
\end{flushleft}
\subsubsection{Input Data}
 \centering
 \begin{tabular}{p{3cm}p{6cm}}
 \hline\hline
 Input & Description\\
 % NEED THE BUILD COMMANDS WITH DESCRIPTIONS
 \hline\hline
   &  \\ %fill in these entries once we have them
 \hline
 \end{tabular}
\subsubsection{Results}

\subsection{Web Interface Text Parsing}
\subsubsection{Description}
\begin{flushleft}
The web interface will be given unintelligable text which it must parse. The interface must respond with the corresponding error message.
\end{flushleft}
\subsubsection{Input Data}
 \centering
 \begin{tabular}{p{3cm}p{6cm}}
 \hline\hline
 Input & Description\\
 % NEED THE BUILD COMMANDS WITH DESCRIPTIONS
 \hline\hline
   &  \\ %fill in these entries once we have them
 \hline
 \end{tabular}
\subsubsection{Results}

\section{Capacity Requirements}

\subsection{Multiple Connections}
\subsubsection{Description}
\begin{flushleft}
The web interface should be able to serve multiple connections. If the interface can support 5 connections at once it is considered a pass.
\end{flushleft}
\subsubsection{Input Data}
 \centering
 \begin{tabular}{p{3cm}p{6cm}}
 \hline\hline
 Input & Description\\
 % NEED THE BUILD COMMANDS WITH DESCRIPTIONS
 \hline\hline
   &  \\ %fill in these entries once we have them
 \hline
 \end{tabular}
\subsubsection{Results}

\subsection{Database Capacity}
\subsubsection{Description}
\begin{flushleft}
The database should contain at least 5GB of data in order to facilitate growth.
\end{flushleft}
\subsubsection{Results}

\section{Scaling of Extensibility Requirements}
\subsection{Deep Learning Training}
\subsubsection{Description}
\begin{flushleft}
The deep learning model should be put through a rigorous test to be sure it is well trained. The test model is given a set of thousands of pictures. This combined with other tests will tell us if the model is well trained.
\end{flushleft}
\subsubsection{Input Data}
 \centering
 \begin{tabular}{p{3cm}p{6cm}}
 \hline\hline
 Input & Description\\
 % NEED THE BUILD COMMANDS WITH DESCRIPTIONS
 \hline\hline
   &  \\ %fill in these entries once we have them
 \hline
 \end{tabular}
\subsubsection{Results}

\section{Operational Environment Requirements}

\subsection{Linux Friendly Tensorflow}
\subsubsection{Description}
\begin{flushleft}
The web interface should be run on a Linux friendly server that can access the Tensorflow model either directly or indirectly. By creating an interface that successfully runs on a Apache or NGINX server.
\end{flushleft}
\subsubsection{Results}
 \centering
 \begin{tabular}{||p{2.5cm}|p{2.5cm}||}
 \hline
 \textbf Test & \textbf Result\\
 \hline\hline
   &  \\ %fill in these entries once we have them
 \hline
 \end{tabular}

\subsection{Tensorflow Library and Model}
\subsubsection{Description}
\begin{flushleft}
The web interface should interact with the Tensorflow library, as the deep learning model cannot be run on the web interface alone. To pass we should be able to have the Tensorflow model on a picture or video and correctly determine if there is a person in it.
\end{flushleft}
\subsubsection{Input Data}
 \centering
 \begin{tabular}{p{3cm}p{6cm}}
 \hline\hline
 Input & Description\\
 % NEED THE BUILD COMMANDS WITH DESCRIPTIONS
 \hline\hline
   &  \\ %fill in these entries once we have them
 \hline
 \end{tabular}
\subsubsection{Results}

\subsection{Export Types}
\subsubsection{Description}
\begin{flushleft}
The project should be able to export multiple types of media (JPEG, PNG, etc) in order to support all major operating
systems.
\end{flushleft}
\subsubsection{Results}
 \centering
 \begin{tabular}{||p{2.5cm}|p{2.5cm}||}
 \hline
 \textbf Test & \textbf Result\\
 \hline\hline
   &  \\ %fill in these entries once we have them
 \hline
 \end{tabular}

\chapter{Traceability} % FILL THIS IF WE HAVE TIME
\chapter{Changes After testing}

\end{document}
