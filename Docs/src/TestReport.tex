\documentclass{scrreprt}
\usepackage{listings}
\usepackage{underscore}
\usepackage{ragged2e}
\usepackage[bookmarks=true]{hyperref}
\usepackage[utf8]{inputenc}
\usepackage[english]{babel}
\usepackage{xcolor}
\usepackage{amsmath,amssymb}
\usepackage{indentfirst}
\usepackage[section]{placeins}
\usepackage{graphicx}
\usepackage{graphics}
\usepackage{longtable}
\usepackage{booktabs}
\usepackage{xcolor} % for different colour comments
\usepackage{tabto}
\usepackage{array}
\usepackage{mdframed}
\mdfsetup{nobreak=true}
\usepackage{xkeyval}
\usepackage{tabularx}
\hypersetup{
    colorlinks,
    citecolor=black,
    filecolor=black,
    linkcolor=red,
    urlcolor=blue
}
\usepackage[skip=2pt, labelfont=bf]{caption}
\usepackage{titlesec}
\usepackage{float}
\graphicspath{ {image/} }

\titleformat{\paragraph}
{\normalfont\normalsize\textbfseries}{\theparagraph}{1em}{}
\titlespacing*{\paragraph}
{0pt}{3.25ex plus 1ex minus .2ex}{1.5ex plus .2ex}


%% Comments
\newif\ifcomments\commentstrue

\ifcomments
\newcommand{\authornote}[3]{\textcolor{#1}{[#3 ---#2]}}
\newcommand{\todo}[1]{\textcolor{red}{[TODO: #1]}}
\else
\newcommand{\authornote}[3]{}
\newcommand{\todo}[1]{}
\fi

\newcommand{\wss}[1]{\authornote{magenta}{SS}{#1}}
\newcommand{\ds}[1]{\authornote{blue}{DS}{#1}}

\begin{document}
\title{\textbf Text to Motion Database\\[\baselineskip]\Large Test Report}
\author{Brendan Duke\\Andrew Kohnen\\Udip Patel\\David Pitkanen\\Jordan Viveiros}
\date{\today}

\maketitle
\tableofcontents
% \listoftables
% \listoffigures
\newpage

\pagenumbering{arabic}


\chapter{Preface}
\section{Revision History}
\begin{table}[H]
\caption{\textbf Revision History}
\begin{tabularx}{\textwidth}{p{3.5cm}p{2cm}X}
\toprule {\textbf Date} & {\textbf Version} & {\textbf Notes}\\
\midrule
March 14, 2017 & 0.0 & File created\\
March 23, 2017 & 0.1 & Initial template completed \\
March 26, 2017 & 0.2 & Completed tables and organized sections\\
\bottomrule
\end{tabularx}
\end{table}
\section{List of Figures}
This document does not utilize figures in order to display our results.
\listoftables

\chapter{Introduction}

\section{Purpose of Document}

The purpose of this manuscript is to document the testing that has been
performed on the Text-to-Motion web application. The testing that has been
performed largely follows the test plan that was provided in the Test Plan document.

\section{Scope of Testing}

Following good design procedures, the Text-To-Motion web application has been
modularized into $3$ conceptual modules.  The first module is a web framework that allows users to access key features like uploading an image, view a previous image, among others. The database is the second module in the application as it is used to store events from the website like the previously listed features. The third module is the most expansive and is the deep learning module that runs pose estimation on the images and pictures uploaded.

Following the application's modular decomposition, the tests have been
decomposed into $3$ conceptual tests, in addition to automated testing.

The first stage of testing is the minimum viable product, and was done to ensure
that a base level of functionality was provided. This is where we test to see that all of the software modules have been created and can work together. With this in mind the types of testing that were done to ensure a minimum viable product involved both system and unit testing.

The second set of tests is Solution Constraints testing or Functional tests.  In this set of tests, the performance of the deep learning algorithm is rigorously tested
and quantified. In addition to testing the deep learning algorithm we test the database, and search capabilities with more specific parameters than those introduced in the minimum viable product.

In the third set of tests, we get a few users to test the non-functional requirements
of the finished product. Examples of these requirements include: usability, look and feel requirements and learnability.

The final set of tests were the automated tests ran on the http-server in order to verify the GET and POST calls to and from the server. 

The final section in this document includes a traceability table that helps to
organize and explain how tests are connected to the requirements that were
created in the software requirements document we created earlier.

%I have added a few examples of what the tables should be.
\chapter{Preliminary Testing}

\section{Minimum Viable Product}
%An example of non functional requirements. In test cases where we we want an average we add an additional cell in bold
\subsection{TextToMotion Availability}

\subsubsection{Description}

This test is done to ensure that the TextToMotion Database (https://brendanduke.ca/) is functioning as intended and available to the public. These tests will be done by first accessing the home page of https://brendanduke.ca/, and then navigating to other pages within the website to ensure they are meeting the availability requirements.

\subsubsection{Results}

\begin{table}[H]
        \centering
        \begin{tabular}[t]{||p{0.75cm}|p{4cm}|p{2.5cm}|p{3cm}|p{2.5cm}|p{1cm}||}
                \hline
                \textbf Test Num & \textbf Test & \textbf Initial State & \textbf Expected Output & \textbf Actual Output & \textbf Result\\
                \hline\hline
                3.1 & Input brendanduke.ca to view the home page & Google.ca & The home page of brendanduke.ca & The home page of brendanduke.ca & Pass \\
                \hline
                3.2 & Add ../Account/Register to the URL or click "Register" & brendanduke.ca & The user registration page & The user registration page & Pass \\
                \hline
                3.3 & Add ../Account/Login to the current URL or click "Login" & brendanduke.ca & The user sign in page & The user sign in page & Pass \\
                \hline
                3.4 & Add ../ImagePoseDraw to the current URL or click "ImagePoseDraw" & brendanduke.ca & The ImagePoseDraw page & The ImagePoseDraw page & Pass \\
                \hline
                3.5 &  Add ../Demo to the current URL or click "Demo" & brendanduke.ca & The Demo page with the camera feed & The Demo page with the camera feed & Pass \\
                \hline
                3.6 &  Add ../TextToMotion to the current URL or click "TextToMotion" & brendanduke.ca & The TextToMotion page & The TextToMotion page & Pass \\
                \hline
        \end{tabular}
\end{table}

\subsection{Updating the database}
\subsubsection{Description}

The database allows users to upload images to the database through the website through brendanduke.ca/ImagePoseDraw. The test will be considered a pass if an image can be uploaded successfully and then viewed through the interface within ImagePoseDraw. 

\subsubsection{Input Data}

Throughout this section the input data is referenced by "Name".jpeg and the website location is referenced through ../ImagePoseDraw which represents brendanduke.ca/ImagePoseDraw.

\begin{table}[H]
        \centering
        \begin{tabular}{p{3cm}p{6cm}}
                \hline\hline
                Input & Description\\
                \hline\hline
                AverageGuy.jpg &  A picture of a man standing still and facing the camera\\ %fill in these entries once we have them
                \hline
                AverageGirl.jpg &  A picture of a woman standing still and facing the camera\\ %fill in these entries once we have them
                \hline
                TomBrady.jpg &  A picture of football star Tom Brady\\ %fill in these entries once we have them
                \hline
        \end{tabular}
\end{table}

\subsubsection{Results}

In order to get to the URL brendanduke.ca/ImagePoseDraw/Create the user can navigate to brendanduke.ca/ImagePoseDraw and click the buttons associated with adding a new image or video. After the upload is complete a user will be taken back to the ImagePoseDraw where they can search for the uploaded media.

\begin{table}[H]
        \centering
        \begin{tabular}[t]{||p{0.75cm}|p{4cm}|p{2.5cm}|p{3cm}|p{2.5cm}|p{1cm}||}
                \hline
                \textbf Test Num & \textbf Test & \textbf Initial State & \textbf Expected Output & \textbf Actual Output & \textbf Result\\
                \hline\hline
                3.6 & Upload TomBrady.jpeg using the ImagePoseDraw functionality & .. /ImagePoseDraw/Create & Land on ../ImagePoseDraw & Landed on ../ImagePoseDraw & Pass\\
                \hline
                3.7 & Upload AverageGirl.jpeg using the ImagePoseDraw functionality & .. /ImagePoseDraw/Create & Land on ../ImagePoseDraw & Landed on ../ImagePoseDraw & Pass\\
                \hline
                3.8 & Upload AverageGuy.jpeg using the ImagePoseDraw functionality & .. /ImagePoseDraw/Create & Land on ../ImagePoseDraw & Landed on ../ImagePoseDraw & Pass\\
                \hline
        \end{tabular}
\end{table}

\subsection{Verifying Pose Estimation}
\subsubsection{Description}

After the user has uploaded an image for pose estimation they should be able to view the pose estimation that has been run on the image. In order to verify that some form of pose estimation has been run on the image the user must view their upload and have visible annotations made to the image to be considered a pass.

\subsubsection{Results}

An image once uploaded can be found within brendanduke.ca/ImagePoseDraw/Details/N, where N represents the number of the uploaded image. This can also be achieved by searching through the uploads.

\begin{table}[H]
        \centering
        \begin{tabular}[t]{||p{0.75cm}|p{4cm}|p{2.5cm}|p{3cm}|p{2.5cm}|p{1cm}||}
                \hline
                \textbf Test Num & \textbf Test & \textbf Initial State & \textbf Expected Output & \textbf Actual Output & \textbf Result\\
                \hline\hline
                3.9 & View TomBrady.jpeg from ../ImagePoseDraw/Details/30 & .. /ImagePoseDraw & An image of Tom Brady with pose estimated limbs & Tom Brady with pose estimated limbs & Pass\\
                \hline
                3.10 & View AvergaeGirl.jpeg from ../ImagePoseDraw/Details/26 & .. /ImagePoseDraw & An image of a female with pose estimated limbs & A female with pose estimated limbs & Pass\\
                \hline
                 3.11 & View AverageGuy.jpeg from ../ImagePoseDraw/Details/27 & .. /ImagePoseDraw & An image of a male with pose estimated limbs & A male with pose estimated limbs & Pass\\
                \hline
        \end{tabular}
\end{table}

\subsection{Search by Name or Description}
\subsubsection{Description}

The web interface has the ability to search through the uploaded images and videos based on their name or information within the associated description. The test will be considered a pass if the name is input in the search bar and image is returned.

\subsubsection{Results}

\begin{table}[H]
        \centering
        \begin{tabular}[t]{||p{0.75cm}|p{4cm}|p{2.5cm}|p{3cm}|p{2.5cm}|p{1cm}||}
                \hline
                \textbf Test Num & \textbf Test & \textbf Initial State & \textbf Expected Output & \textbf Actual Output & \textbf Result\\
                \hline\hline
                3.12 & Input  "Tom" in the search bar within ../ImagePoseDraw & .. /ImagePoseDraw & A list of figured including the Tom Brady image that was uploaded & A single return of the Tom Brady image & Pass\\
                \hline
                3.13 & Input "Average" in the search bar within ../ImagePoseDraw & .. /ImagePoseDraw & A list of results including the AverageGirl image & Two results, one of which was Average Girl & Pass\\
                \hline
                 3.14 & Input "Guy" in the search bar within ../ImagePoseDraw & .. /ImagePoseDraw & A list of results including the AverageGuy image & Two results, one of which was Average Guy & Pass\\
                \hline
        \end{tabular}
\end{table}

\section{Solution Constraints Testing}

\subsection{Deep Learning Methods Test}
\subsubsection{Description}

In order to provide a proper demonstration of the deep learning mechanics that can be associated with pose estimation we will meet with Dr.Taylor to ensure that we are implementing the Bulat et al paper properly. This test will be considered a pass if Dr.Taylor is satisfied with the implementation of the previously mentioned paper.

\subsubsection{Results}

\begin{table}[H]
        \centering
        \begin{tabular}{||p{1cm}|p{7.5cm}|p{1cm}||}
                \hline
                \textbf Test Num & \textbf Test & \textbf Result\\
                \hline\hline
                3.15 & Meet with Dr.Taylor in order to verify the integrity of the deep learning implementation within the TextToMotion Database. & Pass\\ %fill in these entries once we have them
                \hline
        \end{tabular}
\end{table}

\subsection{Standard Data Format Test}

\subsubsection{Description}

An automated test that checks if the human pose data used for the project is
standard and compatible with existing software libraries.

\subsubsection{Results}

This test was technically a failure. The data was not converted to a format
compatible with libraries. Instead the data is stored in JSON strings, which is
a standard format, but not compressed as an ideal format such as HDF5 would be.

\chapter{Functional Requirements}

\section{Input Data}

Throughout this section the input data is referenced by "Name".jpeg and the website location is referenced through ../ImagePoseDraw which represents brendanduke.ca/ImagePoseDraw.

\begin{table}[H]
        \centering
        \begin{tabular}{p{3cm}p{6cm}}
                \hline\hline
                Input & Description\\
                \hline\hline
                E9FY2.MP4  &  A short video of a woman eating a sandwich\\
                \hline
                U4XV9.MP4  &  A short video of a man waking up and getting out of bed\\
                \hline
                Z1A0Q.MP4 & A short video of a man sitting on a stool\\
                \hline
        \end{tabular}
\end{table}

\section{Supported Video Encoding Test}

\subsection{Description}

Tests whether the ReadFrames API is able to decode MP4 files. If we are able to
run pose estimation on the video, then the ReadFrames API is able to process the
frames.

\subsubsection{Results}

\begin{table}[H]
        \centering
        \begin{tabular}[t]{||p{0.75cm}|p{4cm}|p{2.5cm}|p{3cm}|p{2.5cm}|p{1cm}||}
                \hline
                \textbf Test Num & \textbf Test & \textbf Initial State & \textbf Expected Output & \textbf Actual Output & \textbf Result\\
                \hline\hline
                4.1 & Running ReadFrames API on E9FY2.MP4, to run pose estimation & ReadFrame API & A compiled E9FY2.MP4 with pose estimation executed & E9FY2.MP4 that has been pose estimated & Pass\\
                \hline
                4.2 & Running ReadFrames API on U4XV9.MP4, to run pose estimation & ReadFrame API & A compiled U4XV9.MP4 with pose estimation executed & U4XV9.MP4 that has been pose estimated & Pass\\
                \hline
                4.3 & Running ReadFrames API on Z1A0Q.MP4, to run pose estimation & ReadFrame API & A compiled Z1A0Q.MP4 with pose estimation executed & Z1A0Q.MP4 that has been pose estimated & Pass\\
                \hline
        \end{tabular}
\end{table}

\section{Frame Reading Timestamp Accuracy Test}
\subsection{Description}

Tests whether the timestamps on the frames returned by the ReadFrames API match
their temporal position in the original video stream. Our input data is
identical to the previous test.

\subsection{Results}

\begin{table}[H]
        \centering
        \begin{tabular}[t]{||p{0.75cm}|p{4cm}|p{2.5cm}|p{3cm}|p{2.5cm}|p{1cm}||}
                \hline
                \textbf Test Num & \textbf Test & \textbf Initial State & \textbf Expected Output & \textbf Actual Output & \textbf Result\\
                \hline\hline
                4.4 & Run ReadFrames API on E9FY2.MP4, to verify the timestamps match & ReadFrame API & Matching timestamps at 5 seconds & Matching timestamps at 5 seconds & Pass\\
                \hline
                4.5 & Run ReadFrames API on U4XV9.MP4, to verify the timestamps match & ReadFrame API & Matching timestamps at 10 seconds & Matching timestamps at 10 seconds & Pass\\
                \hline
                4.6 & Run ReadFrames API on Z1A0Q.MP4, to verify the timestamps match & ReadFrame API & Matching timestamps at 6 seconds & Matching timestamps at 6 seconds  & Pass\\
                \hline
        \end{tabular}
\end{table}

\section{Human Pose Estimation Data Quality Test}
\subsection{Description}

Test to ensure the data quality produced by the human pose estimator component
was acceptable.

A set Charades videos will be processed by the human pose estimator, and
skeleton animations corresponding to the generated human pose data will be
created (this is a scoped part of the software pipeline). A double-blind test
will be ran, wherein testers will be shown random mixed sets of the skeleton
animations produced by McMaster Text to Motion, together with skeletons from
actual motion capture data coming from CMU'’s motion capture lab. Testers will
indicate whether they think the motion capture data came from actual motion
capture, or from the pose estimation software.

The McMaster Text to motion Results should be guessed as accurate at similar
rates to the Charades tests.

\subsection{Results}

We provide the ratio of Text-to-Motion images guessed as accurate compared to
the Charades Images.

\begin{table}[H]
        \centering
        \begin{tabular}{||p{7.5cm}|p{2.5cm}|p{2.5cm}||}
                \hline
                \textbf Test & \textbf Charades/ TextToMotion & \textbf Result\\
                \hline\hline
                Nick was shown a set of videos and asked to determine which was generated by the TextToMotion Database &  1/1 & Pass\\
                \hline
                Drew was shown a set of videos and asked to determine which was generated by the TextToMotion Database&  7/5  & Pass\\
                \hline
                Sarah was shown a set of videos and asked to determine which was generated by the TextToMotion Database& 5/4 & Pass\\
                \hline
        \end{tabular}
\end{table}

\section{Database Output Full Range Coverage Test}
\subsection{Description}

Tests to be sure all entries in the database can be successfully searched for.
The videos provided from earlier tests are put into the database, and have been
renamed for testing purposes.

\subsection{Input Data}

Throughout this section the input data is referenced by "Name".jpeg and the website location is referenced through ../ImagePoseDraw which represents brendanduke.ca/ImagePoseDraw.

\begin{table}[H]
        \centering
        \begin{tabular}{p{3cm}p{6cm}}
                \hline\hline
                Input & Description\\
                % NEED THE BUILD COMMANDS WITH DESCRIPTIONS
                \hline\hline
                \verb|Waking_Up.mp4| &  Given tags sleeping, boy, man, sleepy, getting up, table\\
                \hline
                \verb|Eating.mp4| &  Given tags sandwich, eating, girl, woman, table\\
                \hline
                \verb|Stool.mp4| &  Given tags man, stool, corner, sitting\\
                \hline
        \end{tabular}
\end{table}

\subsubsection{Results}

The given set of videos appeared in the returned list of videos from a database
search.

\begin{table}[H]
        \centering
        \begin{tabular}[t]{||p{0.75cm}|p{4cm}|p{2.5cm}|p{3cm}|p{2.5cm}|p{1cm}||}
                \hline
                \textbf Test Num & \textbf Test & \textbf Initial State & \textbf Expected Output & \textbf Actual Output & \textbf Result\\
                \hline\hline
                4.7 & Search for the Waking_Up.mp4 within ../ImagePoseDraw by searching "Waking" & .. /ImagePoseDraw & A list of results with Waking_Up in the results & A single result of the Waking_up.mp4 & Pass\\
                \hline
                4.8 & Search for the Eating.mp4 within ../ImagePoseDraw by searching "Eat" & .. /ImagePoseDraw & A list of results with Eating in the results & A single result of the Eat.mp4 & Pass\\
                \hline
                4.9 & Search for the Stool.mp4 within ../ImagePoseDraw by searching "Stool" & .. /ImagePoseDraw & A list of results with Stool in the results & A single result of the Stool.mp4 & Pass\\
                \hline
        \end{tabular}
\end{table}

\section{Database No False Positives}
\subsection{Description}

Tests that the database search does not return any false positives, such as
videos or images that do not contain searched words. The same videos from the
previous test will be used with the same tags. Thus, we will search with tags
other than those provided. If no videos appear, then the test is a success.

\subsubsection{Results}

Using the nonsensical keywords from our input data, no search results were
returned, meaning that the test passed.

\begin{table}[H]
        \centering
        \begin{tabular}[t]{||p{0.75cm}|p{4cm}|p{2.5cm}|p{3cm}|p{2.5cm}|p{1cm}||}
                \hline
                \textbf Test Num & \textbf Test & \textbf Initial State & \textbf Expected Output & \textbf Actual Output & \textbf Result\\
                \hline\hline
                4.10 & Search for "Raptor" within ../ImagePoseDraw & .. /ImagePoseDraw & A list of results that don't contain the input data & No results were returned & Pass\\
                \hline
                4.11 & Search for "Exponent" within ../ImagePoseDraw & .. /ImagePoseDraw & A list of results that don't contain the input data & No results were returned & Pass\\
                \hline
                4.12 & Search for "Glasses" within ../ImagePoseDraw & .. /ImagePoseDraw & A list of results that don't contain the input data & No results were returned & Pass\\
                \hline
        \end{tabular}
\end{table}

%For tests that require a time use this
\chapter{Non-Functional Requirements}
\section{Usability}
\subsection{Description}
In order to determine the usability of the Text-to-Motion database, a small
sample of users were asked to use the website to perform some predetermined
actions and answer questions afterwards.

Before the participants were asked to preform any actions they were given a
minute to familiarize themselves with the interface, but were not given any
guidance or tips from the development team.

Once the time was up they were asked to upload an image on mobile or desktop
through their webcam, a URL or from a file saved within the computer.

While performing the required action the participant's time was recorded and
used to determine if a requirement had passed or failed.

Upon completion of the task the users were asked to rate the style and design
of the website on a scale from 1 to 10.

\subsection{Results}

The results from the participants can be seen throughout the Non-Functional
Requirements along with a pass or fail based on the requirements description.

\section{Look and Feel Requirements}
\subsection{Colour Scheme}
\subsubsection{Description}

A test to see if the colour scheme of the website is visually appealing. Making the colour scheme of the website visually appealing to a larger audience is important as it will help keep users on the website and ideally provide a reason to reference it to a friend or colleague.

When testing the colour scheme anything above a 6 was considered a pass. We chose anything above a 6 because if the users were to rate it above a 6 it means they are within the upper percentage and enjoyed the colour scheme. This may result in them recommending the website to others and providing more traffic.

\subsubsection{Results}

\begin{table}[H]
        \centering
        \begin{tabular}{||p{0.75cm}|p{2.5cm}|p{2.5cm}|p{2.5cm}||}
                \hline
                \textbf Test Num & \textbf User & \textbf Rating & \textbf Result\\
                \hline\hline
                5.1 & Nick & 7 & Pass \\
                \hline
                5.2 & Drew & 9 & Pass\\ %fill in these entries once we have them
                \hline
                5.3 & Sarah & 7 & Pass \\
                \hline
        \end{tabular}
\end{table}

\section{Style Requirements}

\subsection{Minimalistic Web Design}
\subsubsection{Description}

The website interface should be minimal and should inform the user of valid
actions through visual means. This is a modern requirement in website design, the more minimalistic design you can provide the better it will be for the user experience and any new technologies use to enforce this will help drive traffic.

While testing the minimalistic design anything above a 7 was considered a pass. This required a higher rating then the colour scheme as we believed it would be a key factor in providing the user with an enjoyable experience and help provide a greater amount of traffic through uploads.

\subsubsection{Results}

\begin{table}[H]
        \centering
        \begin{tabular}{||p{0.75cm}|p{2.5cm}|p{2.5cm}|p{2.5cm}||}
                \hline
                \textbf Test Num & \textbf User & \textbf Rating & \textbf Result\\
                \hline\hline
                5.4 & Nick & 9 & Pass \\
                \hline
                5.5 & Drew & 9 & Pass\\ %fill in these entries once we have them
                \hline
                5.6 & Sarah & 8 & Pass \\
                \hline
        \end{tabular}
\end{table}

\section{Ease of Use Requirements}
\subsection{Upload/Download}
\subsubsection{Description}

Through the web interface a user should be able to upload a picture using
either a mobile phone camera, URL, or saved file. The participant will start on
the Home page and be asked to upload an image through one of the methods previously
mentioned. 

In order for this to be considered a pass it should take the users 45 seconds or less to complete the upload process and click the button. The required time of 45 seconds was chosen due to the amount of time that was required to find an image, URL, or file  while running trials throughout the development.

\subsubsection{Results}

\begin{table}[H]
        \centering
        \begin{tabular}[t]{||p{0.75cm}|p{4cm}|p{2.5cm}|p{3cm}|p{2.5cm}|p{1cm}||}
                \hline
                \textbf Test Num & \textbf Test & \textbf Initial State & \textbf User & \textbf Time & \textbf Result\\
                \hline\hline
                5.7 & Upload an image from a URL & .. /ImagePoseDraw/Create & Nick & 32 seconds & Pass\\
                \hline
                5.8 & Upload an image from a mobile device & .. /ImagePoseDraw/Create & Nick & 38 seconds & Pass\\
                \hline
                5.9 & Upload a file stored within the Desktop & .. /ImagePoseDraw/Create & Nick & 27 seconds & Pass\\
                \hline
        \end{tabular}
\end{table}

\subsection{Text Box Functionality}
\subsubsection{Description}

The user should be able to input a descriptive word or phrase into a text-box
from within the web interface in order to search for a video. With a the website being part of a larger database the ability to search for an image that the user just uploaded is an important factor in determining the usability for users.

In order to complete this task the users were asked to search for a specific word and
display the results. Any time below 10 seconds will be considered a pass. Each user was allocated 10 seconds because we did not want the action of typing the keyword to be the bottleneck when searching for a database entry.

\subsubsection{Results}

\begin{table}[H]
        \centering
        \begin{tabular}[t]{||p{0.75cm}|p{4cm}|p{2.5cm}|p{3cm}|p{2.5cm}|p{1cm}||}
                \hline
                \textbf Test Num & \textbf Test & \textbf Initial State & \textbf User & \textbf Time & \textbf Result\\
                \hline\hline
                5.10 & Search for "Woman" & .. /ImagePoseDraw & Nick & 4 seconds & Pass\\
                \hline
                5.11 & Search for "f" & .. /ImagePoseDraw & Drew & 3 seconds & Pass\\
                \hline
                5.12 & Saerch for "the" & .. /ImagePoseDraw & Sarah & 4 seconds & Pass\\
                \hline
        \end{tabular}
\end{table}

\section{Learning Requirements}

\subsection{Usability Tests}
\subsubsection{Description}

The user should be able to interact with the website without prior knowledge.
They will given a minute to explore the website. After that time the
participants were asked to rate the usability on a scale of 1-10. An average of
6 is required for a pass.

\subsubsection{Results}

The users' usability ratings allowed the site to pass this test.

\begin{table}[H]
        \centering
        \begin{tabular}{||p{2.5cm}|p{2.5cm}|p{2.5cm}||}
                \hline
                \textbf User & \textbf Rank & \textbf Result\\
                \hline\hline
                Nick & 6 & Pass \\
                \hline
                Drew & 8 & Pass \\
                \hline
                Sarah & 8 & Pass\\
                \hline
        \end{tabular}
\end{table}

\section{Politeness and Understandability Requirements}
\subsection{Hiding the Inner Workings}
\subsubsection{Description}

Users should not be able to see the deep learning model and its training when
using the pose estimation. When prompted the website should display the correct
skeletons without any low-level detail. Once uploaded the participants were
asked if they saw anything that seemed out of place or any information on the
deep learning process, if they did not it will be considered a pass.

\subsubsection{Results}

Users indicated that the deep learning model was encapsulated from their view,
and hence this test passed.

\begin{table}[H]
        \centering
        \begin{tabular}{||p{2.5cm}|p{2.5cm}|p{2.5cm}||}
                \hline
                \textbf Test & \textbf User & \textbf Result\\
                \hline\hline
                Uploading an image from URL & Nick & Pass\\
                \hline\hline
                Uploading an image from mobile & Drew & Pass\\
                \hline\hline
                Uploading an image from desktop & Sarah & Pass\\
                \hline
        \end{tabular}
\end{table}

\section{Speed and Latency Testing}

\subsection{External Database Connection Response Time}
\subsubsection{Description}

The web interface should be able to connect to an external database and store
or query items. In order for this test to be considered a pass the confirmation
of the image being uploaded would have to occur within 30 seconds so that
additional resources are not wasted by the database. Testing this will occur by
uploading an image and testing the total time taken.

\subsubsection{Input Data}

\begin{table}[H]
        \centering
        \begin{tabular}{p{3cm}p{6cm}}
                \hline\hline
                Input & Description\\
                % NEED THE BUILD COMMANDS WITH DESCRIPTIONS
                \hline\hline
                Image from a URL & An image of a male  \\
                \hline\hline
                Image that was saved within the desktop & An image of Seth Rogan \\
                \hline
        \end{tabular}
\end{table}

\subsubsection{Results}

According to the response times of this automated test, the database queries
were executed fast enough for the test to pass.

\begin{table}
        \centering
        \begin{tabular}{||p{1.5cm}|p{1.5cm}|p{1.5cm}||}
                \hline
                \textbf Test & \textbf Time & \textbf Result \\
                \hline\hline
                Uploading the image from a URL & 28 seconds  & Pass\\
                \hline
                Uploading an image from desktop & 29 seconds & Pass\\
                \hline
        \end{tabular}
\end{table}
\vspace{1cm}

\subsection{Website Search Responsiveness}
\subsubsection{Description}

When given a word or phrase the web interface will be able to respond with an
image or video of a pose or action within a two minutes.

\subsubsection{Input Data}

\begin{table}[H]
        \centering
        \begin{tabular}{p{3cm}p{6cm}}
                \hline\hline
                Input & Description\\
                \hline\hline
                Creepy & Searched using a tag within the description\\ %fill in these entries once we have them
                \hline
                Seth Rogan & Searched using a tag within the description\\
                \hline
        \end{tabular}
\end{table}

\subsubsection{Results}

\begin{table}[H]
        \centering
        \begin{tabular}{||p{1.5cm}|p{1.5cm}|p{1.5cm}||}
                \hline
                \textbf Test & \textbf Time & \textbf Result \\
                \hline\hline
                Search for ``Seth'' & 1 second & Pass\\ %fill in these entries once we have them
                \hline\hline
                Search for ``Creepy'' & 2 seconds & Pass\\
                \hline
        \end{tabular}
\end{table}


\chapter{Other Relevant Testing}

Again we test this on the same files we have used in the previous tests:
\verb|TomBrady.jpg|, \verb|AverageGirl.jpg| and \verb|AverageGuy.jpg|.

\section{Precision and Accuracy}
\subsection{Bone and Joint Position}
\subsubsection{Description}

The pose estimation should accurately predict the placement of joints and bones
of the person in the provided photo. This will be determined with visual means
with an uncertainty range of 20 pixels.

\subsubsection{Input Data}

\begin{table}[H]
        \centering
        \begin{tabular}{p{3cm}p{6cm}}
                \hline\hline
                Input & Description\\
                % NEED THE BUILD COMMANDS WITH DESCRIPTIONS
                \hline\hline
                \verb|TomBrady.jpg| &  A picture of a man standing still and facing the camera\\
                \hline
                \verb|AverageGirl.jpg| &  A picture of a woman standing still and facing the camera\\
                \hline
                \verb|TomBrady.jpg| &  A picture of football star Tom Brady\\
                \hline
        \end{tabular}
\end{table}

\subsubsection{Results}

We were able to qualitatively confirm that these tests passed for the given
input images. A more rigorous ``PCKh'' metric is used to formally evaluate the
performance of our deep learning model on single-person pose estimation.

\begin{table}[H]
        \centering
        \begin{tabular}{||p{2.5cm}|p{2.5cm}||}
                \hline
                \textbf Test & \textbf Result\\
                \hline\hline
                Tom Brady & Pass  \\
                \hline\hline
                Average Girl & Pass  \\
                \hline\hline
                Average Guy & Pass  \\
                \hline
        \end{tabular}
\end{table}

\subsection{Deep Learning Model}
\subsubsection{Description}

The \textbf{PCKh} metric, used by the MPII Human Pose Dataset, defines a joint
estimate as matching the ground truth if the estimate lies within 50\% of the
head segment length. The head segment length is
defined as the diagonal across the annotated head rectangle in the MPII data,
multiplied by a factor of 0.6. Details can be found by examining the MATLAB
\href{http://human-pose.mpi-inf.mpg.de/results/mpii_human_pose/evalMPII.zip}{evaluation script}
provided with the MPII dataset.


If our model can achieve 80\% total PCKh then the test is considered a pass.

\subsubsection{Results}

Our model achieves 85\% PCKh, thus this test is a pass.

\begin{table}[H]
    \centering
    \begin{tabular}{||p{2.5cm}|p{2.5cm}||}
        \hline
        \textbf{Test} & \textbf {PCKh Value}\\
         \hline\hline
        r ankle & 68.907\% \\
        \hline
        r knee &  77.201\% \\
        \hline
        r hip & 83.583\% \\
        \hline
        l hip & 84.444\% \\
        \hline
        l knee & 77.419\% \\
        \hline
        l ankle & 69.055\% \\
        \hline
        pelvis & 89.776\% \\
        \hline
        thorax & 98.071\% \\
        \hline
        upper neck & 97.823\% \\
        \hline
        head top & 96.557\% \\
        \hline
        r wrist & 81.288\% \\
        \hline
        r elbow & 87.205\% \\
        \hline
        r shoulder & 92.918\% \\
        \hline
        l shoulder & 92.828\% \\
        \hline
        l elbow & 85.278\% \\
        \hline
        l wrist & 80.719\% \\
        \hline
        \textbf {Total PCKh} & \textbf {85.195}\% \\
        \hline
    \end{tabular}
\end{table}

\section{Reliability and Availability Requirements}
\subsection{Software Availability}
\subsubsection{Description}

The software component of the project should be available at all times. If we
can have an event regularly occur then it will be considered a pass. To do this
we arranged to have the pose estimation algorithm automatically called on to
process a single image at 4 hour intervals, and record the time.

\subsubsection{Results}

The software successfully processed the image at the specified 4 hour intervals
over a period of 2 days.

\subsection{Website Availability}
\subsubsection{Description}

The software component of the project should be available at all times with the
exception of maintenance and migration. When we make a web server call we should
receive an HTTP verified response. To do this we will have three users sending
a HTTP POST request to the server. If they receive a response the test is
considered a pass.

\subsubsection{Results}

All three users were able to receive responses to their HTTP POST requests,
therefore we consider this test a pass.

\section{Robustness or Fault-Tolerance Requirements}
\subsection{Web Interface Error Handling}
\subsubsection{Description}

The web interface should respond to unhandled exceptions by throwing the
corresponding error messages. If an exception is thrown and an error message is
displayed then the test is considered a pass.

\subsubsection{Input Data}

\begin{table}[H]
        \centering
        \begin{tabular}{p{3cm}p{6cm}}
                \hline\hline
                Input & Description\\
                % NEED THE BUILD COMMANDS WITH DESCRIPTIONS
                \hline\hline
                Scenario 1 &  Try uploading an image while the image storage service (Amazon S3) is not available. \\
                \hline
                Scenario 2 & Trying to search an image on the database while the database server is down. \\
                \hline
        \end{tabular}
\end{table}

\subsubsection{Results}

For Scenarios 1 and 2, appropriate exceptions were thrown and error pages
displayed to the user. Therefore this test is considered to have passed.

\section{Capacity Requirements}

\subsection{Multiple Connections}
\subsubsection{Description}

The web interface should be able to serve multiple connections. If the
interface can support 5 connections at once it is considered a pass.

\subsubsection{Input Data}
\begin{table}[H]
        \centering
        \begin{tabular}{p{3cm}p{6cm}}
                \hline\hline
                Input & Description\\
                % NEED THE BUILD COMMANDS WITH DESCRIPTIONS
                \hline\hline
                Andrew & The connection to the website corresponding to tester Andrew \\
                \hline
                Brendan & The connection to the website corresponding to tester Brendan \\
                \hline
                Udip & The connection to the website corresponding to tester Udip \\
                \hline
                Jordan & The connection to the website corresponding to tester Jordan \\
                \hline
                Dave & The connection to the website corresponding to tester Dave \\
                \hline
        \end{tabular}
\end{table}

\subsubsection{Results}

The 5 users were able to successfully connect to the website, and run queries
simultaneously without experiencing significant slow down or waiting.

\subsection{Database Capacity}
\subsubsection{Description}

The database should contain at least 5GB of data in order to facilitate growth.

\subsubsection{Results}

The database successfully handled us uploading a total of 5.1GB of data in the
form of pictures of various qualities. These pictures were of randomly acquired
photos from Google, which were combined to form the data used in this testing.

\section{Scaling of Extensibility Requirements}

\section{Operational Environment Requirements}

\subsection{Linux Friendly TensorFlow}
\subsubsection{Description}

The web interface should be run on a Linux friendly server that can access the
TensorFlow model either directly or indirectly. By creating an interface that
successfully runs on a Apache or nginx server this test will be considered a
pass.

\subsubsection{Results}

Our service is successfully running on a production Linux server using nginx.

\subsection{Export Types}
\subsubsection{Description}

The project should be able to export multiple types of media (JPEG, PNG, etc)
in order to support all major operating systems. We will use \verb|TomBrady.jpg| for
this test.

\subsubsection{Results}

This test was a guaranteed pass due to the fact that our website stores images
as a base64, meaning it can be converted into any type.

\begin{table}[H]
        \centering
        \begin{tabular}{||p{2.5cm}|p{2.5cm}||}
                \hline
                \textbf Test & \textbf Result\\
                \hline\hline
                PNG & Pass \\ %fill in these entries once we have them
                \hline
                JPEG & Pass \\ %fill in these entries once we have them
                \hline
                BMP & Pass \\ %fill in these entries once we have them
                \hline
        \end{tabular}
\end{table}

\chapter{Traceability} % FILL THIS IF WE HAVE TIME
\begin{center}
\begin{longtable}{>{\raggedright\arraybackslash}p{0.3\textwidth}>{\raggedright\arraybackslash}p{0.3\textwidth}>{\raggedright\arraybackslash}p{0.4\textwidth}}
\caption{Traceability Matrix for Test-Requirement Relationships}\label{Table_TestsAndRequirements}
\\\toprule
\textbf Test & \textbf Description & \textbf Requirement\\\midrule
 3.2, 4.2, 4.3, 5.8.2
& Tests which measured performance Accuracy of Deep Learning Algorithm
&  Req 1, 8,  23(Speed and Latency)\\
 4.1,  5.3.1, 5.7.1, 6.4.2 & System tests, measuring reliability of the web framework. &
Req 7,  12,  17, 20, 30, 38, 39
\\
4.4-4.6
 & Unit and Systems Tests Grading Database Search Performance & Req \# 9, 10, 11,
 \\
4.5, 6.3.1, 6.3.1, 6.4.1 & Security and Data Integrity Tests & Req \# 10, 27, 29 \\
4.1.1, 5.8.2, 6.4.1 & Proper Formatting Tests & Req \# 7, 23, 29
\\
4.2.1-5.8.1,6.1.1-6.2.2  & User Interface Tests & Req \# 13-21, 24-26
\\
\bottomrule
\end{longtable}
\end{center}
\section{Modules}
Similarly, the following is a traceability table explicitly relating test cases to modules:\\
\begin{table}
\caption{Tests and Modules Relationships}
\begin{center}
\begin{tabular}{p{6cm} p{4cm}}
\hline
\textbf Test \#  & Module \\
\hline
3.1(all subsections), 3.2.1, 4.4-4.6, 5(all sections), 6.2(all subsections) - 6.4(all subsections), 6.6.2 & ASP.NET and DB \\
\hline
3.1.3, 4.3, 5.8.2, 6.1, 6.5 & TensorFlow Models \\
\hline
3.13, 3.2.2, 4.1-4.2, 5.8.2, 6.2(all subsections), 6.6.1 & Python HTTP Server\\
\hline
\end{tabular}
\end{center}
\end{table}

\chapter{Changes After testing}

The first of our major changes would likely be to the website interface. Though
our testers reviewed it favourably -- there were numerous references -- but we
were also given consistent criticism that an alternate colour scheme might be a
slight improvement. As such we have agreed to experiment with those
improvements.

Testing also revealed that a JavaScript application to allow continuous requests
to the HTTP server as well as utilize a mobile platform would have greater
applicability to the Guelph team. This improvement would come alongside a
function on the website to display statistics. This would help insure greater
availability. The ability to track the database live would be both useful to
the testing team as well as an entertaining aspect for the product demo.

The last major improvement we wish to provide is an improved JavaScript plugin
for drawing the skeleton on top of an image. Testing revealed that there was
flicker with the drawn skeletons, and as such, could lead to false negatives,
where a skeleton that appears inaccurate is actually just limited by our
current skeleton drawing methods.

\end{document}
