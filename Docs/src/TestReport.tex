\documentclass{scrreprt}

\usepackage{xcolor} % for different colour comments
\usepackage{tabto}
\usepackage{mdframed}
\mdfsetup{nobreak=true}
\usepackage{xkeyval}
\usepackage{tabularx}
\usepackage{booktabs}
\usepackage{hyperref}
\hypersetup{
    colorlinks,
    citecolor=black,
    filecolor=black,
    linkcolor=red,
    urlcolor=blue
}
\usepackage[skip=2pt, labelfont=bf]{caption}
\usepackage{titlesec}
\usepackage{graphicx}
\usepackage[section]{placeins}
\graphicspath{ {image/} }

\titleformat{\paragraph}
{\normalfont\normalsize\bfseries}{\theparagraph}{1em}{}
\titlespacing*{\paragraph}
{0pt}{3.25ex plus 1ex minus .2ex}{1.5ex plus .2ex}


%% Comments
\newif\ifcomments\commentstrue

\ifcomments
\newcommand{\authornote}[3]{\textcolor{#1}{[#3 ---#2]}}
\newcommand{\todo}[1]{\textcolor{red}{[TODO: #1]}}
\else
\newcommand{\authornote}[3]{}
\newcommand{\todo}[1]{}
\fi

\newcommand{\wss}[1]{\authornote{magenta}{SS}{#1}}
\newcommand{\ds}[1]{\authornote{blue}{DS}{#1}}

\begin{document}
\title{\bf Text to Motion Database\\[\baselineskip]\Large Test Report}
\author{Brendan Duke\\Andrew Kohnen\\Udip Patel\\David Pitkanen\\Jordan Viveiros}
\date{\today}

\maketitle
\pagenumbering{roman}
\tableofcontents
% \listoftables
% \listoffigures


\begin{table}[bp]
\caption*{\bf Revision History}
\begin{tabularx}{\textwidth}{p{3.5cm}p{2cm}X}
\toprule {\bf Date} & {\bf Version} & {\bf Notes}\\
\midrule
March 14, 2017 & 0.0 & File created\\
\bottomrule
\end{tabularx}
\end{table}

\newpage

\pagenumbering{arabic}


\chapter{List of Tables}
    tables for a specific tests have been removed to improve readability

\chapter{List of Figures}
    Figures used in this document.

    Definitions and Acronyms

\chapter{Introduction}

\section{Purpose of Document}
    This document summarizes the testing and the findings of the tests for the Text to Motion project. This documentuses the implementation outlined in the test plan.

\section{Scope of Testing}

\section{Organization}
    Section 1 is our introduction and introduces this report. Section 2 describes the test cases as outlined in our test plan and their results. Section 4 describes traceability to the requirements. Section 4 describes what changes have been made in response to our testing.

    \textbf{Testing structure}

    Automated Testing

    \quad generating inputs

    Manual Testing

%I have added a few examples of what the tables should be.
\chapter{Testing}

\section{Proof of Concept}
%An example of non functional requirements. In test cases where we we want an average we add an additional cell in bold
\subsection{Functional website for pose estimation}
\subsubsection{Description}
The user should be able to run pose estimation of a video or image through the website interface
\subsubsection{Results}
%Any comments we wish to make or add can be added here should we wish to expand on the results. An overall summary can also be here
\begin{table}[h!]
 \centering
 \begin{tabular}{||p{2.5cm}|p{2.5cm}||}
 \hline
 \bf Test & \bf Result\\
 \hline\hline
   &  \\ %fill in these entries once we have them
 \hline
 \end{tabular}
 \label{table:1}
\end{table}

%test for functional requirements.
\section{Solution Constraints Testing}
\subsection{Standard Data Format Test}
\subsubsection{Description}
An automated test that checks if the human pose data used for the project is standard and compatible with existing software libraries.
\subsubsection{Input Data}
\begin{table}[h!]
 \centering
 \begin{tabular}{p{3cm}p{6cm}}
 \hline\hline
 Intput & Description\\
 \hline\hline
   &  \\ %fill in these entries once we have them
 \hline
 \end{tabular}
 \label{table:1}
\end{table}
\subsubsection{Results}
%Results are largely unchanged from nonfunctional requirements, though somtimes we might choose to simply describe the result without a table.

%For tests that require a time use this
\section{Non-Functional Requirements}
\subsection{Ease of Use Requirements 1: Upload/Download}
\subsubsection{Description}
Through the web interface a user should be able to upload then download a picture within 30 seconds
\subsubsection{Results}
Passing conditions are that this process can be completed in 30 seconds
\begin{table}[h!]
 \centering
 \begin{tabular}{||p{1.5cm}|p{1.5cm}|p{1.5cm}||}
 \hline
 \bf Test & \bf Time & \bf Result \\
 \hline\hline
   &  & \\ %fill in these entries once we have them
 \hline
 \end{tabular}
 \label{table:1}
\end{table}

\end{document}
