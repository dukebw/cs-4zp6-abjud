\documentclass{scrreprt}
\usepackage{listings}
\usepackage{underscore}
\usepackage{ragged2e}
\usepackage[bookmarks=true]{hyperref}
\usepackage[utf8]{inputenc}
\usepackage[english]{babel}
\usepackage{xcolor}
\usepackage{amsmath,amssymb}
\usepackage{indentfirst}
\usepackage[section]{placeins}
\usepackage{graphicx}
\usepackage{graphics}
\usepackage{longtable}
\usepackage{booktabs}
\usepackage{xcolor} % for different colour comments
\usepackage{tabto}
\usepackage{array}
\usepackage{mdframed}
\mdfsetup{nobreak=true}
\usepackage{xkeyval}
\usepackage{tabularx}
\hypersetup{
    colorlinks,
    citecolor=black,
    filecolor=black,
    linkcolor=red,
    urlcolor=blue
}
\usepackage[skip=2pt, labelfont=bf]{caption}
\usepackage{titlesec}
\usepackage{float}
\graphicspath{ {image/} }

\titleformat{\paragraph}
{\normalfont\normalsize\textbfseries}{\theparagraph}{1em}{}
\titlespacing*{\paragraph}
{0pt}{3.25ex plus 1ex minus .2ex}{1.5ex plus .2ex}


%% Comments
\newif\ifcomments\commentstrue

\ifcomments
\newcommand{\authornote}[3]{\textcolor{#1}{[#3 ---#2]}}
\newcommand{\todo}[1]{\textcolor{red}{[TODO: #1]}}
\else
\newcommand{\authornote}[3]{}
\newcommand{\todo}[1]{}
\fi

\newcommand{\wss}[1]{\authornote{magenta}{SS}{#1}}
\newcommand{\ds}[1]{\authornote{blue}{DS}{#1}}

\begin{document}
\title{\textbf Text to Motion Database\\[\baselineskip]\Large Test Report}
\author{Brendan Duke\\Andrew Kohnen\\Udip Patel\\David Pitkanen\\Jordan Viveiros}
\date{\today}

\maketitle
\tableofcontents
% \listoftables
% \listoffigures
\newpage

\pagenumbering{arabic}


\chapter{Preface}
\section{Revision History}
\begin{table}[bp]
\caption*{\textbf Revision History}
\begin{tabularx}{\textwidth}{p{3.5cm}p{2cm}X}
\toprule {\textbf Date} & {\textbf Version} & {\textbf Notes}\\
\midrule
March 14, 2017 & 0.0 & File created\\
March 23, 2017 & 0.1 & Initial Template Completed \\
\bottomrule
\end{tabularx}
\end{table}

\chapter{Introduction}

\section{Purpose of Document}

The purpose of this manuscript is to document the testing that has been
performed on the Text-to-Motion web application. The testing that has been
performed largely follows the test plan that was provided in the Test Plan document.

\section{Scope of Testing}

Following good design procedures, the Text-To-Motion web application has been
modularized into $3$ conceptual modules.  There is a web framework that
consists of a front end and allows users to access the Text-To-Motion
features and to input new data.  There is also a back-end, which consists of two
separate components: a database module and a deep-learning module.

Following the application's modular decomposition, the tests have been
decomposed into $3$ conceptual tests.

The first stage of testing, the Proof of Concept testing, was done to ensure
that a minimal set of milestones were reached.  These milestones were set to
define a minimal viable product, namely the proof of concept.  Here we test to
see that all of the software modules mentioned above have been created and
these modules can work together.  So, this range of tests involves system and
unit testing.

The second set of tests is Solution Constraints testing.  In this set of
tests, the performance of the deep learning algorithm is rigorously tested
and quantified, and different formats of data that can be inputted are also
tested.  We also test how easy the project is to build from in our recommended
environment.

In the final set of tests, we test the functional and non-functional requirements
of the finished product. Examples of these requirements include: usability,
supported video encodings, and maintainability of the web site.

The final section in this document includes a traceability table that helps to
organize and explain how tests are connected to the requirements that were
created in the software requirements document we created earlier.

%I have added a few examples of what the tables should be.
\chapter{Preliminary Testing}

\section{Proof of Concept}
%An example of non functional requirements. In test cases where we we want an average we add an additional cell in bold
\subsection{Functional website for pose estimation}

\subsubsection{Description}

This is a baseline test to make sure the website is running, appears in a web
browser when its URL is entered, and that pictures can be uploaded and
viewed on the site.

Put the proper URL into the browser and click on each of the links to make sure
that no error occurs, meaning that there are no excessive wait times and that
the browser does not crash. Make sure there are human poses shown on the site.

\subsubsection{Results}
%Any comments we wish to make or add can be added here should we wish to expand on the results. An overall summary can also be here

All of the tests passed. The website was accessible through the browser, and
all hyperlinks were functional. Poses were displayed on the ImagePoseDraw page.

\begin{table}[H]
        \centering
        \begin{tabular}{||p{5cm}|p{2.5cm}||}
                \hline
                \textbf Test & \textbf Result\\
                \hline\hline
                TextToMotion link & Pass  \\
                \hline
                Register Link & Pass  \\
                \hline
                Sign up Link & Pass  \\
                \hline
                Poses shown when TextToMotion is clicked & Pass  \\
                \hline
                Search Link clicked & Pass  \\
                \hline
                About Link clicked & Pass  \\
                \hline
                ImagePoseDraw link clicked & Pass  \\
                \hline
        \end{tabular}
\end{table}

\subsection{Updating the database}
\subsubsection{Description}

The database allows users to upload images to the database through the website
interface. A test is considered a pass if an image or video can be uploaded,
then accessed by user through the web interface.

\subsubsection{Input Data}

\begin{table}[H]
        \centering
        \begin{tabular}{p{3cm}p{6cm}}
                \hline\hline
                Input & Description\\
                % NEED THE BUILD COMMANDS WITH DESCRIPTIONS
                \hline\hline
                TomBrady.jpg &  A picture of a man standing still and facing the camera\\ %fill in these entries once we have them
                \hline
                AverageGirl.jpg &  A picture of a woman standing still and facing the camera\\ %fill in these entries once we have them
                \hline
                AverageGuy.jpg &  A picture of football star Tom Brady\\ %fill in these entries once we have them
                \hline
        \end{tabular}
\end{table}

\subsubsection{Results}

It was possible to upload all of the test images to the database through the
website interface.

\begin{table}[H]
        \centering
        \begin{tabular}{||p{2.5cm}|p{2.5cm}||}
                \hline
                \textbf Test & \textbf Result\\
                \hline\hline
                Tom Brady & Pass  \\
                \hline\hline
                Average Girl & Pass  \\
                \hline\hline
                Average Guy & Pass  \\
                \hline
        \end{tabular}
\end{table}

\subsection{Running Pose Estimation}
\subsubsection{Description}

Once a user uploads an image they should be able to run pose estimation on an
uploaded image. A test is considered a pass if an image or video can be
uploaded, have pose estimation run on it, and have the result be visible to the
user.

\subsubsection{Results}

It was possible to run pose estimation on the uploaded images, by submitting
the images in a web form either through URL or from file.

The resultant poses showed up under the \verb|ImagePoseDraw/Details/N| page for
each upload.

\begin{table}[H]
        \centering
        \begin{tabular}{||p{2.5cm}|p{2.5cm}||}
                \hline
                \textbf Test & \textbf Result\\
                \hline\hline
                Tom Brady & Pass  \\
                \hline\hline
                Average Girl & Pass  \\
                \hline\hline
                Average Guy & Pass  \\
                \hline
        \end{tabular}
\end{table}

\subsection{Search by Tag or Name}
\subsubsection{Description}

The database has the ability to be queried using a video tag or the video name.
A test is considered a pass if we can search the database using a tag or name
and the correct video appears.

\subsubsection{Results}

\begin{table}[H]
        \centering
        \begin{tabular}{||p{2.5cm}|p{2.5cm}||}
                \hline
                \textbf Test & \textbf Result\\
                \hline\hline
                Tom Brady & Pass  \\
                \hline\hline
                Average Girl & Pass  \\
                \hline\hline
                Average Guy & Pass  \\
                \hline
        \end{tabular}
\end{table}

All of these tests passed. It was possible to search through the database and
retrieve entries by tag and video name.
%test for functional requirements.

\section{Solution Constraints Testing}

\subsection{Deep Learning Methods Test}
\subsubsection{Description}

Our supervisor Dr. Taylor will check if the deep learning methods used are
modern and up-to-date.

\subsubsection{Results}

Our supervisor confirmed that the deep learning methods used achieve near
state-of-the-art results using modern methods, based on the paper of Bulat et
al.

\begin{table}[H]
        \centering
        \begin{tabular}{||p{2.5cm}|p{2.5cm}||}
                \hline
                \textbf Test & \textbf Result\\
                \hline\hline
                Dr. Taylor &  Pass\\ %fill in these entries once we have them
                \hline
        \end{tabular}
\end{table}

\subsection{Standard Data Format Test}

\subsubsection{Description}

An automated test that checks if the human pose data used for the project is
standard and compatible with existing software libraries.

\subsubsection{Results}

This test was technically a failure. The data was not converted to a format
compatible with libraries. Instead the data is stored in JSON strings, which is
a standard format, but not compressed as an ideal format such as HDF5 would be.

\chapter{Functional Requirements}

\section{Supported Video Encodings Test}

\subsection{Description}

Tests whether the ReadFrames API is able to decode MP4 files. If we are able to
run pose estimation on the video, then the ReadFrames API is able to process the
frames.

\subsection{Input Data}

\begin{table}[H]
        \centering
        \begin{tabular}{p{3cm}p{6cm}}
                \hline\hline
                Input & Description\\
                \hline\hline
                E9FY2.MP4  &  A short video of a woman eating a sandwich\\
                \hline
                U4XV9.MP4  &  A short video of a man waking up and getting out of bed\\
                \hline
                Z1A0Q.MP4 & A short video of a man sitting on a stool\\
                \hline
        \end{tabular}
\end{table}

\subsubsection{Results}

\begin{table}[H]
        \centering
        \begin{tabular}{||p{2.5cm}|p{2.5cm}||}
                \hline
                \textbf Test & \textbf Result\\
                \hline\hline
                \verb|E9FY2.MP4|  &  Pass\\
                \hline
                \verb|U4XV9.MP4|  &  Pass\\
                \hline
                \verb|Z1A0Q.MP4| & Pass\\
                \hline
        \end{tabular}
\end{table}

\section{Frame Reading Timestamp Accuracy Test}
\subsection{Description}

Tests whether the timestamps on the frames returned by the ReadFrames API match
their temporal position in the original video stream. Our input data is
identical to the previous test.

\subsection{Results}

\begin{table}[H]
        \centering
        \begin{tabular}{||p{2.5cm}|p{2.5cm}||}
                \hline
                \textbf Test & \textbf Result\\
                \hline\hline
                \verb|E9FY2.MP4|  &  Pass\\ %fill in these entries once we have them
                \hline
                \verb|U4XV9.MP4|  &  Pass\\
                \hline
                \verb|Z1A0Q.MP4| & Pass\\
                \hline
        \end{tabular}
\end{table}

\section{Human Pose Estimation Data Quality Test}
\subsection{Description}

Test to ensure the data quality produced by the human pose estimator component
was acceptable.

A set of Charades videos will be processed by the human pose estimator, and
skeleton animations corresponding to the generated human pose data will be
created (this is a scoped part of the software pipeline). A double-blind test
will be ran, wherein testers will be shown random mixed sets of the skeleton
animations produced by McMaster Text to Motion, together with skeletons from
actual motion capture data coming from CMU'’s motion capture lab. Testers will
indicate whether they think the motion capture data came from actual motion
capture, or from the pose estimation software.

The McMaster Text to motion Results should be guessed as accurate at similar
rates to the Charades tests.

\subsection{Results}

We provide the ratio of Text-to-Motion images guessed as accurate compared to
the Charades Images.

\begin{table}[H]
        \centering
        \begin{tabular}{||p{2.5cm}|p{5.0cm}|p{2.5cm}||}
                \hline
                \textbf Test & \textbf Charades/TextToMotion & \textbf Result\\
                \hline\hline
                Nick &  1/1 & Pass\\
                \hline
                Maddy &  7/5  & Pass\\
                \hline
                Sarah & 5/4 & Pass\\
                \hline
        \end{tabular}
\end{table}

\section{Database Output Full Range Coverage Test}
\subsection{Description}

Tests to be sure all entries in the database can be successfully searched for.
The videos provided from earlier tests are put into the database, and have been
renamed for testing purposes.

\subsection{Input Data}

\begin{table}[H]
        \centering
        \begin{tabular}{p{3cm}p{6cm}}
                \hline\hline
                Input & Description\\
                % NEED THE BUILD COMMANDS WITH DESCRIPTIONS
                \hline\hline
                \verb|Waking_Up.mp4| &  Given tags sleeping, boy, man, sleepy, getting up, table\\
                \hline
                \verb|Eating.mp4| &  Given tags sandwich, eating, girl, woman, table\\
                \hline
                \verb|Stool.mp4| &  Given tags man, stool, corner, sitting\\
                \hline
        \end{tabular}
\end{table}

\subsubsection{Results}

The given set of videos appeared in the returned list of videos from a database
search.

\begin{table}[H]
        \centering
        \begin{tabular}{||p{2.5cm}|p{2.5cm}||}
                \hline
                \textbf Test & \textbf Result\\
                \hline\hline
                \verb|Waking_Up.mp4| &  Pass \\
                \hline
                \verb|Eating.mp4| &  Pass\\
                \hline
                \verb|Stool.mp4| &  Pass\\
                \hline
        \end{tabular}
\end{table}

\section{Database No False Positives}
\subsection{Description}

Tests that the database search does not return any false positives, such as
videos or images that do not contain searched words. The same videos from the
previous test will be used with the same tags. Thus, we will search with tags
other than those provided. If no videos appear, then the test is a success.

\subsection{Input Data}

\begin{table}[H]
        \centering
        \begin{tabular}{p{3cm}p{6cm}}
                \hline\hline
                Input & Description\\
                % NEED THE BUILD COMMANDS WITH DESCRIPTIONS
                \hline\hline
                \verb|Waking_Up.mp4| &  Given tags luggage, chair, end, loading\\
                \hline
                \verb|Eating.mp4| &  Given tags recollection, breaking, band, insult\\
                \hline
                \verb|Stool.mp4| &  Given tags tackling, interview, virus, sunk\\
                \hline
        \end{tabular}
\end{table}

\subsubsection{Results}

Using the nonsensical keywords from our input data, no search results were
returned, meaning that the test passed.

\begin{table}[H]
        \centering
        \begin{tabular}{||p{2.5cm}|p{2.5cm}||}
                \hline
                \textbf Test & \textbf Result\\
                \hline\hline
                \verb|Waking_Up.mp4| &  Pass \\
                \hline
                \verb|Eating.mp4| &  Pass\\
                \hline
                \verb|Stool.mp4| &  Pass\\
                \hline
        \end{tabular}
\end{table}

\section{Full Text Search Order by Relevance Test}

\subsection{Description}

Using the data from the previous test, we will conduct a search with multiple tags
and the videos output by the search should should be ordered from the most
relevant video to the least relevant.

\subsection{Input Data}

\begin{table}[H]
        \centering
        \begin{tabular}{p{3cm}p{6cm}}
                \hline\hline
                Input & Description\\
                % NEED THE BUILD COMMANDS WITH DESCRIPTIONS
                \hline\hline
                man, stool &  should return \verb|Stool.mp4| followed by \verb|Waking_Up.mp4| \\ %fill in these entries once we have them
                \hline
                man, bed &  should return \verb|Waking_Up.mp4| followed by \verb|Stool.mp4| \\ %fill in these entries once we have them
                \hline
                table, girl &  should return \verb|Eating.mp4| followed by \verb|Waking_Up.mp4| \\ %fill in these entries once we have them
                \hline
        \end{tabular}
\end{table}

\subsubsection{Results}

The search results returned the correct relative ordering for the videos.

\begin{table}[H]
        \centering
        \begin{tabular}{||p{2.5cm}|p{2.5cm}||}
                \hline
                \textbf Test & \textbf Result\\
                \hline\hline
                man, stool &  Pass \\
                \hline
                man, bed &  Pass\\
                \hline
                table, girl &  Pass\\
                \hline
        \end{tabular}
\end{table}

%For tests that require a time use this
\chapter{Non-Functional Requirements}
\section{Usability}

In order to determine the usability of the Text-to-Motion database, a small
sample of users were asked to use the website to perform some predetermined
actions and answer questions afterwards.

Before the participants were asked to preform any actions they were given a
minute to familiarize themselves with the interface, but were not given any
guidance or tips from the development team.

Once the time was up they were asked to upload an image on mobile or desktop
through their webcam, a URL or from a file saved within the computer.

While performing the required action the participant's time was recorded and
used to determine if a requirement had passed or failed.

Upon completion of the task the users were asked to rate the style and design
of the website on a scale from 1 to 10.

\section{Results}

The results from the participants can be seen throughout the Non-Functional
Requirements along with a pass or fail based on the requirements description.

\section{Look and Feel Requirements}
\subsection{colour Scheme}
\subsubsection{Description}

A test to see if the colour scheme of the website is visually appealing. The
participants were asked to rate the websites colour scheme on a scale from
1-10, and any result above a 6 will be considered a pass.

\subsubsection{Results}

The users polled rated the visual appeal of the colour scheme high enough for
this test to pass.

\begin{table}[H]
        \centering
        \begin{tabular}{||p{2.5cm}|p{2.5cm}|p{2.5cm}||}
                \hline
                \textbf User & \textbf Rating & \textbf Result\\
                \hline\hline
                Nick & 7 & Pass \\
                \hline
                Maddy & 8 & Pass\\ %fill in these entries once we have them
                \hline
                Sarah & 7 & Pass \\
                \hline
        \end{tabular}
\end{table}

\section{Style Requirements}

\subsection{Minimalistic Web Design}
\subsubsection{Description}

The website interface should be minimal and should inform the user of valid
actions through visual means. Participants were asked to rate the design from
1-10, and any result above a 5 will be considered a pass.

\subsubsection{Results}

Participants rated the design of the website interface highly enough to warrant
a pass.

\begin{table}[H]
        \centering
        \begin{tabular}{||p{2.5cm}|p{2.5cm}|p{2.5cm}||}
                \hline
                \textbf User & \textbf Rating & \textbf Result\\
                \hline\hline
                Nick & 9 & Pass \\
                \hline
                Maddy & 8 & Pass\\ %fill in these entries once we have them
                \hline
                Sarah & 8 & Pass \\
                \hline
        \end{tabular}
\end{table}

\section{Ease of Use Requirements}
\subsection{Upload/Download}
\subsubsection{Description}

Through the web interface a user should be able to upload a picture using
either a mobile phone camera, URL, or saved file. The participant will start on
the Home page and be asked to upload an image through one of the methods just
mentioned. In order for this to be considered a pass it should take the users
30 seconds or less to complete the upload process and click the button.

\subsubsection{Results}

All users were able to upload images within the required time, and therefore
the tests passed.

\begin{table}[H]
        \centering
        \begin{tabular}{||p{4.5cm}|p{2.5cm}|p{2.5cm}|p{2.5cm}||}
                \hline
                \textbf Test & \textbf User & \textbf Time & \textbf Result \\
                \hline
                Uploading based on URL & Nick & 28 seconds & Pass\\
                \hline
                Uploading image from a mobile device & Maddy & 22 seconds  & Pass\\
                \hline
                Uploading file from Desktop &  Sarah & 26 seconds & Pass\\
                \hline
        \end{tabular}
\end{table}

\subsection{Text Box Functionality}
\subsubsection{Description}

The user should be able to input a descriptive word or phrase into a text-box
from within the web interface in order to search for a video. In order to
complete this task the users were asked to search for a specific word and
display the results. Any time below 10 seconds will be considered a pass.

\subsubsection{Results}

\begin{table}[H]
        \centering
        \begin{tabular}{||p{4.5cm}|p{2.5cm}|p{2.5cm}|p{2.5cm}||}
                \hline
                \textbf Test & \textbf User & \textbf Time & \textbf Result \\
                \hline\hline
                Search for ``Woman'' & Nick & 4 seconds  & Pass\\ %fill in these entries once we have them
                \hline
                Search for ``f'' & Maddy & 3 seconds  & Pass\\
                \hline
                Search for ``the'' & Sarah & 4 seconds  & Pass\\
                \hline
        \end{tabular}
\end{table}

\section{Learning Requirements}

\subsection{Usability Tests}
\subsubsection{Description}

The user should be able to interact with the website without prior knowledge.
They will given a minute to explore the website. After that time the
participants were asked to rate the usability on a scale of 1-10. An average of
6 is required for a pass.

\subsubsection{Results}

The users' usability ratings allowed the site to pass this test.

\begin{table}[H]
        \centering
        \begin{tabular}{||p{2.5cm}|p{2.5cm}|p{2.5cm}||}
                \hline
                \textbf User & \textbf Rank & \textbf Result\\
                \hline\hline
                Nick & 6 & Pass \\
                \hline
                Maddy & 8 & Pass \\
                \hline
                Sarah & 8 & Pass\\
                \hline
        \end{tabular}
\end{table}

\section{Politeness and Understandability Requirements}
\subsection{Hiding the Inner Workings}
\subsubsection{Description}

Users should not be able to see the deep learning model and its training when
using the pose estimation. When prompted the website should display the correct
skeletons without any low-level detail. Once uploaded the participants were
asked if they saw anything that seemed out of place or any information on the
deep learning process, if they did not it will be considered a pass.

\subsubsection{Results}

Users indicated that the deep learning model was encapsulated from their view,
and hence this test passed.

\begin{table}[H]
        \centering
        \begin{tabular}{||p{2.5cm}|p{2.5cm}|p{2.5cm}||}
                \hline
                \textbf Test & \textbf User & \textbf Result\\
                \hline\hline
                Uploading an image from URL & Nick & Pass\\
                \hline\hline
                Uploading an image from mobile & Maddy & Pass\\
                \hline\hline
                Uploading an image from desktop & Sarah & Pass\\
                \hline
        \end{tabular}
\end{table}

\section{Speed and Latency Testing}

\subsection{External Database Connection Response Time}
\subsubsection{Description}

The web interface should be able to connect to an external database and store
or query items. In order for this test to be considered a pass the confirmation
of the image being uploaded would have to occur within 30 seconds so that
additional resources are not wasted by the database. Testing this will occur by
uploading an image and testing the total time taken.

\subsubsection{Input Data}

\begin{table}[H]
        \centering
        \begin{tabular}{p{3cm}p{6cm}}
                \hline\hline
                Input & Description\\
                % NEED THE BUILD COMMANDS WITH DESCRIPTIONS
                \hline\hline
                Image from a URL & An image of a male  \\
                \hline\hline
                Image that was saved within the desktop & An image of Seth Rogan \\
                \hline
        \end{tabular}
\end{table}

\subsubsection{Results}

According to the response times of this automated test, the database queries
were executed fast enough for the test to pass.

\begin{table}
        \centering
        \begin{tabular}{||p{1.5cm}|p{1.5cm}|p{1.5cm}||}
                \hline
                \textbf Test & \textbf Time & \textbf Result \\
                \hline\hline
                Uploading the image from a URL & 28 seconds  & Pass\\
                \hline\hline
                Uploading an image from desktop & 29 seconds & Pass\\
                \hline
        \end{tabular}
\end{table}
\vspace{1cm}

\subsection{Website Search Responsiveness}
\subsubsection{Description}

When given a word or phrase the web interface will be able to respond with an
image or video of a pose or action within a two minutes.

\subsection{Input Data}

\begin{table}[H]
        \centering
        \begin{tabular}{p{3cm}p{6cm}}
                \hline\hline
                Input & Description\\
                \hline\hline
                Creepy & Searched using a tag within the description\\ %fill in these entries once we have them
                \hline
                Seth Rogan & Searched using a tag within the description\\
                \hline
        \end{tabular}
\end{table}

\subsubsection{Results}

\begin{table}[H]
        \centering
        \begin{tabular}{||p{1.5cm}|p{1.5cm}|p{1.5cm}||}
                \hline
                \textbf Test & \textbf Time & \textbf Result \\
                \hline\hline
                Search for ``Seth'' & 1 second & Pass\\ %fill in these entries once we have them
                \hline\hline
                Search for ``Creepy'' & 2 seconds & Pass\\
                \hline
        \end{tabular}
\end{table}


\chapter{Other Relevant Testing}

Again we test this on the same files we have used in the previous tests:
\verb|TomBrady.jpg|, \verb|AverageGirl.jpg| and \verb|AverageGuy.jpg|.

\section{Precision and Accuracy}
\subsection{Bone and Joint Position}
\subsubsection{Description}

The pose estimation should accurately predict the placement of joints and bones
of the person in the provided photo. This will be determined with visual means
with an uncertainty range of 20 pixels.

\subsubsection{Input Data}

\begin{table}[H]
        \centering
        \begin{tabular}{p{3cm}p{6cm}}
                \hline\hline
                Input & Description\\
                % NEED THE BUILD COMMANDS WITH DESCRIPTIONS
                \hline\hline
                \verb|TomBrady.jpg| &  A picture of a man standing still and facing the camera\\
                \hline
                \verb|AverageGirl.jpg| &  A picture of a woman standing still and facing the camera\\
                \hline
                \verb|TomBrady.jpg| &  A picture of football star Tom Brady\\
                \hline
        \end{tabular}
\end{table}

\subsubsection{Results}

We were able to qualitatively confirm that these tests passed for the given
input images. A more rigorous ``PCKh'' metric is used to formally evaluate the
performance of our deep learning model on single-person pose estimation.

\begin{table}[H]
        \centering
        \begin{tabular}{||p{2.5cm}|p{2.5cm}||}
                \hline
                \textbf Test & \textbf Result\\
                \hline\hline
                Tom Brady & Pass  \\
                \hline\hline
                Average Girl & Pass  \\
                \hline\hline
                Average Guy & Pass  \\
                \hline
        \end{tabular}
\end{table}

\section{Reliability and Availability Requirements}
\subsection{Software Availability}
\subsubsection{Description}

The software component of the project should be available at all times. If we
can have an event regularly occur then it will be considered a pass. To do this
we arranged to have the pose estimation algorithm automatically called on to
process a single image at 4 hour intervals, and record the time.

\subsubsection{Results}

The software successfully processed the image at the specified 4 hour intervals
over a period of 2 days.

\subsection{Software Availability}
\subsubsection{Description}

The software component of the project should be available at all times with the
exception of maintenance and migration. When we make a web server call we should
receive an HTTP verified response. To do this we will have three users sending
a HTTP POST request to the server. If they receive a response the test is
considered a pass.

\subsubsection{Results}

All three users were able to receive responses to their HTTP POST requests,
therefore we consider this test a pass.

\section{Robustness or Fault-Tolerance Requirements}
\subsection{Web Interface Error Handling}
\subsubsection{Description}

The web interface should respond to unhandled exceptions by throwing the
corresponding error messages. If an exception is thrown and an error message is
displayed then the test is considered a pass.

\subsubsection{Input Data}

\begin{table}[H]
        \centering
        \begin{tabular}{p{3cm}p{6cm}}
                \hline\hline
                Input & Description\\
                % NEED THE BUILD COMMANDS WITH DESCRIPTIONS
                \hline\hline
                Scenario 1 &  Try uploading an image while the image storage service (Amazon S3) is not available. \\
                \hline
                Scenario 2 & Trying to search an image on the database while the database server is down.
        \end{tabular}
\end{table}

\subsubsection{Results}

For Scenarios 1 and 2, appropriate exceptions were thrown and error pages
displayed to the user. Therefore this test is considered to have passed.

\section{Capacity Requirements}

\subsection{Multiple Connections}
\subsubsection{Description}

The web interface should be able to serve multiple connections. If the
interface can support 5 connections at once it is considered a pass.

\subsubsection{Input Data}
\begin{table}[H]
        \centering
        \begin{tabular}{p{3cm}p{6cm}}
                \hline\hline
                Input & Description\\
                % NEED THE BUILD COMMANDS WITH DESCRIPTIONS
                \hline\hline
                Andrew & The connection to the website corresponding to tester Andrew \\
                \hline
                Brendan & The connection to the website corresponding to tester Brendan \\
                \hline
                Udip & The connection to the website corresponding to tester Udip \\
                \hline
                Jordan & The connection to the website corresponding to tester Jordan \\
                \hline
                Dave & The connection to the website corresponding to tester Dave \\
                \hline
        \end{tabular}
\end{table}

\subsubsection{Results}

The 5 users were able to successfully connect to the website, and run queries
simultaneously without experiencing significant slow down or waiting.

\subsection{Database Capacity}
\subsubsection{Description}

The database should contain at least 5GB of data in order to facilitate growth.

\subsubsection{Results}

The database successfully handled us uploading a total of 5.1GB of data in the
form of pictures of various qualities. These pictures were of randomly acquired
photos from Google, which were combined to form the data used in this testing.

\section{Scaling of Extensibility Requirements}

\section{Operational Environment Requirements}

\subsection{Linux Friendly TensorFlow}
\subsubsection{Description}

The web interface should be run on a Linux friendly server that can access the
TensorFlow model either directly or indirectly. By creating an interface that
successfully runs on a Apache or nginx server this test will be considered a
pass.

\subsubsection{Results}

Our service is successfully running on a production Linux server using nginx.

\subsection{Export Types}
\subsubsection{Description}

The project should be able to export multiple types of media (JPEG, PNG, etc)
in order to support all major operating systems. We will use \verb|TomBrady.jpg| for
this test.

\subsubsection{Results}

This test was a guaranteed pass due to the fact that our website stores images
as a base64, meaning it can be converted into any type.

\begin{table}[H]
        \centering
        \begin{tabular}{||p{2.5cm}|p{2.5cm}||}
                \hline
                \textbf Test & \textbf Result\\
                \hline\hline
                PNG & Pass \\ %fill in these entries once we have them
                \hline
                JPEG & Pass \\ %fill in these entries once we have them
                \hline
                BMP & Pass \\ %fill in these entries once we have them
                \hline
        \end{tabular}
\end{table}

\chapter{Traceability} % FILL THIS IF WE HAVE TIME
\begin{center}
\begin{longtable}{>{\raggedright\arraybackslash}p{0.1\textwidth}>{\raggedright\arraybackslash}p{0.3\textwidth}>{\raggedright\arraybackslash}p{0.5\textwidth}}
\caption{Traceability Matrix for Test-Requirement Relationships}\label{Table_TestsAndRequirements}
\\\toprule
\textbf Test \#  & \textbf Description & \textbf Requirement\\\midrule
\# 3.2,  2.1.4,  2.2.1
& Tests which measured performance Accuracy of Deep Learning Algorithm
&  Req \# 1, 8,  23(Speed and Latency)\\
\# 2.1.1,  2.1.2, 2.1.3, 2.2.3 & System tests, measuring reliability of the web framework. &
Req \# 7,  12,  17, 20,  28,  30, 38, 39
\\
\# 2.1.5
 & Unit and Systems Tests Grading Database Search Performance & Req \# 9, 10, 11,
 \\
2.1.3, 2.2.2,  & Security and Data Integrity Tests & Req \# 10, 27, 29 \\
2.2.2, 2.3.1, 2.3.2 & Proper Formatting Tests & Req \# 2, 7, 23, 29
\\
\# 2.2.1, 2.4.1 & User Interface Tests & Req \# 13-21, 24-26
\\
\bottomrule
\end{longtable}
\end{center}
\section{Modules}
Similarly, the following is a traceability table explicitly relating test cases to modules:\\
\begin{table}
\caption{Tests and Modules Relationships}
\begin{center}
\begin{tabular}{p{6cm} p{4cm}}
\hline
\textbf Test \#  & Module \\
\hline
3.1(all subsections), 3.2.1, 4.4-4.6, 5(all sections), 6.2(all subsections) - 6.4(all subsections), 6.6.2 & ASP.NET and DB \\
\hline
3.1.3, 4.3, 5.8.2, 6.1, 6.5 & TensorFlow Models \\
\hline
3.13, 3.2.2, 4.1-4.2, 5.8.2, 6.2(all subsections), 6.6.1 & Python HTTP Server\\
\hline
\end{tabular}
\end{center}
\end{table}

\chapter{Changes After testing}

First of our major changes would likely be to the website interface. Though our
testers reviewed it favourably, there were numerous references, but we were also
given consistent criticism that an alternate colour scheme might be a slight
improvement. As such we have agreed to experiment with those improvements.

Testing also revealed that a JavaScript application to allow continuous requests
to the HTTP server as well as utilize a mobile platform would have greater
applicability to the Guelph team. This improvement would come alongside a
function on the website to display statistics. This would help insure greater
availability. The ability to track the database live would be both useful to
the testing team as well as an entertaining aspect for the product demo.

The last major improvement we wish to provide is an improved JavaScript plugin
for drawing the skeleton on top of an image. Testing revealed that there was
flicker with the drawn skeletons, and as such, could lead to false negatives,
where a skeleton that appears inaccurate is actually just limited by our
current skeleton drawing methods.

\end{document}
