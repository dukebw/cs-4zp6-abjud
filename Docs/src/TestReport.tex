\documentclass{scrreprt}
\usepackage{listings}
\usepackage{underscore}
\usepackage{ragged2e}
\usepackage[bookmarks=true]{hyperref}
\usepackage[utf8]{inputenc}
\usepackage[english]{babel}
\usepackage{xcolor}
\usepackage{indentfirst}
\usepackage[section]{placeins}
\usepackage{graphicx}
\usepackage{graphics}
\usepackage{longtable}
\usepackage{booktabs}
\usepackage{xcolor} % for different colour comments
\usepackage{tabto}
\usepackage{array}
\usepackage{mdframed}
\mdfsetup{nobreak=true}
\usepackage{xkeyval}
\usepackage{tabularx}
\hypersetup{
    colorlinks,
    citecolor=black,
    filecolor=black,
    linkcolor=red,
    urlcolor=blue
}
\usepackage[skip=2pt, labelfont=bf]{caption}
\usepackage{titlesec}
\usepackage[section]{placeins}
\graphicspath{ {image/} }

\titleformat{\paragraph}
{\normalfont\normalsize\bfseries}{\theparagraph}{1em}{}
\titlespacing*{\paragraph}
{0pt}{3.25ex plus 1ex minus .2ex}{1.5ex plus .2ex}


%% Comments
\newif\ifcomments\commentstrue

\ifcomments
\newcommand{\authornote}[3]{\textcolor{#1}{[#3 ---#2]}}
\newcommand{\todo}[1]{\textcolor{red}{[TODO: #1]}}
\else
\newcommand{\authornote}[3]{}
\newcommand{\todo}[1]{}
\fi

\newcommand{\wss}[1]{\authornote{magenta}{SS}{#1}}
\newcommand{\ds}[1]{\authornote{blue}{DS}{#1}}

\begin{document}
\title{\bf Text to Motion Database\\[\baselineskip]\Large Test Report}
\author{Brendan Duke\\Andrew Kohnen\\Udip Patel\\David Pitkanen\\Jordan Viveiros}
\date{\today}

\maketitle
\tableofcontents
% \listoftables
% \listoffigures


\begin{table}[bp]
\caption*{\bf Revision History}
\begin{tabularx}{\textwidth}{p{3.5cm}p{2cm}X}
\toprule {\bf Date} & {\bf Version} & {\bf Notes}\\
\midrule
March 14, 2017 & 0.0 & File created\\
\bottomrule
\end{tabularx}
\end{table}

\newpage

\pagenumbering{arabic}


\chapter{List of Tables}
    tables for a specific tests have been removed to improve readability

\chapter{Introduction}

\section{Purpose of Document}
    This document summarizes the testing and the findings of the tests for the Text to Motion project. This documentuses the implementation outlined in the test plan.

\section{Scope of Testing}

\section{Organization}
    Section 1 is our introduction and introduces this report. Section 2 describes the test cases as outlined in our test plan and their results. Section 4 describes traceability to the requirements. Section 4 describes what changes have been made in response to our testing.

    \textbf{Testing structure}

    Automated Testing

    \quad generating inputs

    Manual Testing

%I have added a few examples of what the tables should be.
\chapter{Preliminary Testing}

\section{Proof of Concept}
%An example of non functional requirements. In test cases where we we want an average we add an additional cell in bold
\subsection{Functional website for pose estimation}
\subsubsection{Description}
The user should be able to run pose estimation of a video or image through the website interface
\subsubsection{Results}
%Any comments we wish to make or add can be added here should we wish to expand on the results. An overall summary can also be here
 \centering
 \begin{tabular}{||p{2.5cm}|p{2.5cm}||}
 \hline
 \bf Test & \bf Result\\
 \hline\hline
   &  \\ %fill in these entries once we have them
 \hline
 \end{tabular}

\subsection{Databse Website Pairing}
\subsubsection{Description}
The database will be paired to the website so that the website provides images or video to the user. A test is consider a pass if the database can be accessed through the web interface
\subsubsection{Results}
 \centering
 \begin{tabular}{||p{2.5cm}|p{2.5cm}||}
 \hline
 \bf Test & \bf Result\\
 \hline\hline
   &  \\ %fill in these entries once we have them
 \hline
 \end{tabular}

\subsection{Updating the database}
\subsubsection{Description}
The database allows users to upload images to the database through the website interface. A test is considered a pass if an image or video can be uploaded, then accessed by user through the web interface
\subsubsection{Results}
 \centering
 \begin{tabular}{||p{2.5cm}|p{2.5cm}||}
 \hline
 \bf Test & \bf Result\\
 \hline\hline
   &  \\ %fill in these entries once we have them
 \hline
 \end{tabular}

\subsection{Running Pose Estimation}
\subsubsection{Description}
Once a user uploads an image they should be able to run pose estimation on an uploaded image. A test is considered a pass if an image or video can be uploaded, have pose estimation run on it, and have the result be visible to the user.
\subsubsection{Results}
 \centering
 \begin{tabular}{||p{2.5cm}|p{2.5cm}||}
 \hline
 \bf Test & \bf Result\\
 \hline\hline
   &  \\ %fill in these entries once we have them
 \hline
 \end{tabular}

\subsection{Search by Tag or Name}
\subsubsection{Description}
The database has the ability to search through the database using a video tag or the video name. A test is considered a pass if we can search the dataase using a tag or name and the correct video appears.
\subsubsection{Results}
 \centering
 \begin{tabular}{||p{2.5cm}|p{2.5cm}||}
 \hline
 \bf Test & \bf Result\\
 \hline\hline
   &  \\ %fill in these entries once we have them
 \hline
 \end{tabular}

%test for functional requirements.
\section{Solution Constraints Testing}

\subsection{Deep Learning Methods Test}
\subsubsection{Description}
Our Supervisor Dr. Taylor will check if the Deep Learning Methods used are modern and up-to-date.
\subsubsection{Results}
 \centering
 \begin{tabular}{||p{2.5cm}|p{2.5cm}||}
 \hline
 \bf Test & \bf Result\\
 \hline\hline
   &  \\ %fill in these entries once we have them
 \hline
 \end{tabular}

\subsection{Standard Data Format Test}
\subsubsection{Description}
An automated test that checks if the human pose data used for the project is standard and compatible with existing software libraries.
\subsubsection{Input Data}
 \centering
 \begin{tabular}{p{3cm}p{6cm}}
 \hline\hline
 Input & Description\\
 \hline\hline
   &  \\ %fill in these entries once we have them
 \hline
 \end{tabular}
\subsubsection{Results}
%Results are largely unchanged from nonfunctional requirements, though somtimes we might choose to simply describe the result without a table.

\subsection{Linux Platform Build and Run Test}
\subsubsection{Description}
an automated test that confirms that the pose estimation software can be built correctly on a Linux platform.
\subsubsection{Input Data}
 \centering
 \begin{tabular}{p{3cm}p{6cm}}
 \hline\hline
 Input & Description\\
 % NEED THE BUILD COMMANDS WITH DESCRIPTIONS
 \hline\hline
   &  \\ %fill in these entries once we have them
 \hline
 \end{tabular}
\subsubsection{Results}

\subsection{Python API Hook Testing}
\subsubsection{Description}
An automated test that confirms the major interfaces for the project have working python hooks.
\subsubsection{Input Data}
 \centering
 \begin{tabular}{p{3cm}p{6cm}}
 \hline\hline
 Input & Description\\
 % NEED THE BUILD COMMANDS WITH DESCRIPTIONS
 \hline\hline
   &  \\ %fill in these entries once we have them
 \hline
 \end{tabular}
\subsubsection{Results}

\chapter{Functional Requirements}
\section{Supported Video Encodings Test}
\subsection{Description}
Tests whether the ReadFrames API is able to decode MP4, MP2 and AAC video files.
\subsection{Input Data}
 \centering
 \begin{tabular}{p{3cm}p{6cm}}
 \hline\hline
 Input & Description\\
 % NEED THE BUILD COMMANDS WITH DESCRIPTIONS
 \hline\hline
   &  \\ %fill in these entries once we have them
 \hline
 \end{tabular}
\subsection{Results}

\section{Frame Reading Timestamp Accuracy Test}
\subsection{Description}
Tests whether the timestamps on the frames returned by the ReadFrames API match their temporal position in the original video stream.
\subsection{Input Data}
 \centering
 \begin{tabular}{p{3cm}p{6cm}}
 \hline\hline
 Input & Description\\
 % NEED THE BUILD COMMANDS WITH DESCRIPTIONS
 \hline\hline
   &  \\ %fill in these entries once we have them
 \hline
 \end{tabular}
\subsection{Results}

\section{Human Pose Estimation Data Quality Test}
\subsection{Description}
\begin{flushleft}
Test to ensure the data quality produced by the human pose estimator component. A set of Charades videos will be processed by the human pose estimator, and skeleton animations corresponding to the generated human pose data will be created (this is a scoped part of the software pipeline). A double-blind test will be ran, where testers will be shown random mixed sets of the skeleton animations produced by McMaster Text to Motion, together with skeletons from actual motion capture data coming from CMU’s motion capture lab. Testers will indicate whether they think the motion capture data came from actual motion capture, or from the pose estimation software.
\end{flushleft}
\subsection{Results}
 \centering
 \begin{tabular}{||p{2.5cm}|p{2.5cm}||}
 \hline
 \bf Test & \bf Result\\
 \hline\hline
   &  \\ %fill in these entries once we have them
 \hline
 \end{tabular}

\section{Database Output Full Range Coverage Test}
\subsection{Description}
\begin{flushleft}
Tests to be sure all entries in the database can be successfully searched for.
\end{flushleft}
\subsection{Input Data}
 \centering
 \begin{tabular}{p{3cm}p{6cm}}
 \hline\hline
 Input & Description\\
 % NEED THE BUILD COMMANDS WITH DESCRIPTIONS
 \hline\hline
   &  \\ %fill in these entries once we have them
 \hline
 \end{tabular}
\subsection{Results}

\section{Databse No False Positives}
\subsection{Description}
\begin{flushleft}
Tests that the database search does not return any false positives, such as videos or images that do not contains searched words.
\end{flushleft}
\subsection{Input Data}
 \centering
 \begin{tabular}{p{3cm}p{6cm}}
 \hline\hline
 Input & Description\\
 % NEED THE BUILD COMMANDS WITH DESCRIPTIONS
 \hline\hline
   &  \\ %fill in these entries once we have them
 \hline
 \end{tabular}
\subsection{Results}

\section{Full Text Search Order by Relevance Test}
\subsection{Description}
\begin{flushleft}
%TO FILL
\end{flushleft}
\subsection{Input Data}
 \centering
 \begin{tabular}{p{3cm}p{6cm}}
 \hline\hline
 Input & Description\\
 % NEED THE BUILD COMMANDS WITH DESCRIPTIONS
 \hline\hline
   &  \\ %fill in these entries once we have them
 \hline
 \end{tabular}
\subsection{Results}

%For tests that require a time use this
\chapter{Non-Functional Requirements}

\section{Look and Feel Requirements}
\subsection{Description}
\begin{flushleft}
A test to see if the color scheme of the website is visually appealing. To determine this we ask users to rank the color scheme on a scale of 1 out of 10. An average of 6 will be enough to satisfy this requirement.
\subsection{Results}
\end{flushleft}
 \centering
 \begin{tabular}{||p{2.5cm}|p{2.5cm}||}
 \hline
 \bf User & \bf Rank\\
 \hline\hline
 & \\
 \hline
 Average &  \\ %fill in these entries once we have them
 \hline
 Pass? & \\
 \hline
 \end{tabular}

\section{Style Requirements}
\section{Description}
\begin{flushleft}
The website interface should be minimal and should inform the user of valid actions through visual means. Users will rate the design on a scale of 1 to 10. An average of 5 is required for a pass.
\subsubsection{Results}
\end{flushleft}
 \centering
 \begin{tabular}{||p{2.5cm}|p{2.5cm}||}
 \hline
 \bf User & \bf Rank\\
 \hline\hline
 & \\
 \hline
 Average &  \\ %fill in these entries once we have them
 \hline
 Pass? & \\
 \hline
 \end{tabular}

\section{Ease of Use Requirements}
\subsection{Upload/Download}
\subsubsection{Description}
\begin{flushleft}
Through the web interface a user should be able to upload then download a picture within 30 seconds
\subsubsection{Results}
\end{flushleft}
 \centering
 \begin{tabular}{||p{1.5cm}|p{1.5cm}|p{1.5cm}||}
 \hline
 \bf Test & \bf Time & \bf Result \\
 \hline\hline
   &  & \\ %fill in these entries once we have them
 \hline
 \end{tabular}

\subsection{Text Box Functionality}
\subsubsection{Description}
\begin{flushleft}
The user should be able to input a descriptive word or phrase into a text-box from within the web interface in order to search for a video.
\subsubsection{Results}
\end{flushleft}
 \centering
 \begin{tabular}{||p{1.5cm}|p{1.5cm}|p{1.5cm}||}
 \hline
 \bf Test & \bf Time & \bf Result \\
 \hline\hline
   &  & \\ %fill in these entries once we have them
 \hline
 \end{tabular}

\section{Learning Requirements}
\subsection{Usability Tests}
\subsubsection{Description}
\begin{flushleft}
The user should be able to interact with the website without prior knowledge. They will given a minute to explore the website. After that time they are asked to rate the usability on a scale of 1 to 10. An average of 6 is required for a pass.
\subsubsection{Results}
\end{flushleft}
 \centering
 \begin{tabular}{||p{2.5cm}|p{2.5cm}||}
 \hline
 \bf User & \bf Rank\\
 \hline\hline
 & \\
 \hline
 Average &  \\ %fill in these entries once we have them
 \hline
 Pass? & \\
 \hline
 \end{tabular}

\subsection{Text to Motion Training }
\subsubsection{Description}
\begin{flushleft}
The user should be able to interact with the website without prior knowledge. They will given a minute to explore the website. After that time they are asked to rate the usability on a scale of 1 to 10. An average of 6 is required for a pass.
\subsubsection{Results}
\end{flushleft}
 \centering
 \begin{tabular}{||p{2.5cm}|p{2.5cm}||}
 \hline
 \bf User & \bf Rank\\
 \hline\hline
 & \\
 \hline
 Average &  \\ %fill in these entries once we have them
 \hline
 Pass? & \\
 \hline
 \end{tabular}

\chapter{Other Relavent Testing}
\section{Presicion and Accuracy}
\section{Robustness/Fault Tolerance}
\section{Capacity Requirements}
\section{}
\chapter{Changes After testing}

\end{document}
