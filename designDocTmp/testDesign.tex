\documentclass{scrreprt}

\usepackage{xcolor} % for different colour comments
\usepackage{tabto}
\usepackage{mdframed}
\mdfsetup{nobreak=true}
\usepackage{xkeyval}
\usepackage{tabularx}
\usepackage{booktabs}
\usepackage{hyperref}
\hypersetup{
    colorlinks,
    citecolor=black,
    filecolor=black,
    linkcolor=red,
    urlcolor=blue
}
\usepackage[skip=2pt, labelfont=bf]{caption}
\usepackage{titlesec}
\usepackage{graphicx}
\usepackage[section]{placeins}
\graphicspath{ {image/} }

\titleformat{\paragraph}
{\normalfont\normalsize\bfseries}{\theparagraph}{1em}{}
\titlespacing*{\paragraph}
{0pt}{3.25ex plus 1ex minus .2ex}{1.5ex plus .2ex}


%% Comments
\newif\ifcomments\commentstrue

\ifcomments
\newcommand{\authornote}[3]{\textcolor{#1}{[#3 ---#2]}}
\newcommand{\todo}[1]{\textcolor{red}{[TODO: #1]}}
\else
\newcommand{\authornote}[3]{}
\newcommand{\todo}[1]{}
\fi

\newcommand{\wss}[1]{\authornote{magenta}{SS}{#1}}
\newcommand{\ds}[1]{\authornote{blue}{DS}{#1}}


%% The following are used for pretty printing of events and requirements
\makeatletter

\define@cmdkey      [TP] {test}     {name}       {}
\define@cmdkey      [TP] {test}     {desc}       {}
\define@cmdkey      [TP] {test}     {type}       {}
\define@cmdkey      [TP] {test}     {init}       {}
\define@cmdkey      [TP] {test}     {input}      {}
\define@cmdkey      [TP] {test}     {output}     {}
\define@cmdkey      [TP] {test}     {pass}       {}
\define@cmdkey      [TP] {test}     {user}       {}
\define@cmdkey      [TP] {test}     {reqnum}     {}


\newcommand{\getCurrentSectionNumber}{%
  \ifnum\c@section=0 %
  \thechapter
  \else
  \ifnum\c@subsection=0 %
  \thesection
  \else
  \ifnum\c@subsubsection=0 %
  \thesubsection
  \else
  \thesubsubsection
  \fi
  \fi
  \fi
}

\newcounter{TestNum}

\@addtoreset{TestNum}{section}
\@addtoreset{TestNum}{subsection}
\@addtoreset{TestNum}{subsubsection}

\newcommand{\testauto}[1]{
\setkeys[TP]{test}{#1}
\refstepcounter{TestNum}
\begin{mdframed}[linewidth=1pt]
\begin{tabularx}{\textwidth}{@{}p{3cm}X@{}}
{\bf Test \getCurrentSectionNumber.\theTestNum:} & {\bf \cmdTP@test@name}\\[\baselineskip]
{\bf Description:} & \cmdTP@test@desc\\[0.5\baselineskip]
{\bf Type:} & \cmdTP@test@type\\[0.5\baselineskip]
{\bf Initial State:} & \cmdTP@test@init\\[0.5\baselineskip]
{\bf Input:} & \cmdTP@test@input\\[0.5\baselineskip]
{\bf Output:} & \cmdTP@test@output\\[0.5\baselineskip]
{\bf Pass:} & \cmdTP@test@pass\\[0.5\baselineskip]
{\bf Req. \#:} & \cmdTP@test@reqnum
\end{tabularx}
\end{mdframed}
}

\newcommand{\testmanual}[1]{
\setkeys[TP]{test}{#1}
\refstepcounter{TestNum}
\begin{mdframed}[linewidth=1pt]
\begin{tabularx}{\textwidth}{@{}p{3cm}X@{}}
{\bf Test \getCurrentSectionNumber.\theTestNum:} & {\bf \cmdTP@test@name}\\[\baselineskip]
{\bf Description:} & \cmdTP@test@desc\\[0.5\baselineskip]
{\bf Type:} & \cmdTP@test@type\\[0.5\baselineskip]
{\bf Testers:} & \cmdTP@test@user\\[0.5\baselineskip]
{\bf Pass:} & \cmdTP@test@pass\\[0.5\baselineskip]
{\bf Req. \#:} & \cmdTP@test@reqnum
\end{tabularx}
\end{mdframed}
}

\makeatother

\newcommand{\ZtoT}{
\begin{tabularx}{3.85cm}{@{}p{0.35cm}p{0.35cm}p{0.35cm}p{0.35cm}p{0.35cm}p{0.35cm}p{0.35cm}p{0.35cm}p{0.35cm}p{0.35cm}p{0.35cm}@{}}
0 & 1 & 2 & 3 & 4 & 5 & 6 & 7 & 8 & 9 & 10
\end{tabularx}
}

\begin{document}
\title{\bf Text to Motion Database\\[\baselineskip]\Large Design Document}
\author{Brendan Duke\\Andrew Kohnen\\Udip Patel\\David Pitkanen\\Jordan Viveiros}
\date{\today}

\maketitle

\pagenumbering{roman}
\tableofcontents
% \listoftables
% \listoffigures


\begin{table}[bp]
\caption*{\bf Revision History}
\begin{tabularx}{\textwidth}{p{3.5cm}p{2cm}X}
\toprule {\bf Date} & {\bf Version} & {\bf Notes}\\
\midrule
January 5, 2017 & 0.0 & File created\\
\bottomrule
\end{tabularx}
\end{table}

\newpage

\pagenumbering{arabic}

\chapter{User Experience}

\section{User Journey}

\section{Home Page}

\section{About}

\section{Contact}

\section{Sign Up}

\section{Login}

\section{Navigation Bar Links}

\section{Text to Motion}

\subsection{Search}

\subsection{Search Results}

\section{Image Pose Draw}

\subsection{Create}

\subsection{Edit/Details}

\subsection{View Uploads}





\chapter{Database Structure}

\section{Database Schema}
\section{Table Description}


\chapter{Module Decomposition}

\section{Text To Motion - ASP.NET Application}


\subsection{Overview}

...might be a good place to mention .net MVC. Mention of how sections will be about Models, Controllers (referencing views)...

\subsection{Dependency Injection}

...describe anything important that is in project.json...

\subsection{Models}

    \begin{itemize}
        \item \textbf User

        ...

        \item \textbf ...
    \end{itemize}


\subsection{HomeController}

    \begin{itemize}
        \item \textbf FunctionName:

        ...description of func...

        ...reference any models used...

        ...reference the view that is returned...
    \end{itemize}

\subsection{AccountController}

...

\subsection{ManageController}

...

\subsection{ImagePoseDrawController}

...

\subsection{TextToMotionController}

...

\break

\section{Flowing Convnets - Human Pose Estimation}

\subsection{Overview}

This component of the project actually renders the skeleton overlay onto an image submitted to the website.
\\
Mapping out the joints of a person in an image requires the use of image manipulation and deep learning libraries. As of now, this proces is based on a research paper and is implemented with \textbf{Caffe} and \textbf{OpenCV} in \textbf{C++}. (*The parameters for functions given below will reference 'caffe' and 'cv' types in c++)



\subsection{Shared Object File}

The website is able to take an uploaded image and process it by using a shared object file (.so). The web app can make function calls to functions in the shared object file and pass in images as the parameters.
\\\\
The C++ file \textbf{"estimate\_pose.cpp"} contains all of the functions that interface with Caffe and OpenCV. The C file \textbf{"estimate\_pose\_wrapper.c"} is used to wrap the C++ function in C, and create the shared object file so that the C++ function can be accessed from the website.

\subsection{Pose Estimation C Program (estimate\_pose\_wrapper.c)}
The file \textbf{"estimate\_pose\_wrapper.c"} just contains 1 function. This function references a C++ function in \textbf{estimate\_pose.cpp}
\\\\
\textbf{function: int32\_t estimate\_pose\_wrapper(args)}
\begin{itemize}
    \item \textbf{Expected Arguments:}

    void\quad\textit{*image}
    \\
    uint32\_t\quad\textit{*size\_bytes}
    \\
    uint32\_t\quad\textit{max\_size\_Bytes}

    \item\quad\textbf{Returns:}

    if \textit{image} is processed and saved, returns:\\\textbf{int32\_t size}
    (describing size of file that was uploaded)
    \\\\
    else returns error object

    \item \textbf{Description:}

    this function makes a direct call to the C++ function \textbf{"estimate\_pose\_from\_c"} in \textbf{"estimate\_pose.cpp"} and simply returns the result of that C++ function call.
    \\
    The C++ function takes in the same args as this function
\end{itemize}

\subsection{Pose Estimation C++ Program (estimate\_pose.cpp)}

This C++ Program file contains 8 functions. All of these functions take in objects from the 'openCV'(cv) and 'Caffe'(caffe) libraries as arguments
\\\\
The key function in this file is \textbf{estimate\_pose\_from\_c}, and most of the functions serve as helpers to this function
\\\\
\textbf{List of Functions and Descriptions (PE = Pose Estimation):}
\\
\subsubsection{PE function 1: void channels\_from\_blob(args)}
\begin{itemize}
    \item \textbf{Expected Arguments:}

    std::vector\textless cv::Mat\textgreater\quad\textit{channels}
    \\
    boost::shared\_ptr\textless caffe::Blob\textgreater\quad\textit{blob}
    \\
    int32\_t\quad\textit{width}
    \\
    int32\_t\quad\textit{height}

    \item \textbf{Returns:}

    void (saves data into \textit{channels})

    \item \textbf{Description:}

    The \textit{blob} object contains concatenated mulit-channel data
    \\
    The \textit{channels} object is empty to begin with
    \\
    This function converts the raw data in a Caffe blob into a 'container of channels' (vector of openCV matrices)
    \\
    The \textit{width} and \textit{heigth} parameters let the program know what the dimensions of the channels are in the \textit{blob}
    \\
    The extracted information from \textit{blobs} is saved into the \textit{channels} vector
    \\\\
    This function is just used as a helper for other functions
\end{itemize}


\subsubsection{PE function 2: void copy\_image\_to\_input\_blob(args)}
\begin{itemize}
    \item \textbf{Expected Arguments:}

    caffe::Net \textless float\textgreater\quad\textit{heatmap\_net}
    \\
    cv::Mat\quad\textit{image}

    \item \textbf{Returns:}

    void (saves data into \textit{heatmap\_net})

    \item \textbf{Description:}

    This function converts the \textit{image} object from OpenCV BGR format to 32-bit-floating point RGB format and copies the image to the input blob of \textit{heatmap\_net}. It lso divides the input layer of the \textit{heatmap\_net} from a multi-channel array into several single-channel arrays by calling a helper function
    \\
    \textit{image} is the image that will serve as the input layer to the caffe network
    \\
    \textit{heatmap\_net} is the caffe network that will get its input layer filled with the RGB pixel data from \textit{image}
    \\\\
    This function makes a call to \textbf{PE-cpp function 1} when it splits up the newly updated image in \textit{heatmap\_net}'s input layer into several 'input\_channels'
\end{itemize}


\subsubsection{PE function 3: void get\_joints\_from\_network(args)}
\begin{itemize}
    \item \textbf{Expected Arguments:}

    cv::Point \quad\textit{*joints}
    \\
    cv::Size \quad\textit{channel\_size}
    \\
    caffe::Net\textless float\textgreater\quad\textit{heatmap\_net}

    \item \textbf{Returns:}

    void (saves data into \textit{joints})

    \item \textbf{Description:}

    This function uses the \textit{heatmap\_net}'s "conv5\_fusion" layer to get a set of joint locations for that heatmap.
    \\
    The joint locations get saved into \textit{*joints}
    \\
    The \textit{channel\_size} is used to maintain the accuracy of the position of the joints relative to the image as the image matrix is resized multiple times
    \\\\
    This function makes a call to \textbf{PE-cpp function 1} when it uses the \textit{heatmap\_net} to save all of the joint locations in a 'joint\_channel' vector of cv::Mat objects
\end{itemize}


\subsubsection{PE function 4: void draw\_skeleton(args)}
\begin{itemize}
    \item \textbf{Expected Arguments:}

    cv::Mat \quad\textit{image}
    \\
    cv::Point \quad\textit{*joints}

    \item \textbf{Returns:}

    void (saves image data into \textit{image})

    \item \textbf{Description:}

    This function uses the joint\_locations described in \textit{*joints} to draw an upper-body skeleton on the \textit{image} matrix passed in.
    \\
    \textit{image} is the image to draw the skeleton overlay on
    \\
    \textit{*joints} contain the locations for the set of joints (wrists, elbows, shoulders and head)
\end{itemize}

\subsubsection{PE function 5: std::unique\_ptr\textless caffe::Net\textless float\textgreater\textgreater init\_pose\_estimator\_network(args)}
\begin{itemize}
    \item \textbf{Expected Arguments:}

    std::string\quad\textit{model}
    \\
    std::string\quad\textit{trained\_weights}

    \item \textbf{Returns:}

    \textbf{std::unique\_ptr\textless caffe::Net\textless float\textgreater\textgreater} heatmap\_net
    \\
    pointer to an object that represents a whole caffe network

    \item \textbf{Description:}

    This function creates a Caffe network and copies over the trained layers from a given option for \textit{trained\_weights}
    \\
    For this application, a caffe network is initialized with the default settings:
    \\
    (\textit{model} = 'MODEL\_DEFAULT', \textit{trained\_weights} = TRAINED\_WEIGHTS\_DEFAULT)
\end{itemize}


\subsubsection{PE function 6: void image\_pose\_overlay(args)}
\begin{itemize}
    \item \textbf{Expected Arguments:}

    caffe::Net\textless float\textgreater\quad\textit{heatmap\_net}
    \\
    cv::Mat\quad\textit{image}

    \item \textbf{Returns:}

    void (saves to \textit{image})

    \item \textbf{Description:}

    This function processes the \textit{image} passed in using the \textit{heatmap\_net} to draw a skeleton on the openCV Matrix
    \\\\
    Details on the actions taken by the function:
    \begin{enumerate}
        \item Resizes \textit{image} to 256x256

        \item calls \textbf{PE function 2} to copy the image into the input layer of the \textit{heatmap\_net}

        \item after allowing the network to extract some data, declares an array object of cv::Point called \textit{joints}

        \item calls \textbf{PE function 3} to load in joint locations into \textit{joints}

        \item converts image to a format so that it can be drawn on by the program

        \item calls \textbf{PE function 4} to draw the skeleton overlay on top of the \textit{image}
    \end{enumerate}
\end{itemize}


\subsubsection{PE function 7: void square\_image\_with\_borders(args)}
\begin{itemize}
    \item \textbf{Expected Arguments:}

    cv::Mat\quad\textit{image\_mat}

    \item \textbf{Returns:}

    void (saves image data to \textit{image\_mat})

    \item \textbf{Description:}

    If the image's dimensions do not fit a square (length != width), this function makes it so that the image dimensions are expanded so that it fits into a square
    \\
    This is to avoid distorting the image drastically when the image is resized to 256x256
    \\\\
    This is just a simple helper function to pre-process the image
\end{itemize}



\subsubsection{PE function 8: int32\_t estimate\_pose\_from\_c(args)}
\begin{itemize}
    \item \textbf{Expected Arguments:}

    void\quad\textit{*image}
    \\
    uint32\_t\quad\textit{*size\_bytes}
    \\
    unit32\_t\quad\textit{max\_size\_bytes}

    \item \textbf{Returns:}

    if \textit{image} is processed and saved, returns:\\\textbf{int32\_t size}
    (describing size of file that was uploaded)
    \\\\
    else returns error object

    \item \textbf{Description:}

    This function is the key method in this file. This function overwrites the contents of the memory allocated for \textit{*image} so that a given image is updated to show the skeleton overlay on that image
    \\\\
    This function creates a new Caffe network and calls a lot of helper functions needed to process the \textit{image}
    \\\\
    Details on the actions taken by the function:
    \begin{enumerate}
        \item calls \textbf{PE function 5} to create a new caffe Network, stores new network in \textit{heatmap\_net}

        \item uses \textit{*image} and \textit{*size\_bytes} to create a cv::InputArray object to represent the uploaded image

        \item creates a cv::Mat image matrix, and decodes the contents of the cv::InputArray object into the newly created matrix object

        \item calls \textbf{PE function 7} to pre-process the image matrix to reaffirm that the image is square

        \item calls \textbf{PE function 6} using \textit{heatmap\_net} and the image matrix so that the image\_matrix includes the skeleton overlay

        \item converts and compresses the image\_matrix into a png

        \item finally, overwrites the contents of \textit{*image} with the newly created image that has the pose estimation 'skeleton overlay'
    \end{enumerate}

\end{itemize}




\section{Features In Development}

\subsection{Video Estimation C++ Program}

...
\subsection{Tensorflow}

..just mention to the possibility of using it...

\subsection{Standalone HTTP Server}

...

\chapter{Communication Protocol}

...just mention to HTTP...


\chapter{Development Details}

\section{Languages}
\section{Software}
\section{Hardware}


\end{document}
