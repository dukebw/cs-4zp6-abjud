\documentclass{scrreprt}

\usepackage{xcolor} % for different colour comments
\usepackage{tabto}
\usepackage{mdframed}
\mdfsetup{nobreak=true}
\usepackage{xkeyval}
\usepackage{tabularx}
\usepackage{booktabs}
\usepackage{hyperref}
\hypersetup{
    colorlinks,
    citecolor=black,
    filecolor=black,
    linkcolor=red,
    urlcolor=blue
}
\usepackage[skip=2pt, labelfont=bf]{caption}
\usepackage{titlesec}
\usepackage{graphicx}
\usepackage[section]{placeins}
\graphicspath{ {image/} }

\titleformat{\paragraph}
{\normalfont\normalsize\bfseries}{\theparagraph}{1em}{}
\titlespacing*{\paragraph}
{0pt}{3.25ex plus 1ex minus .2ex}{1.5ex plus .2ex}


%% Comments
\newif\ifcomments\commentstrue

\ifcomments
\newcommand{\authornote}[3]{\textcolor{#1}{[#3 ---#2]}}
\newcommand{\todo}[1]{\textcolor{red}{[TODO: #1]}}
\else
\newcommand{\authornote}[3]{}
\newcommand{\todo}[1]{}
\fi

\newcommand{\wss}[1]{\authornote{magenta}{SS}{#1}}
\newcommand{\ds}[1]{\authornote{blue}{DS}{#1}}


%% The following are used for pretty printing of events and requirements
\makeatletter

\define@cmdkey      [TP] {test}     {name}       {}
\define@cmdkey      [TP] {test}     {desc}       {}
\define@cmdkey      [TP] {test}     {type}       {}
\define@cmdkey      [TP] {test}     {init}       {}
\define@cmdkey      [TP] {test}     {input}      {}
\define@cmdkey      [TP] {test}     {output}     {}
\define@cmdkey      [TP] {test}     {pass}       {}
\define@cmdkey      [TP] {test}     {user}       {}
\define@cmdkey      [TP] {test}     {reqnum}     {}


\newcommand{\getCurrentSectionNumber}{%
  \ifnum\c@section=0 %
  \thechapter
  \else
  \ifnum\c@subsection=0 %
  \thesection
  \else
  \ifnum\c@subsubsection=0 %
  \thesubsection
  \else
  \thesubsubsection
  \fi
  \fi
  \fi
}

\newcounter{TestNum}

\@addtoreset{TestNum}{section}
\@addtoreset{TestNum}{subsection}
\@addtoreset{TestNum}{subsubsection}

\newcommand{\testauto}[1]{
\setkeys[TP]{test}{#1}
\refstepcounter{TestNum}
\begin{mdframed}[linewidth=1pt]
\begin{tabularx}{\textwidth}{@{}p{3cm}X@{}}
{\bf Test \getCurrentSectionNumber.\theTestNum:} & {\bf \cmdTP@test@name}\\[\baselineskip]
{\bf Description:} & \cmdTP@test@desc\\[0.5\baselineskip]
{\bf Type:} & \cmdTP@test@type\\[0.5\baselineskip]
{\bf Initial State:} & \cmdTP@test@init\\[0.5\baselineskip]
{\bf Input:} & \cmdTP@test@input\\[0.5\baselineskip]
{\bf Output:} & \cmdTP@test@output\\[0.5\baselineskip]
{\bf Pass:} & \cmdTP@test@pass\\[0.5\baselineskip]
{\bf Req. \#:} & \cmdTP@test@reqnum
\end{tabularx}
\end{mdframed}
}

\newcommand{\testmanual}[1]{
\setkeys[TP]{test}{#1}
\refstepcounter{TestNum}
\begin{mdframed}[linewidth=1pt]
\begin{tabularx}{\textwidth}{@{}p{3cm}X@{}}
{\bf Test \getCurrentSectionNumber.\theTestNum:} & {\bf \cmdTP@test@name}\\[\baselineskip]
{\bf Description:} & \cmdTP@test@desc\\[0.5\baselineskip]
{\bf Type:} & \cmdTP@test@type\\[0.5\baselineskip]
{\bf Testers:} & \cmdTP@test@user\\[0.5\baselineskip]
{\bf Pass:} & \cmdTP@test@pass\\[0.5\baselineskip]
{\bf Req. \#:} & \cmdTP@test@reqnum
\end{tabularx}
\end{mdframed}
}

\makeatother

\newcommand{\ZtoT}{
\begin{tabularx}{3.85cm}{@{}p{0.35cm}p{0.35cm}p{0.35cm}p{0.35cm}p{0.35cm}p{0.35cm}p{0.35cm}p{0.35cm}p{0.35cm}p{0.35cm}p{0.35cm}@{}}
0 & 1 & 2 & 3 & 4 & 5 & 6 & 7 & 8 & 9 & 10
\end{tabularx}
}

\begin{document}
\title{\bf Text to Motion Database\\[\baselineskip]\Large Design Document}
\author{Brendan Duke\\Andrew Kohnen\\Udip Patel\\David Pitkanen\\Jordan Viveiros}
\date{\today}

\maketitle

\pagenumbering{roman}
\tableofcontents
% \listoftables
% \listoffigures


\begin{table}[bp]
\caption*{\bf Revision History}
\begin{tabularx}{\textwidth}{p{3.5cm}p{2cm}X}
\toprule {\bf Date} & {\bf Version} & {\bf Notes}\\
\midrule
January 5, 2017 & 0.0 & File created\\
\bottomrule
\end{tabularx}
\end{table}

\newpage

\pagenumbering{arabic}

\chapter{User Experience}

\section{User Journey}

\section{Home Page}

\section{About}

\section{Contact}

\section{Sign Up}

\section{Login}

\section{Navigation Bar Links}

\section{Text to Motion}

\subsection{Search}

\subsection{Search Results}

\section{Image Pose Draw}

\subsection{Create}

\subsection{Edit/Details}

\subsection{View Uploads}





\chapter{Database Structure}

\section{Database Schema}
\section{Table Description}


\chapter{Module Decomposition}

\section{Text To Motion - ASP.NET Application}


\subsection{Overview}

...might be a good place to mention .net MVC. Mention of how sections will be about Models, Controllers (referencing views)...

\subsection{Dependency Injection}

...describe anything important that is in project.json...

\subsection{Models}

    \begin{itemize}
        \item \textbf User

        ...

        \item \textbf ...
    \end{itemize}


\subsection{HomeController}

    \begin{itemize}
        \item \textbf FunctionName:

        ...description of func...

        ...reference any models used...

        ...reference the view that is returned...
    \end{itemize}

\subsection{AccountController}

...

\subsection{ManageController}

...

\subsection{ImagePoseDrawController}

...

\subsection{TextToMotionController}

...

\break

\section{Flowing Convnets - Human Pose Estimation}

\subsection{Overview}

This component of the project actually renders the skeleton overlay onto an image submitted to the website.
\\
Mapping out the joints of a person in an image requires the use of image manipulation and deep learning libraries. As of now, this proces is implemented with \textbf{Caffe} and \textbf{OpenCV} in \textbf{C++}.
\\
\\
**The parameters for functions given below will reference 'caffe' and 'cv' types in c++



\subsection{Shared Object File}

The website is able to take an uploaded image and process it by using a shared object file (.so). The web app can make function calls to functions in the shared object file and pass in images as the parameters.


\subsection{Pose Estimation C Program}

To create the shared object file, the C++ function that interfaces with Caffe and OpenCV is extended into a C function.
\\\\
The C file \textbf{"estimate\_pose\_wrapper.c"} just contains one function.
\\\\
\textbf{function: estimate\_pose\_wrapper: (args)}
\begin{itemize}
    \item \textbf{Expects Arguments:}

    void * image

    uint32\_t *size\_bytes

    uint32\_t max\_size\_Bytes

    \item \textbf{Returns:}

    if image is processed, returns \textbf{int32\_t size}
    \\(describing size of file that was uploaded)
    \\\\
    else returns error object

    \item \textbf{Description:}

    this function makes a call to a C++ function in \textbf{"estimate\_pose.c"} which has a defined function prototype for C programs. This wrapper function returns the result of that C++ function call.
    \\
    The C++ funciton also has the same args as this function
    \\\\
    \textbf{referenced function name: estimate\_pose\_from\_c(args)}

\end{itemize}


\subsection{Pose Estimation C++ Program}

...

\section{**Development**}

\subsection{Video Estimation C++ Program}

...
\subsection{Tensorflow}

..just mention to the possibility of using it...

\subsection{Standalone HTTP Server}

...

\chapter{Communication Protocol}

...just mention to HTTP...


\chapter{Development Details}

\section{Languages}
\section{Software}
\section{Hardware}


\end{document}
